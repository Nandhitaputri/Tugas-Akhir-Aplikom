\documentclass{article}

\usepackage{eumat}

\begin{document}
\begin{eulernotebook}
\begin{eulercomment}
Nama : Nandhita Putri Shalsabila\\
NIM : 23030630075\\
Kelas : Matematika B 2023

\end{eulercomment}
\eulersubheading{Menggambar Grafik 2D dengan EMT}
\begin{eulercomment}
Notebook ini menjelaskan tentang cara menggambar berbagaikurva dan
grafik 2D dengan software EMT. EMT menyediakan fungsi plot2d() untuk
menggambar berbagai kurva dan grafik dua dimensi (2D).\\
\end{eulercomment}
\eulersubheading{Basic Plots}
\begin{eulercomment}
Ada fungsi plot yang sangat mendasar. Terdapat koordinat layar yang
selalu berkisar antara 0 hingga 1024 di setiap sumbu, tidak peduli
apakah layarnya berbentuk persegi atau tidak. Semut terdapat koordinat
plot, yang dapat diatur dengan setplot(). Pemetaan antar koordinat
bergantung pada jendela plot saat ini. Misalnya, shrinkwindow()
default menyisakan ruang untuk label sumbu dan judul plot.

Dalam contoh ini, kita hanya menggambar beberapa garis acak dengan
berbagai warna. Untuk rincian tentang fungsi-fungsi ini, pelajari
fungsi inti EMT.
\end{eulercomment}
\begin{eulerprompt}
>clg; // clear screen
\end{eulerprompt}
\begin{eulercomment}
Perintah ini untuk membersihkan layar grafik atau jendela grafik.
\end{eulercomment}
\begin{eulerprompt}
>window(0,0,1024,1024); // use all of the window
\end{eulerprompt}
\begin{eulercomment}
Perintah window(0,0,1024,1024);, untuk mengatur jendela grafik atau
tampilan utama agar memanfaatkan seluruh ruang dengan ukuran 1024x1024
piksel dan mulai dari posisi (0,0) di layar.
\end{eulercomment}
\begin{eulerprompt}
>setplot(0,1,0,1); // set plot coordinates
\end{eulerprompt}
\begin{eulercomment}
Perintah setplot(0,1,0,1);, untuk mengatur sumbu X untuk berkisar dari
0 hingga 1 dan sumbu Y untuk berkisar dari 0 hingga 1
\end{eulercomment}
\begin{eulerprompt}
>hold on; // start overwrite mode
\end{eulerprompt}
\begin{eulercomment}
Perintah ini untuk menambahkan beberapa plot atau elemen grafis ke
dalam satu jendela grafik tanpa menghapus plot sebelumnya.
\end{eulercomment}
\begin{eulerprompt}
>n=100; X=random(n,2); Y=random(n,2);  // get random points
\end{eulerprompt}
\begin{eulercomment}
n = 100;: Menetapkan variabel n dengan nilai 100. Ini menentukan
jumlah baris yang akan digunakan dalam matriks yang akan dihasilkan.

X = random(n,2);: Menghasilkan matriks X dengan ukuran 100x2 yang
berisi angka acak. Setiap elemen di X adalah angka acak antara 0 dan
1.

Y = random(n,2);: Menghasilkan matriks Y dengan ukuran 100x2 yang juga
berisi angka acak antara 0 dan 1.
\end{eulercomment}
\begin{eulerprompt}
>colors=rgb(random(n),random(n),random(n)); // get random colors
\end{eulerprompt}
\begin{eulercomment}
random(n): Menghasilkan vektor atau matriks dengan n elemen yang
berisi angka acak antara 0 dan 1. Misalnya, jika n = 100, maka
random(n) akan menghasilkan vektor dengan 100 elemen acak dalam
rentang [0,1].

rgb(random(n), random(n), random(n)): Menggunakan tiga vektor acak
dari fungsi random(n) untuk mengatur komponen merah, hijau, dan biru
dari warna. Setiap komponen (merah, hijau, biru) adalah vektor dengan
n elemen acak yang mengindikasikan intensitas warna dalam rentang
[0,1].\\
colors: Menyimpan hasil dari rgb(random(n), random(n), random(n)),
yang merupakan matriks warna dengan n baris, di mana setiap baris
berisi nilai RGB acak. Setiap baris dari colors mewakili satu warna
acak.
\end{eulercomment}
\begin{eulerprompt}
>loop 1 to n; color(colors[#]); plot(X[#],Y[#]); end; // plot
\end{eulerprompt}
\begin{eulercomment}
Perintah ini digunakan untuk memplot n titik, di mana setiap titik
memiliki koordinat (X[i], Y[i]) dan warna yang berbeda sesuai dengan
warna acak yang telah dihasilkan dan disimpan dalam variabel colors.
Setiap iterasi dari loop akan memplot satu titik dengan warna yang
sesuai, sehingga menghasilkan visualisasi di mana setiap titik mungkin
memiliki warna yang berbeda.
\end{eulercomment}
\begin{eulerprompt}
>hold off; // end overwrite mode
\end{eulerprompt}
\begin{eulercomment}
perintah \textgreater{}hold off; untuk memastikan bahwa setiap kali grafik baru
dibuat, grafik sebelumnya dihapus dan tidak ditampilkan lagi.
\end{eulercomment}
\begin{eulerprompt}
>insimg; // insert to notebook
\end{eulerprompt}
\eulerimg{27}{images/Plot2D_Nandhita Putri Shalsabila-001.png}
\begin{eulercomment}
Perintah \textgreater{}insimg; untuk menyisipkan gambar grafik atau visualisasi
yang baru saja dihasilkan ke dalam notebook. Setelah menghasilkan
grafik atau visualisasi, perintah ini akan menambahkannya sebagai
gambar yang dapat dilihat di bagian notebook.
\end{eulercomment}
\begin{eulerprompt}
>reset;
\end{eulerprompt}
\begin{eulercomment}
perintah ini mengembalikan EMT ke keadaan seperti baru dibuka,
sehingga Anda dapat memulai sesi baru tanpa harus menutup dan membuka
ulang aplikasi. 

Grafik perlu ditahan, karena perintah plot() akan menghapus jendela
plot.

Untuk menghapus semua yang kami lakukan, kami menggunakan reset().

Untuk menampilkan gambar hasil plot di layar notebook, perintah
plot2d() dapat diakhiri dengan titik dua (:). Cara lain adalah
perintah plot2d() diakhiri dengan titik koma (;), kemudian menggunakan
perintah insimg() untuk menampilkan gambar hasil plot.

Contoh lain, kita menggambar plot sebagai sisipan di plot lain. Hal
ini dilakukan dengan mendefinisikan jendela plot yang lebih kecil.
Perhatikan bahwa jendela ini tidak memberikan ruang untuk label sumbu
di luar jendela plot. Kita harus menambahkan beberapa margin untuk ini
sesuai kebutuhan. Perhatikan bahwa kita menyimpan dan memulihkan
jendela penuh, dan menahan plot saat ini sementara kita memplot inset.
\end{eulercomment}
\begin{eulerprompt}
>plot2d("x^3-x");
\end{eulerprompt}
\begin{eulercomment}
Perintah ini akan menghasilkan grafik kurva dari fungsi y = x\textasciicircum{}3 - x\\
di bidang koordinat 2D.
\end{eulercomment}
\begin{eulerprompt}
>xw=200; yw=100; ww=300; hw=300;
\end{eulerprompt}
\begin{eulercomment}
Jendela grafik akan diposisikan pada koordinat (200, 100) di layar.\\
Jendela grafik akan memiliki ukuran lebar 300 dan tinggi 300 piksel.\\
engan kata lain, perintah ini untuk mengatur ukuran dan posisi jendela
grafik yang akan muncul di layar.
\end{eulercomment}
\begin{eulerprompt}
>ow=window();
\end{eulerprompt}
\begin{eulercomment}
Perintah \textgreater{}ow=window(); untuk mendapatkan informasi tentang posisi dan
ukuran jendela grafik saat ini.
\end{eulercomment}
\begin{eulerprompt}
>window(xw,yw,xw+ww,yw+hw);
\end{eulerprompt}
\begin{eulercomment}
Perintah tersebut akan memindahkan jendela grafik ke koordinat (200,
100) di layar dengan ukuran lebar 300 piksel dan tinggi 300 piksel.
\end{eulercomment}
\begin{eulerprompt}
>hold on;
\end{eulerprompt}
\begin{eulercomment}
Perintah \textgreater{}hold on; digunakan untuk menahan grafik saat ini sehingga
grafik baru yang ditambahkan akan digabungkan dengan grafik yang sudah
ada tanpa menghapus grafik sebelumnya.
\end{eulercomment}
\begin{eulerprompt}
>barclear(xw-50,yw-10,ww+60,ww+60);
\end{eulerprompt}
\begin{eulercomment}
xw-50: Koordinat x dari pojok kiri atas area yang ingin dihapus,
dikurangi 50 piksel dari xw (posisi horizontal jendela grafik).

w-10: Koordinat y dari pojok kiri atas area yang ingin dihapus,
dikurangi 10 piksel dari yw (posisi vertikal jendela grafik).

w+60: Lebar area yang akan dihapus, ditambah 60 piksel dari ww (lebar
jendela grafik).

w+60: Tinggi area yang akan dihapus, sama dengan lebar tambahan,
sehingga membentuk area persegi atau bujur sangkar.

Dengan perintah ini, dapat menghapus area tertentu dari jendela grafik
yang mungkin diperlukan untuk membersihkan ruang sebelum menggambar
atau mengubah grafik yang ada.
\end{eulercomment}
\begin{eulerprompt}
>plot2d("x^4-x",grid=6):
\end{eulerprompt}
\eulerimg{27}{images/Plot2D_Nandhita Putri Shalsabila-002.png}
\begin{eulercomment}
plot2d("x\textasciicircum{}4-x"); untuk membuat grafik fungsi y=x\textasciicircum{}4-x dalam sistem
koordinat 2D.\\
rid=6: Menentukan interval grid atau skala grid pada grafik. Angka 6
di sini menunjukkan jarak atau interval garis-garis grid pada grafik.
Misalnya, jika grid=6, maka garis grid akan muncul setiap 6 unit pada
sumbu.
\end{eulercomment}
\begin{eulerprompt}
>hold off;
\end{eulerprompt}
\begin{eulercomment}
Dengan perintah \textgreater{}hold off;, mode hold dinonaktifkan, dan setiap grafik
baru yang dibuat akan menggantikan grafik sebelumnya. Dengan kata
lain, grafik sebelumnya akan dihapus, dan hanya grafik yang baru akan
ditampilkan.
\end{eulercomment}
\begin{eulerprompt}
>window(ow);
\end{eulerprompt}
\begin{eulercomment}
Perintah \textgreater{}window(ow); digunakan untuk mengatur jendela grafik ke\\
posisi dan ukuran yang telah disimpan sebelumnya dalam variabel ow.

Plot dengan banyak gambar dicapai dengan cara yang sama. Ada
fungsiutilitas figure() untuk ini.

\end{eulercomment}
\eulersubheading{Contoh soal}
\begin{eulerprompt}
>plot2d("x^4-x",grid=5):
\end{eulerprompt}
\eulerimg{27}{images/Plot2D_Nandhita Putri Shalsabila-003.png}
\begin{eulercomment}
\end{eulercomment}
\eulersubheading{Aspek Plot}
\begin{eulercomment}
Plot default menggunakan jendela plot persegi. Anda dapat mengubahnya
dengan fungsi aspek(). Jangan lupa untuk mengatur ulang aspeknya
nanti. Anda juga dapat mengubah default ini di menu dengan "Set
Aspect" ke rasio aspek tertentu atau ke ukuran jendela grafik saat
ini.

Tapi Anda juga bisa mengubahnya untuk satu plot. Untuk ini, ukuran
area plot saat ini diubah, dan jendela diatur sehingga label memiliki
cukup ruang.
\end{eulercomment}
\begin{eulerprompt}
>aspect(2); // rasio panjang dan lebar 2:1
\end{eulerprompt}
\begin{eulercomment}
aspect(2); // rasio panjang dan lebar 2:1: Perintah ini mengatur rasio
aspek dari plot yang dihasilkan. Dengan aspect(2), maka dapat mengatur
rasio panjang dan lebar plot menjadi 2:1, yang berarti lebar plot akan
dua kali lipat dari tingginya. Ini berguna untuk memastikan bahwa
grafik ditampilkan dengan proporsi yang sesuai sehingga data lebih
mudah dianalisis.
\end{eulercomment}
\begin{eulerprompt}
>plot2d(["sin(x)","cos(x)"],0,2pi):
\end{eulerprompt}
\eulerimg{13}{images/Plot2D_Nandhita Putri Shalsabila-004.png}
\begin{eulercomment}
plot2d(["sin(x)","cos(x)"],0,2pi);: Perintah ini menghasilkan grafik
2D dari dua fungsi, yaitu sin(x) dan cos(x), dalam rentang 0 hingga
2pi. plot2d adalah fungsi yang digunakan untuk membuat plot dua
dimensi, dan parameter [“sin(x)”, “cos(x)”] menentukan fungsi yang
akan diplot. Rentang 0 hingga 2pi menunjukkan domain x untuk
fungsi-fungsi tersebut.
\end{eulercomment}
\begin{eulerprompt}
>aspect();
>reset;
\end{eulerprompt}
\begin{eulercomment}
Fungsi reset() mengembalikan default plot termasuk rasio aspek.\\
\begin{eulercomment}
\eulerheading{Plot 2D di Euler}
\begin{eulercomment}
EMT Math Toolbox memiliki plot dalam 2D, baik untuk data maupun
fungsi. EMT menggunakan fungsi plot2d. Fungsi ini dapat memplot fungsi
dan data.

Dimungkinkan untuk membuat plot di Maxima menggunakan Gnuplot atau
dengan Python menggunakan Math Plot Lib.

Euler dapat membuat plot 2D

- ekspresi\\
- fungsi, variabel, atau kurva berparameter,\\
- vektor nilai x-y,\\
- awan titik di pesawat,\\
- kurva implisit dengan level atau wilayah level.\\
- Fungsi kompleks

Gaya plot mencakup berbagai gaya untuk garis dan titik, plot batang,
dan plot berbayang.

\begin{eulercomment}
\eulerheading{Plot Ekspresi atau Variabel}
\begin{eulercomment}
Ekspresi tunggal dalam "x" (misalnya "4*x\textasciicircum{}2") atau nama suatu fungsi
(misalnya "f") menghasilkan grafik fungsi tersebut.

Berikut adalah contoh paling dasar, yang menggunakan rentang default
dan menetapkan rentang y yang tepat agar sesuai dengan plot fungsinya.

Catatan: Jika Anda mengakhiri baris perintah dengan titik dua ":",
plot akan dimasukkan ke dalam jendela teks. Jika tidak, tekan TAB
untuk melihat plot jika jendela plot tertutup.
\end{eulercomment}
\begin{eulerprompt}
>plot2d("x^2"):
\end{eulerprompt}
\eulerimg{27}{images/Plot2D_Nandhita Putri Shalsabila-005.png}
\begin{eulercomment}
plot2d("x\textasciicircum{}2"); akan menghasilkan grafik dari fungsi x\textasciicircum{}2 dalam rentang
default, menampilkan parabola yang memotong sumbu-y di titik (0,0) dan
membesar ke arah sumbu-x positif dan negatif.
\end{eulercomment}
\begin{eulerprompt}
>aspect(1.5); plot2d("x^3-x"):
\end{eulerprompt}
\eulerimg{17}{images/Plot2D_Nandhita Putri Shalsabila-006.png}
\begin{eulercomment}
Perintah aspect(1.5); plot2d("x\textasciicircum{}3-x"); akan mengatur rasio aspek plot
menjadi 1.5:1 dan kemudian menggambar grafik dari fungsi x\textasciicircum{}3-x. Grafik
ini akan menampilkan bentuk kurva dari fungsi tersebut, dan rasio
aspek yang diatur akan mempengaruhi bagaimana grafik tersebut terlihat
dalam hal proporsinya.
\end{eulercomment}
\begin{eulerprompt}
>a:=5.6; plot2d("exp(-a*x^2)/a"); insimg(30); // menampilkan gambar hasil plot setinggi 25 baris
\end{eulerprompt}
\eulerimg{17}{images/Plot2D_Nandhita Putri Shalsabila-007.png}
\begin{eulercomment}
Perintah a:=5.6; plot2d("exp(-a*x\textasciicircum{}2)/a"); insimg(30);\\
Mendefinisikan nilai a sebagai 5.6 kemudian membuat plot 2D dari
fungsi (exp(-5.6.x\textasciicircum{}2)/5.6) dan menampilkan hasil plot dengan ukuran
gambar yang tinggi 25 baris (atau sesuai dengan spesifikasi insimg).

Dari beberapa contoh sebelumnya Anda dapat melihat bahwa aslinya
gambar plot menggunakan sumbu X dengan rentang nilai dari -2 sampai
dengan 2. Untuk mengubah rentang nilai X dan Y, Anda dapat menambahkan
nilai-nilai batas X (dan Y) di belakang ekspresi yang digambar.

Rentang plot diatur dengan parameter yang ditetapkan sebagai berikut

- a,b: rentang x (default -2,2)\\
- c,d: rentang y (default: skala dengan nilai)\\
- r: alternatifnya radius di sekitar pusat plot\\
- cx,cy: koordinat pusat plot (default 0,0)
\end{eulercomment}
\begin{eulerprompt}
>plot2d("x^3-x",-1,2):
\end{eulerprompt}
\eulerimg{17}{images/Plot2D_Nandhita Putri Shalsabila-008.png}
\begin{eulercomment}
Perintah plot2d("x\textasciicircum{}3-x",-1,2); untuk membuat grafik dari fungsi\\
x\textasciicircum{}3-x dengan sumbu-x yang terbentang dari -1 hingga 2. Hal ini untuk
melihat perilaku fungsi dalam rentang x yang spesifik tersebut. Grafik
ini akan menunjukkan bagaimana fungsi kubik ini berubah dalam interval
yang ditentukan.

\end{eulercomment}
\begin{eulerprompt}
>plot2d("sin(x)",-2*pi,2*pi): // plot sin(x) pada interval [-2pi, 2pi]
\end{eulerprompt}
\eulerimg{17}{images/Plot2D_Nandhita Putri Shalsabila-009.png}
\begin{eulercomment}
Perintah plot2d("sin(x)", -2*pi, 2*pi); digunakan untuk membuat grafik
2D dari fungsi sin(x) dengan sumbu-x yang terbentang dari -2pi hingga
2pi. Ini akan menampilkan gelombang sinusoidal yang penuh dalam
rentang tersebut,untuk melihat beberapa periode dari fungsi sinus.
\end{eulercomment}
\begin{eulerprompt}
>plot2d("cos(x)","sin(3*x)",xmin=0,xmax=2pi):
\end{eulerprompt}
\eulerimg{17}{images/Plot2D_Nandhita Putri Shalsabila-010.png}
\begin{eulercomment}
Perintah plot2d("cos(x)","sin(3*x)", xmin=0, xmax=2pi); akan
menghasilkan grafik dua fungsi:\\
cos(x) dan sin(3*x), dalam rentang sumbu-x dari 0 hingga 2pi.

Alternatif untuk titik dua adalah perintah insimg(baris), yang
menyisipkan plot yang menempati sejumlah baris teks tertentu.

Dalam opsi, plot dapat diatur agar muncul

- di jendela terpisah yang dapat diubah ukurannya,\\
- di jendela buku catatan.

Lebih banyak gaya dapat dicapai dengan perintah plot tertentu.

Bagaimanapun, tekan tombol tabulator untuk melihat plotnya, jika
tersembunyi.

Untuk membagi jendela menjadi beberapa plot, gunakan perintah
figure(). Dalam contoh, kita memplot x\textasciicircum{}1 hingga x\textasciicircum{}4 menjadi 4 bagian
jendela. gambar(0) mengatur ulang jendela default.
\end{eulercomment}
\begin{eulerprompt}
>reset;
>figure(2,2); ...
>for n=1 to 4; figure(n); plot2d("x^"+n); end; ...
>figure(0):
\end{eulerprompt}
\eulerimg{27}{images/Plot2D_Nandhita Putri Shalsabila-011.png}
\begin{eulercomment}
Perintah reset; figure(2,2); for n=1 to 4; figure(n); plot2d("x\textasciicircum{}"+n);
end; figure(0); untuk melakukan hal berikut:\\
1. Mengatur ulang lingkungan kerja.\\
2. Mengatur ukuran jendela grafik menjadi 2x2 unit.\\
3. Menggunakan loop untuk menghasilkan empat jendela grafik terpisah,
masing-masing menampilkan grafik dari fungsi x\textasciicircum{}n untuk n dari 1 hingga
4.\\
4. Mengembalikan fokus ke jendela grafik utama atau jendela grafik
default.

Di plot2d(), ada gaya alternatif yang tersedia dengan grid=x. Untuk
gambaran umum, kami menampilkan berbagai gaya kisi dalam satu gambar
(lihat di bawah untuk perintah figure()). Gaya grid=0 tidak
disertakan. Ini tidak menunjukkan kisi dan bingkai.
\end{eulercomment}
\begin{eulerprompt}
>figure(3,3); ...
>for k=1:9; figure(k); plot2d("x^3-x",-2,1,grid=k); end; ...
>figure(0):
\end{eulerprompt}
\eulerimg{27}{images/Plot2D_Nandhita Putri Shalsabila-012.png}
\begin{eulercomment}
Perintah ini melakukan hal berikut:\\
1. Mengatur ukuran jendela grafik menjadi 3x3 unit.\\
2. Menggunakan loop untuk menghasilkan sembilan jendela grafik
terpisah, masing-masing menampilkan grafik dari fungsi x\textasciicircum{}3-x dalam
rentang dari -2 hingga 1.\\
3. Mengatur grid untuk setiap jendela grafik dengan intensitas grid
yang berbeda, berdasarkan nilai k dari 1 hingga 9.\\
4. Mengembalikan fokus ke jendela grafik default setelah semua grafik
telah digambar.

Jika argumen pada plot2d() adalah ekspresi yang diikuti oleh empat
angka, angka-angka tersebut adalah rentang x dan y untuk plot
tersebut.

Alternatifnya, a, b, c, d dapat ditentukan sebagai parameter yang
ditetapkan sebagai a=... dll.

Pada contoh berikut, kita mengubah gaya kisi, menambahkan label, dan
menggunakan label vertikal untuk sumbu y.
\end{eulercomment}
\begin{eulerprompt}
>aspect(1.5); plot2d("sin(x)",0,2pi,-1.2,1.2,grid=3,xl="x",yl="sin(x)"):
\end{eulerprompt}
\eulerimg{17}{images/Plot2D_Nandhita Putri Shalsabila-013.png}
\begin{eulercomment}
Perintah ini menghasilkan grafik dari fungsi sin(x) dalam interval
dari 0 hingga 2pi, dengan sumbu-y yang mencakup rentang dari -1.2
hingga 1.2. Grafik ini akan ditampilkan dengan rasio aspek 1.5:1, grid
dengan tingkat kepadatan 3, serta label yang sesuai pada sumbu-x dan
sumbu-y.
\end{eulercomment}
\begin{eulerprompt}
>plot2d("sin(x)+cos(2*x)",0,4pi):
\end{eulerprompt}
\eulerimg{17}{images/Plot2D_Nandhita Putri Shalsabila-014.png}
\begin{eulercomment}
Perintah plot2d("sin(x)+cos(2*x)",0,4pi); untuk membuat grafik dari
fungsi sin(x)+cos(2x) dalam rentang x dari 0 hingga 2pi. Ini
memungkinkan visualisasi pola gelombang yang dihasilkan oleh kombinasi
fungsi sinus dan kosinus dalam interval yang luas.

Gambar yang dihasilkan dengan memasukkan plot ke dalam jendela teks
disimpan di direktori yang sama dengan buku catatan, secara default di
subdirektori bernama "gambar". Mereka juga digunakan oleh ekspor HTML.

Anda cukup menandai gambar apa saja dan menyalinnya ke clipboard
dengan Ctrl-C. Tentu saja, Anda juga dapat mengekspor grafik saat ini
dengan fungsi di menu File.

Fungsi atau ekspresi di plot2d dievaluasi secara adaptif. Agar lebih
cepat, nonaktifkan plot adaptif dengan \textless{}adaptive dan tentukan jumlah
subinterval dengan n=... Hal ini hanya diperlukan pada kasus yang
jarang terjadi.
\end{eulercomment}
\begin{eulerprompt}
>plot2d("sign(x)*exp(-x^2)",-1,1,<adaptive,n=10000):
\end{eulerprompt}
\eulerimg{17}{images/Plot2D_Nandhita Putri Shalsabila-015.png}
\begin{eulercomment}
Perintah plot2d("sign(x)*exp(-x\textasciicircum{}2)",-1,1,\textless{}adaptive,n=10000); untuk
menghasilkan grafik dua dimensi dari fungsi sign(x).exp(-x\textasciicircum{}2) dalam
rentang x dari -1 hingga 1. Grafik ini dibuat dengan menggunakan
metode adaptif untuk penyesuaian titik data dan dengan resolusi tinggi
(10.000 titik data) untuk memastikan detail yang akurat dan
representasi yang halus dari fungsi tersebut.
\end{eulercomment}
\begin{eulerprompt}
>plot2d("x^x",r=1.2,cx=1,cy=1):
\end{eulerprompt}
\eulerimg{17}{images/Plot2D_Nandhita Putri Shalsabila-016.png}
\begin{eulercomment}
Perintah plot2d("x\textasciicircum{}x",r=1.2,cx=1,cy=1); dapat menghasilkan grafik dari
fungsi x\textasciicircum{}x dengan pengaturan tampilan sebagai berikut:\\
1. Rasio aspek diatur dengan r=1.2 untuk mengubah proporsi grafik.\\
2. Grafik dipusatkan pada koordinat (1, 1) dengan cx=1 dan cy=1, yang
bisa mempengaruhi penempatan grafik dalam jendela plot.

Perhatikan bahwa x\textasciicircum{}x tidak ditentukan untuk x\textless{}=0. Fungsi plot2d
menangkap kesalahan ini, dan mulai membuat plot segera setelah
fungsinya ditentukan. Ini berfungsi untuk semua fungsi yang
mengembalikan NAN di luar jangkauan definisinya.
\end{eulercomment}
\begin{eulerprompt}
>plot2d("log(x)",-0.1,2):
\end{eulerprompt}
\eulerimg{17}{images/Plot2D_Nandhita Putri Shalsabila-017.png}
\begin{eulercomment}
Perintah plot2d("log(x)", -0.1, 2); dapat menghasilkan grafik dua
dimensi dari fungsi log(x) dengan rentang sumbu-x dari -0.1 hingga 2.
Dalam praktiknya, grafik akan valid hanya untuk bagian dari sumbu-x di
atas 0, dan mungkin hanya menunjukkan plot dari x yang mendekati 0
hingga 2, dengan bagian dari sumbu-x yang negatif tidak digunakan
dalam grafik.

Parameter square=true (atau \textgreater{}square) memilih rentang y secara otomatis
sehingga hasilnya adalah jendela plot persegi. Perhatikan bahwa secara
default, Euler menggunakan spasi persegi di dalam jendela plot.
\end{eulercomment}
\begin{eulerprompt}
>plot2d("x^3-x",>square):
\end{eulerprompt}
\eulerimg{17}{images/Plot2D_Nandhita Putri Shalsabila-018.png}
\begin{eulercomment}
Perintah plot2d("x\textasciicircum{}3-x", \textgreater{}square); dapat menghasilkan grafik dua
dimensi dari fungsi x\textasciicircum{}3-x dengan jendela plot yang diatur dalam bentuk
persegi, memastikan bahwa grafik tidak terdistorsi dan sumbu-x serta
sumbu-y memiliki skala yang sama.
\end{eulercomment}
\begin{eulerprompt}
>plot2d(''integrate("sin(x)*exp(-x^2)",0,x)'',0,2): // plot integral
\end{eulerprompt}
\eulerimg{17}{images/Plot2D_Nandhita Putri Shalsabila-019.png}
\begin{eulercomment}
Perintah plot2d('integrate("sin(x)*exp(-x\textasciicircum{}2)",0,x)',0,2); untuk
melakukan hal berikut:\\
1. Menghitung integral dari fungsi\\
sin(x).exp(-x\textasciicircum{}2)dari 0 hingga x, menghasilkan fungsi integral
kumulatif.\\
2. Memplot hasil integral ini dalam rentang x dari 0 hingga 2.\\
Ini dapat memvisualisasikan bagaimana nilai integral kumulatif dari
fungsisin(x).exp(-x\textasciicircum{}2)berubah sepanjang interval\\
x dari 0 hingga 2.

Jika Anda memerlukan lebih banyak ruang untuk label y, panggil
shrinkwindow() dengan parameter lebih kecil, atau tetapkan nilai
positif untuk "lebih kecil" di plot2d().
\end{eulercomment}
\begin{eulerprompt}
>plot2d("gamma(x)",1,10,yl="y-values",smaller=6,<vertical):
\end{eulerprompt}
\eulerimg{17}{images/Plot2D_Nandhita Putri Shalsabila-020.png}
\begin{eulercomment}
Perintah plot2d("gamma(x)",1,10,yl="y-values",smaller=6,\textless{}vertical);
untuk melakukan hal berikut:\\
1. Menghasilkan grafik dua dimensi dari fungsi Gamma(x)dalam rentang\\
x dari 1 hingga 10.\\
2. Menetapkan label sumbu-y sebagai "y-values".\\
3. Mengatur ukuran elemen teks dalam grafik menjadi lebih kecil dengan
ukuran font 6.\\
4. Mengubah orientasi teks atau elemen grafik menjadi vertikal.

Ekspresi simbolik juga dapat digunakan karena disimpan sebagai
ekspresi string sederhana.
\end{eulercomment}
\begin{eulerprompt}
>x=linspace(0,2pi,1000); plot2d(sin(5x),cos(7x)):
\end{eulerprompt}
\eulerimg{17}{images/Plot2D_Nandhita Putri Shalsabila-021.png}
\begin{eulercomment}
Perintah x=linspace(0,2pi,1000); plot2d(sin(5x), cos(7x)); untuk
melakukan hal berikut:\\
1. Menghasilkan vektor x dengan 1000 titik dari 0 hingga 2pi
menggunakan linspace.\\
2. Membuat grafik dua dimensi dari fungsi sin(5x) dan cos(7x)
menggunakan vektor x yang dihasilkan, menampilkan kedua fungsi dalam
grafik yang sama.\\
Hal ini untuk melihat bagaimana fungsi sinus dan kosinus dengan
frekuensi berbeda berperilaku dan dibandingkan satu sama lain dalam
interval dari 0 hingga 2pi.
\end{eulercomment}
\begin{eulerprompt}
>a:=5.6; expr &= exp(-a*x^2)/a; // define expression
>plot2d(expr,-2,2): // plot from -2 to 2
\end{eulerprompt}
\eulerimg{17}{images/Plot2D_Nandhita Putri Shalsabila-022.png}
\begin{eulercomment}
Perintah a:=5.6; expr \&= exp(-a*x\textasciicircum{}2)/a; untuk mendefinisikan sebuah
fungsi (exp(-5.6.x\textasciicircum{}2)/5.6) di mana a adalah parameter dalam fungsi
tersebut, dan plot2d(expr, -2, 2); untuk menghasilkan grafik dari
fungsi ini dalam rentang x dari -2 hingga 2.\\
Grafik ini akan menunjukkan bentuk lonceng dari fungsi eksponensial
tereduksi dengan parameter a=5.6, mencakup interval sumbu-x dari -2
hingga 2.

\end{eulercomment}
\begin{eulerprompt}
>plot2d(expr,r=1,thickness=2): // plot in a square around (0,0)
\end{eulerprompt}
\eulerimg{17}{images/Plot2D_Nandhita Putri Shalsabila-023.png}
\begin{eulercomment}
Perintah plot2d(expr, r=1, thickness=2); untuk menghasilkan grafik
dari fungsi yang didefinisikan (expr), dengan tampilan sebagai
berikut:\\
1. r=1: Memastikan bahwa plot ditampilkan dalam format persegi (rasio
aspek 1:1), sehingga sumbu-x dan sumbu-y memiliki skala yang sama.\\
2. thickness=2: Mengatur ketebalan garis grafik menjadi 2 unit,
membuat garis grafik lebih tebal dan lebih jelas terlihat.\\
Dengan ini, grafik fungsi (exp(-5.6.x\textasciicircum{}2)/5.6)\\
kan ditampilkan dalam jendela plot yang berbentuk persegi dengan garis
yang lebih tebal, memudahkan visualisasi fungsi tersebut di sekitar
titik (0,0).
\end{eulercomment}
\begin{eulerprompt}
>plot2d(&diff(expr,x),>add,style="--",color=red): // add another plot
\end{eulerprompt}
\eulerimg{17}{images/Plot2D_Nandhita Putri Shalsabila-024.png}
\begin{eulercomment}
Perintah plot2d(\&diff(expr,x), \textgreater{}add, style="--", color=red); utnuk
melakukan hal berikut:\\
1.Menghitung turunan pertama dari ekspresi expr (yang sebelumnya
didefinisikan sebagai (exp(-5.6.x\textasciicircum{}2)/5.6)\\
2. Menambahkan plot dari turunan fungsi ke grafik yang sudah ada.\\
3. Menggunakan garis putus-putus untuk grafik turunan.\\
4. Mengatur warna garis grafik turunan menjadi merah.\\
Dengan ini, grafik turunan dari fungsi (exp(-5.6.x\textasciicircum{}2)/5.6)\\
akan ditambahkan ke grafik yang sudah ada, menampilkan fungsi asal
bersama dengan grafik dari turunannya dalam warna merah dan gaya garis
putus-putus.
\end{eulercomment}
\begin{eulerprompt}
>plot2d(&diff(expr,x,2),a=-2,b=2,c=-2,d=1): // plot in rectangle
\end{eulerprompt}
\eulerimg{17}{images/Plot2D_Nandhita Putri Shalsabila-025.png}
\begin{eulercomment}
Perintah plot2d(\&diff(expr,x,2), a=-2, b=2, c=-2, d=1); untuk
melakukan hal berikut:\\
1. Menghitung turunan kedua dari fungsi yang didefinisikan sebelumnya
(expr), yang dalam hal ini adalah (exp(-5.6.x\textasciicircum{}2)/5.6).\\
2. Memplot turunan kedua fungsi tersebut dalam rentang sumbu-x dari -2
hingga 2 dan sumbu-y dari -2 hingga 1.\\
Plot ini akan menunjukkan bagaimana kelengkungan dari fungsi
(exp(5.6.x\textasciicircum{}2)/5.6)berubah dalam interval yang ditentukan, memberikan
informasi tentang titik-titik di mana fungsi tersebut cekung atau
cembung dalam rentang yang ditentukan.
\end{eulercomment}
\begin{eulerprompt}
>plot2d(&diff(expr,x),a=-2,b=2,>square): // keep plot square
\end{eulerprompt}
\eulerimg{17}{images/Plot2D_Nandhita Putri Shalsabila-026.png}
\begin{eulercomment}
Perintah plot2d(\&diff(expr,x), a=-2, b=2, \textgreater{}square); untuk melakukan
hal berikut:\\
1.Menghitung turunan pertama dari fungsi yang didefinisikan sebagai
expr, yaitu (exp(-5.6.x\textasciicircum{}2)/5.6).\\
2. Memplot grafik turunan pertama fungsi tersebut dalam rentang
sumbu-x dari -2 hingga 2.\\
3. Menetapkan tampilan grafik dengan rasio aspek 1:1 menggunakan
\textgreater{}square, sehingga grafik ditampilkan dalam format persegi.\\
Dengan pengaturan ini, grafik dari turunan pertama fungsi akan
ditampilkan dengan skala yang konsisten di sumbu-x dan sumbu-y,
memberikan visualisasi yang akurat dari perubahan turunan dalam
rentang yang ditentukan.
\end{eulercomment}
\begin{eulerprompt}
>plot2d("x^2",0,1,steps=1,color=red,n=10):
\end{eulerprompt}
\eulerimg{17}{images/Plot2D_Nandhita Putri Shalsabila-027.png}
\begin{eulercomment}
Perintah plot2d("x\textasciicircum{}2",0,1,steps=1,color=red,n=10); untuk melakukan hal
berikut:\\
1. Menghasilkan grafik dua dimensi dari fungsi x\textasciicircum{}2 dalam rentang
sumbu-x dari 0 hingga 1.\\
2. Menggunakan metode penggambaran yang melibatkan interpolasi linear
antar titik data dengan steps=1.\\
3. Mewarnai grafik dengan warna merah.\\
4. Menggunakan 10 titik data untuk menggambar grafik, memberikan
kekasaran atau kehalusan pada garis grafik.\\
Dengan ini, grafik dari fungsi x\textasciicircum{}2 akan digambar dalam warna merah,
menggunakan 10 titik data dalam rentang yang ditentukan, dan
menghubungkan titik-titik dengan garis lurus.
\end{eulercomment}
\begin{eulerprompt}
>plot2d("x^2",>add,steps=2,color=blue,n=10):
\end{eulerprompt}
\eulerimg{17}{images/Plot2D_Nandhita Putri Shalsabila-028.png}
\begin{eulercomment}
Perintah plot2d("x\textasciicircum{}2", \textgreater{}add, steps=2, color=blue, n=10); untuk
melakukan hal berikut:\\
1. Menambahkan grafik dari fungsi x\textasciicircum{}2 ke grafik yang sudah ada.\\
2. Menggunakan metode penggambaran dengan interpolasi antar dua titik
data.\\
3. Mengatur warna grafik menjadi biru.\\
4. Menggunakan 10 titik data untuk menggambar grafik.\\
Dengan ini, grafik dari fungsi x\textasciicircum{}2 akan ditambahkan ke grafik yang
sudah ada sebelumnya, dengan interpolasi yang melibatkan dua titik
data, menggunakan warna biru, dan dengan resolusi yang ditentukan oleh
10 titik data.

\begin{eulercomment}
\eulerheading{Fungsi dalam satu Parameter}
\begin{eulercomment}
Fungsi plot yang paling penting untuk plot planar adalah plot2d().
Fungsi ini diimplementasikan dalam bahasa Euler di file "plot.e", yang
dimuat di awal program.

Berikut beberapa contoh penggunaan suatu fungsi. Seperti biasa di EMT,
fungsi yang berfungsi untuk fungsi atau ekspresi lain, Anda bisa
meneruskan parameter tambahan (selain x) yang bukan variabel global ke
fungsi dengan parameter titik koma atau dengan kumpulan panggilan.
\end{eulercomment}
\begin{eulerprompt}
>function f(x,a) := x^2/a+a*x^2-x; // define a function
>a=0.3; plot2d("f",0,1;a): // plot with a=0.3
\end{eulerprompt}
\eulerimg{17}{images/Plot2D_Nandhita Putri Shalsabila-029.png}
\begin{eulercomment}
Perintah ini akan menghasilkan grafik dari fungsi f dalam interval [0,
1] dengan parameter a = 0.3.

memplot gambar grafik dengan fungsi f nya , interval x nya titik koma
a nya.
\end{eulercomment}
\begin{eulerprompt}
>plot2d("f",0,1;0.4): // plot with a=0.4
\end{eulerprompt}
\eulerimg{17}{images/Plot2D_Nandhita Putri Shalsabila-030.png}
\begin{eulercomment}
Perintah ini akan menggambarkan grafik dari fungsi f(x, 0.4) dalam
interval [0, 1], untuk melihat bagaimana perubahan nilai a
mempengaruhi bentuk grafik fungsi.
\end{eulercomment}
\begin{eulerprompt}
>plot2d(\{\{"f",0.2\}\},0,1): // plot with a=0.2
\end{eulerprompt}
\eulerimg{17}{images/Plot2D_Nandhita Putri Shalsabila-031.png}
\begin{eulercomment}
Perintah ini akan menghasilkan grafik dari fungsi f(x, 0.2) dalam
interval [0, 1], untuk menganalisis bagaimana nilai a yang berbeda
mempengaruhi grafik fungsi tersebut

memplot grafik fungsi dengan fungsi f nya, parameter atau a \{x nya\}
\end{eulercomment}
\begin{eulerprompt}
>plot2d(\{\{"f(x,b)",b=0.1\}\},0,1): // plot with 0.1
\end{eulerprompt}
\eulerimg{17}{images/Plot2D_Nandhita Putri Shalsabila-032.png}
\begin{eulercomment}
Perintah ini akan menghasilkan grafik dari fungsi f(x, 0.1) dalam
interval [0, 1]. untuk melihat bagaimana fungsi berperilaku dengan
nilai b yang spesifik.

memplot grafik fungsi yang sudah diketahui nilai b(parameter)nya.
\end{eulercomment}
\begin{eulerprompt}
>function f(x) := x^3-x; ...
>plot2d("f",r=1):
\end{eulerprompt}
\eulerimg{17}{images/Plot2D_Nandhita Putri Shalsabila-033.png}
\begin{eulercomment}
Perintah tersebut digunakan untuk mendefinisikan fungsi kubik\\
f(x)=x\textasciicircum{}3-x dan kemudian memplot grafiknya dalam rentang x dari -1
hingga 1.

Berikut ini ringkasan fungsi yang diterima

- ekspresi atau ekspresi simbolik di x\\
- fungsi atau fungsi simbolik dengan nama "f"\\
- fungsi simbolik hanya dengan nama f

Fungsi plot2d() juga menerima fungsi simbolik. Untuk fungsi simbolik,
namanya saja yang berfungsi.
\end{eulercomment}
\begin{eulerprompt}
>function f(x) &= diff(x^x,x)
\end{eulerprompt}
\begin{euleroutput}
  
                              x
                             x  (log(x) + 1)
  
\end{euleroutput}
\begin{eulercomment}
Perintah tersebut mendefinisikan fungsi yang mengembalikan turunan
dari\\
x\textasciicircum{}x ,yaitu f(x)=d/dx(x\textasciicircum{}x)=x\textasciicircum{}x (ln(x)+1).
\end{eulercomment}
\begin{eulerprompt}
>plot2d(f,0,2):
\end{eulerprompt}
\eulerimg{17}{images/Plot2D_Nandhita Putri Shalsabila-034.png}
\begin{eulercomment}
Perintah ini akan memplot grafik dari turunan f(x)=d/dx(x\textasciicircum{}x)untuk x
dalam rentang dari 0 hingga 2.

Tentu saja, untuk ekspresi atau ekspresi simbolik, nama variabel sudah
cukup untuk memplotnya.
\end{eulercomment}
\begin{eulerprompt}
>expr &= sin(x)*exp(-x)
\end{eulerprompt}
\begin{euleroutput}
  
                                - x
                               E    sin(x)
  
\end{euleroutput}
\begin{eulercomment}
Perintah ini, expr mengandung ekspresi simbolik sin(x). e\textasciicircum{}-x, dan
dapat melakukan operasi matematika lebih lanjut dengan ekspresi
tersebut, seperti plotting atau menghitung turunan atau integral.

"/".\\
- Fillcolor : Lihat di atas untuk mengetahui warna yang tersedia.

Warna isian ditentukan oleh argumen "fillcolor", dan pada \textless{}outline
opsional, mencegah menggambar batas untuk semua gaya kecuali gaya
default.
\end{eulercomment}
\begin{eulerprompt}
>plot2d(expr,0,3pi):
\end{eulerprompt}
\eulerimg{17}{images/Plot2D_Nandhita Putri Shalsabila-035.png}
\begin{eulercomment}
Perintah ini akan memplot grafik dari ekspresi sin(x). e\textasciicircum{}-x dalam
rentang x dari 0 hingga sekitar 9.42 (atau 3pi).
\end{eulercomment}
\begin{eulerprompt}
>function f(x) &= x^x;
>plot2d(f,r=1,cx=1,cy=1,color=blue,thickness=2);
>plot2d(&diff(f(x),x),>add,color=red,style="-.-"):
\end{eulerprompt}
\eulerimg{17}{images/Plot2D_Nandhita Putri Shalsabila-036.png}
\begin{eulercomment}
Perintah ini memplot grafik fungsi x\textasciicircum{}x dalam rentang yang ditentukan,
dengan warna biru dan ketebalan garis 2.\\
Kemudian, perintah kedua memplot turunan dari fungsi tersebut
f'(x)=x\textasciicircum{}x (ln(x)+1) di atas grafik yang sama dengan warna merah dan
gaya garis yang berupa titik-titik bercampur dengan garis putus-putus.\\
Hasilnya adalah grafik dari fungsi x\textasciicircum{}x dan grafik turunannya diplot
secara bersamaan untuk perbandingan visual.

Yang biru adalah fungsinya dan yang merah adalah turunan dari
fungsinya.

Untuk gaya garis ada berbagai pilihan.

- gaya="...". Pilih dari "-", "--", "-.", ".", ".-.", "-.-".\\
- Warna: Lihat di bawah untuk warna.\\
- ketebalan: Defaultnya adalah 1.

Warna dapat dipilih sebagai salah satu warna default, atau sebagai
warna RGB.

- 0..15: indeks warna default.\\
- konstanta warna: putih, hitam, merah, hijau, biru, cyan, zaitun,
abu-abu muda, abu-abu, abu-abu tua, oranye, hijau muda, pirus, biru
muda, oranye muda, kuning\\
- rgb(merah,hijau,biru): parameternya real di [0,1].
\end{eulercomment}
\begin{eulerprompt}
>plot2d("exp(-x^2)",r=2,color=red,thickness=3,style="--"):
\end{eulerprompt}
\eulerimg{17}{images/Plot2D_Nandhita Putri Shalsabila-037.png}
\begin{eulercomment}
Perintah ini membuat grafik fungsi e\textasciicircum{}-x2 dengan:\\
Rentang x dari -2 hingga 2\\
Warna garis merah\\
Garis yang lebih tebal (ketebalan 3 unit)\\
Dan garis bergaya putus-putus.\\
Grafik ini akan menunjukkan bentuk lonceng simetris yang dikenal
sebagai fungsi Gaussian atau kurva lonceng, yang sering digunakan
dalam probabilitas dan statistik.

Berikut adalah tampilan warna EMT yang telah ditentukan sebelumnya.
\end{eulercomment}
\begin{eulerprompt}
>aspect(2); columnsplot(ones(1,16),lab=0:15,grid=0,color=0:15):
\end{eulerprompt}
\eulerimg{13}{images/Plot2D_Nandhita Putri Shalsabila-038.png}
\begin{eulercomment}
Rasio aspek grafik diatur menjadi 2 (lebih lebar).\\
Sebuah plot kolom dengan 16 kolom dibuat, semua kolom memiliki tinggi
1.\\
Label pada sumbu-x ditetapkan dari 0 hingga 15, dan warna setiap kolom
berbeda, diambil dari skala warna yang tersedia (warna 0 hingga 15).\\
Grid tidak ditampilkan.\\
Hasil akhirnya adalah plot kolom dengan kolom yang berwarna berbeda,
berlabel dari 0 hingga 15, dan rasio lebar grafik dua kali tinggi.

Tapi Anda bisa menggunakan warna apa saja.
\end{eulercomment}
\begin{eulerprompt}
>columnsplot(ones(1,16),grid=0,color=rgb(0,0,linspace(0,1,15))):
\end{eulerprompt}
\eulerimg{13}{images/Plot2D_Nandhita Putri Shalsabila-039.png}
\begin{eulercomment}
Plot kolom memiliki 16 kolom, semuanya dengan tinggi 1.\\
Warna kolom bervariasi dari biru gelap ke biru terang berdasarkan
gradasi yang dibuat oleh fungsi linspace.\\
Grid dinonaktifkan, sehingga tidak ada garis kisi yang muncul di
grafik.\\
Hasil akhirnya adalah sebuah plot kolom dengan warna biru gradasi yang
berubah dari gelap ke terang di sepanjang kolom, tanpa grid.

\begin{eulercomment}
\eulerheading{Menggambar Beberapa Kurva pada bidang koordinat yang sama}
\begin{eulercomment}
Plot lebih dari satu fungsi (multiple function) ke dalam satu jendela
dapat dilakukan dengan berbagai cara. Salah satu metodenya adalah
menggunakan \textgreater{}add untuk beberapa panggilan ke plot2d secara
keseluruhan, kecuali panggilan pertama. Kami telah menggunakan fitur
ini pada contoh di atas.
\end{eulercomment}
\begin{eulerprompt}
>aspect(); plot2d("cos(x)",r=2,grid=6); plot2d("x",style=".",>add):
\end{eulerprompt}
\eulerimg{27}{images/Plot2D_Nandhita Putri Shalsabila-040.png}
\begin{eulercomment}
aspect(): Mengatur rasio aspek dari grafik agar proporsional.\\
plot2d("cos(x)", r=2, grid=6): Menggambar grafik fungsi cos(x) dengan
rentang sumbu r=2 dan grid berjumlah 6. Ini menunjukkan fungsi cosinus
dalam interval yang ditentukan.\\
plot2d("x", style=".", \textgreater{}add): Menggambar garis lurus\\
y=x dengan gaya titik, dan menggunakan \textgreater{}add untuk menambahkan grafik
ini ke grafik sebelumnya tanpa menghapusnya.\\
Jadi, perintah tersebut secara keseluruhan menggambarkan grafik dari\\
cos(x) dan garis\\
y=x dalam satu gambar.
\end{eulercomment}
\begin{eulerprompt}
>aspect(1.5); plot2d("sin(x)",0,2pi); plot2d("cos(x)",color=blue,style="--",>add):
\end{eulerprompt}
\eulerimg{17}{images/Plot2D_Nandhita Putri Shalsabila-041.png}
\begin{eulercomment}
spect(1.5): Mengatur rasio aspek grafik, sehingga sumbu x dan y
memiliki proporsi 1.5:1, memberikan tampilan yang lebih baik pada
grafik.\\
plot2d("sin(x)", 0, 2pi): Menggambar grafik fungsi sin(x) dalam
interval dari 0 hingga 2pi.\\
plot2d("cos(x)", color=blue, style="--", \textgreater{}add): Menggambar grafik
fungsi\\
cos(x) dengan garis putus-putus (style="--") dan warna biru
(color=blue), serta menambahkan grafik ini ke grafik sebelumnya tanpa
menghapus grafik sin(x).

Salah satu kegunaan \textgreater{}add adalah untuk menambahkan titik pada kurva.
\end{eulercomment}
\begin{eulerprompt}
>plot2d("sin(x)",0,pi); plot2d(2,sin(2),>points,>add):
\end{eulerprompt}
\eulerimg{17}{images/Plot2D_Nandhita Putri Shalsabila-042.png}
\begin{eulercomment}
plot2d("sin(x)", 0, pi): Menggambar grafik fungsi sin(x) dari 0 hingga
pi.\\
plot2d(2, sin(2), \textgreater{}points, \textgreater{}add): Menambahkan titik ke grafik pada
koordinat (2,sin(2)).\\
\textgreater{}points: Menyatakan bahwa titik yang ditambahkan akan ditampilkan
sebagai titik pada grafik.\\
\textgreater{}add: Menambahkan titik tersebut ke grafik yang sudah ada tanpa
menghapus grafik sin(x).\\
Jadi, perintah ini menggambarkan grafik\\
sin(x) dalam interval 0 hingga pi dan menambahkan titik di posisi
(2,sin(2)) ke grafik tersebut.

Kita tambahkan titik perpotongan dengan label (pada posisi "cl" untuk
kiri tengah), dan masukkan hasilnya ke dalam buku catatan. Kami juga
menambahkan judul pada plot.
\end{eulercomment}
\begin{eulerprompt}
>plot2d(["cos(x)","x"],r=1.1,cx=0.5,cy=0.5, ...
>  color=[black,blue],style=["-","."], ...
>  grid=1);
>x0=solve("cos(x)-x",1);  ...
>  plot2d(x0,x0,>points,>add,title="Intersection Demo");  ...
>  label("cos(x) = x",x0,x0,pos="cl",offset=20):
\end{eulerprompt}
\eulerimg{17}{images/Plot2D_Nandhita Putri Shalsabila-043.png}
\begin{eulercomment}
Secara keseluruhan, perintah ini menggambarkan grafik fungsi cos(x)
dan y=x, menunjukkan titik perpotongan antara keduanya, dan
menambahkan label untuk memberikan informasi lebih lanjut tentang
titik tersebut.

Dalam contoh berikut, kita memplot fungsi sin(x)=sin(x)/x dan ekspansi
Taylor ke-8 dan ke-16. Kami menghitung perluasan ini menggunakan
Maxima melalui ekspresi simbolik.\\
Plot ini dilakukan dalam perintah multi-baris berikut dengan tiga
panggilan ke plot2d(). Yang kedua dan ketiga memiliki kumpulan tanda
\textgreater{}add, yang membuat plot menggunakan rentang sebelumnya.

Kami menambahkan kotak label yang menjelaskan fungsinya.
\end{eulercomment}
\begin{eulerprompt}
>$taylor(sin(x)/x,x,0,4)
\end{eulerprompt}
\begin{eulerformula}
\[
\frac{x^4}{120}-\frac{x^2}{6}+1
\]
\end{eulerformula}
\begin{eulercomment}
memplot fungsi sin x dari pangkat 0 sampai 4
\end{eulercomment}
\begin{eulerprompt}
>plot2d("sinc(x)",0,4pi,color=green,thickness=2); ...
>  plot2d(&taylor(sin(x)/x,x,0,8),>add,color=blue,style="--"); ...
>  plot2d(&taylor(sin(x)/x,x,0,16),>add,color=red,style="-.-"); ...
>  labelbox(["sinc","T8","T16"],styles=["-","--","-.-"], ...
>    colors=[black,blue,red]):
\end{eulerprompt}
\eulerimg{17}{images/Plot2D_Nandhita Putri Shalsabila-045.png}
\begin{eulercomment}
Secara keseluruhan, perintah ini menggambarkan grafik fungsi \\
sinc(x) serta dua polinomial Taylor untuk membandingkan seberapa baik
aproksimasi Taylor mendekati fungsi asli pada interval yang diberikan,
dengan label yang jelas untuk setiap grafik.

Dalam contoh berikut, kami menghasilkan Polinomial Bernstein.

\end{eulercomment}
\begin{eulerformula}
\[
B_i(x) = \binom{n}{i} x^i (1-x)^{n-i}
\]
\end{eulerformula}
\begin{eulerprompt}
>plot2d("(1-x)^10",0,1); // plot first function
>for i=1 to 10; plot2d("bin(10,i)*x^i*(1-x)^(10-i)",>add); end;
>insimg;
\end{eulerprompt}
\eulerimg{17}{images/Plot2D_Nandhita Putri Shalsabila-047.png}
\begin{eulercomment}
Cara kedua adalah dengan menggunakan pasangan matriks bernilai x dan
matriks bernilai y yang berukuran sama.

Kami menghasilkan matriks nilai dengan satu Polinomial Bernstein di
setiap baris. Untuk ini, kita cukup menggunakan vektor kolom i. Lihat
pendahuluan tentang bahasa matriks untuk mempelajari lebih detail.
\end{eulercomment}
\begin{eulerprompt}
>x=linspace(0,1,500);
>n=10; k=(0:n)'; // n is row vector, k is column vector
>y=bin(n,k)*x^k*(1-x)^(n-k); // y is a matrix then
>plot2d(x,y):
\end{eulerprompt}
\eulerimg{17}{images/Plot2D_Nandhita Putri Shalsabila-048.png}
\begin{eulercomment}
Secara keseluruhan, perintah ini menggambarkan grafik distribusi
probabilitas binomial untuk n=10 pada interval x dari 0 hingga 1,
menunjukkan bagaimana probabilitas berubah berdasarkan nilai k.

Perhatikan bahwa parameter warna dapat berupa vektor. Kemudian setiap
warna digunakan untuk setiap baris matriks.
\end{eulercomment}
\begin{eulerprompt}
>x=linspace(0,1,200); y=x^(1:10)'; plot2d(x,y,color=1:10):
\end{eulerprompt}
\eulerimg{17}{images/Plot2D_Nandhita Putri Shalsabila-049.png}
\begin{eulercomment}
Secara keseluruhan, perintah ini menggambarkan grafik dari fungsi\\
y=x\textasciicircum{}k untuk k dari 1 hingga 10, dengan setiap kurva ditampilkan dalam
warna yang berbeda. Ini memberikan visualisasi bagaimana fungsi
pangkat berubah saat berkisar antara 0 hingga 1.

Metode lain adalah menggunakan vektor ekspresi (string). Anda kemudian
dapat menggunakan susunan warna, susunan gaya, dan susunan ketebalan
dengan panjang yang sama.
\end{eulercomment}
\begin{eulerprompt}
>plot2d(["sin(x)","cos(x)"],0,2pi,color=4:5): 
\end{eulerprompt}
\eulerimg{17}{images/Plot2D_Nandhita Putri Shalsabila-050.png}
\begin{eulercomment}
Perintah ini menggambarkan grafik dari fungsi sin(x) dan cos(x) dalam
interval dari 0 hingga 2pi, dengan warna yang berbeda untuk
masing-masing fungsi, sehingga keduanya dapat dengan mudah dibedakan
dalam grafik.

\end{eulercomment}
\begin{eulerprompt}
>plot2d(["sin(x)","cos(x)"],0,2pi): // plot vector of expressions
\end{eulerprompt}
\eulerimg{17}{images/Plot2D_Nandhita Putri Shalsabila-051.png}
\begin{eulercomment}
Perintah ini menggambarkan grafik fungsi sin(x) dan cos(x) dalam
interval dari 0 hingga 2pi. Kedua grafik ini akan ditampilkan dalam
satu plot, memungkinkan perbandingan visual antara kedua fungsi
trigonometri tersebut.

Kita bisa mendapatkan vektor seperti itu dari Maxima menggunakan
makelist() dan mxm2str().
\end{eulercomment}
\begin{eulerprompt}
>v &= makelist(binomial(10,i)*x^i*(1-x)^(10-i),i,0,10) // make list
\end{eulerprompt}
\begin{euleroutput}
  
                 10            9              8  2             7  3
         [(1 - x)  , 10 (1 - x)  x, 45 (1 - x)  x , 120 (1 - x)  x , 
             6  4             5  5             4  6             3  7
  210 (1 - x)  x , 252 (1 - x)  x , 210 (1 - x)  x , 120 (1 - x)  x , 
            2  8              9   10
  45 (1 - x)  x , 10 (1 - x) x , x  ]
  
\end{euleroutput}
\begin{eulercomment}
Perintah ini menghasilkan daftar v yang berisi nilai distribusi
probabilitas binomial untuk n=10 berdasarkan berbagai nilai i dari 0
hingga 10. Setiap elemen dalam daftar v mewakili probabilitas untuk
masing-masing jumlah keberhasilan i dalam 10 percobaan, tergantung
pada nilai x.

fungsi makelist digunakan untuk membuat list dari elemen-elemen yang
dihasilkan oleh ekspresi didalamnya berdasarkan parameter yang
diberikan.
\end{eulercomment}
\begin{eulerprompt}
>mxm2str(v) // get a vector of strings from the symbolic vector
\end{eulerprompt}
\begin{euleroutput}
  (1-x)^10
  10*(1-x)^9*x
  45*(1-x)^8*x^2
  120*(1-x)^7*x^3
  210*(1-x)^6*x^4
  252*(1-x)^5*x^5
  210*(1-x)^4*x^6
  120*(1-x)^3*x^7
  45*(1-x)^2*x^8
  10*(1-x)*x^9
  x^10
\end{euleroutput}
\begin{eulercomment}
Perintah ini bertujuan untuk menghasilkan vektor string dari vektor
simbolik yang ada, memudahkan manipulasi dan presentasi data.

mxm2str untuk vektor string
\end{eulercomment}
\begin{eulerprompt}
>plot2d(mxm2str(v),0,1): // plot functions
\end{eulerprompt}
\eulerimg{17}{images/Plot2D_Nandhita Putri Shalsabila-052.png}
\begin{eulercomment}
Perintah ini menggambarkan grafik dari fungsi-fungsi yang terdapat
dalam vektor simbolik v dalam rentang x dari 0 hingga 1.

Alternatif lain adalah dengan menggunakan bahasa matriks Euler.

Jika suatu ekspresi menghasilkan matriks fungsi, dengan satu fungsi di
setiap baris, semua fungsi tersebut akan diplot ke dalam satu plot.

Untuk ini, gunakan vektor parameter dalam bentuk vektor kolom. Jika
array warna ditambahkan maka akan digunakan untuk setiap baris plot.
\end{eulercomment}
\begin{eulerprompt}
>n=(1:10)'; plot2d("x^n",0,1,color=1:10):
\end{eulerprompt}
\eulerimg{17}{images/Plot2D_Nandhita Putri Shalsabila-053.png}
\begin{eulercomment}
Perintah ini bertujuan untuk menghasilkan grafik dari fungsi x\textasciicircum{}n untuk
n dari 1 hingga 10 dalam interval x dari 0 hingga 1. Dengan demikian,
akan dapat melihat bagaimana fungsi pangkat berubah dengan nilai
eksponen yang berbeda, serta membedakan masing-masing grafik
berdasarkan warnanya.

Ekspresi dan fungsi satu baris dapat melihat variabel global.

Jika Anda tidak dapat menggunakan variabel global, Anda perlu
menggunakan fungsi dengan parameter tambahan, dan meneruskan parameter
ini sebagai parameter titik koma.

Berhati-hatilah, untuk meletakkan semua parameter yang ditetapkan di
akhir perintah plot2d. Dalam contoh ini kita meneruskan a=5 ke fungsi
f, yang kita plot dari -10 hingga 10.
\end{eulercomment}
\begin{eulerprompt}
>function f(x,a) := 1/a*exp(-x^2/a); ...
>plot2d("f",-10,10;5,thickness=2,title="a=5"):
\end{eulerprompt}
\eulerimg{17}{images/Plot2D_Nandhita Putri Shalsabila-054.png}
\begin{eulercomment}
Perintah ini bertujuan untuk menggambarkan grafik fungsi distribusi
normal dengan parameter a=5 dalam rentang x dari -10 hingga 10. Grafik
ini akan menunjukkan bagaimana fungsi eksponensial ini berperilaku
dalam rentang tersebut.

Alternatifnya, gunakan koleksi dengan nama fungsi dan semua parameter
tambahan. Daftar khusus ini disebut kumpulan panggilan, dan ini adalah
cara yang lebih disukai untuk meneruskan argumen ke suatu fungsi yang
kemudian diteruskan sebagai argumen ke fungsi lain.

Pada contoh berikut, kita menggunakan loop untuk memplot beberapa
fungsi (lihat tutorial tentang pemrograman loop).
\end{eulercomment}
\begin{eulerprompt}
>plot2d(\{\{"f",1\}\},-10,10); ...
>for a=2:10; plot2d(\{\{"f",a\}\},>add); end:
\end{eulerprompt}
\eulerimg{17}{images/Plot2D_Nandhita Putri Shalsabila-055.png}
\begin{eulercomment}
Perintah ini untuk menggambarkan grafik fungsi f untuk parameter a
dari 1 hingga 10 dalam rentang x dari -10 hingga 10. Setiap grafik
akan ditambahkan ke grafik sebelumnya, sehingga dapat melihat
bagaimana fungsi berubah dengan variasi nilai a. Dengan demikian,
grafik akan menampilkan fungsi distribusi normal untuk berbagai nilai
a dalam satu plot.

Kita dapat mencapai hasil yang sama dengan cara berikut menggunakan
bahasa matriks EMT. Setiap baris matriks f(x,a) merupakan satu fungsi.
Selain itu, kita dapat mengatur warna untuk setiap baris matriks. Klik
dua kali pada fungsi getspectral() untuk penjelasannya.
\end{eulercomment}
\begin{eulerprompt}
>x=-10:0.01:10; a=(1:10)'; plot2d(x,f(x,a),color=getspectral(a/10)):
\end{eulerprompt}
\eulerimg{17}{images/Plot2D_Nandhita Putri Shalsabila-056.png}
\begin{eulercomment}
Perintah ini untuk menggambarkan grafik fungsi f(x,a) untuk nilai dari
1 hingga 10 dalam rentang x dari -10 hingga 10. Dengan menggunakan
palet warna spektral, grafik akan ditampilkan dengan warna yang
berbeda untuk setiap nilai a, memungkinkan perbandingan visual antara
fungsi untuk berbagai parameter.

\end{eulercomment}
\eulersubheading{Label Teks}
\begin{eulercomment}
Dekorasi sederhana pun bisa

- judul dengan judul = "..."\\
- label x dan y dengan xl="...", yl="..."\\
- label teks lain dengan label("...",x,y)

Perintah label akan memplot ke plot saat ini pada koordinat plot
(x,y). Hal ini memerlukan argumen posisional.
\end{eulercomment}
\begin{eulerprompt}
>plot2d("x^3-x",-1,2,title="y=x^3-x",yl="y",xl="x"):
\end{eulerprompt}
\eulerimg{17}{images/Plot2D_Nandhita Putri Shalsabila-057.png}
\begin{eulercomment}
Perintah ini menghasilkan grafik dari fungsi y=x\textasciicircum{}3-x dalam rentang\\
x elemen atau anggota [-1,2], dengan judul dan label pada kedua sumbu.
\end{eulercomment}
\begin{eulerprompt}
>expr := "log(x)/x"; ...
>  plot2d(expr,0.5,5,title="y="+expr,xl="x",yl="y"); ...
>  label("(1,0)",1,0); label("Max",E,expr(E),pos="lc"):
\end{eulerprompt}
\eulerimg{17}{images/Plot2D_Nandhita Putri Shalsabila-058.png}
\begin{eulercomment}
Perintah ini menggambar grafik fungsi

\end{eulercomment}
\begin{eulerformula}
\[
y= \frac{\log(x)}{x}
\]
\end{eulerformula}
\begin{eulercomment}
menambahkan label pada titik (1,0), dan juga menandai titik maksimum
di x=e dengan label "Max".

Ada juga fungsi labelbox(), yang dapat menampilkan fungsi dan teks.
Dibutuhkan vektor string dan warna, satu item untuk setiap fungsi.
\end{eulercomment}
\begin{eulerprompt}
>function f(x) &= x^2*exp(-x^2);  ...
>plot2d(&f(x),a=-3,b=3,c=-1,d=1);  ...
>plot2d(&diff(f(x),x),>add,color=blue,style="--"); ...
>labelbox(["function","derivative"],styles=["-","--"], ...
>   colors=[black,blue],w=0.4):
\end{eulerprompt}
\eulerimg{17}{images/Plot2D_Nandhita Putri Shalsabila-060.png}
\begin{eulercomment}
Perintah ini menggambar grafik dari fungsi f(x)=x\textasciicircum{}2 . e\textasciicircum{}-x \textasciicircum{}2 dan
turunannya f'(x) pada rentang elemen atau anggota [-3,3], serta
menambahkan legenda untuk membedakan antara grafik fungsi asli dan
grafik turunannya.

Kotak ini berlabuh di kanan atas secara default, tetapi \textgreater{}kiri berlabuh
di kiri atas. Anda dapat memindahkannya ke tempat mana pun yang Anda
suka. Posisi jangkar berada di pojok kanan atas kotak, dan angkanya
merupakan pecahan dari ukuran jendela grafis. Lebarnya otomatis.

Untuk plot titik, kotak label juga berfungsi. Tambahkan parameter
\textgreater{}points, atau vektor bendera, satu untuk setiap label.

Pada contoh berikut, hanya ada satu fungsi. Jadi kita bisa menggunakan
string sebagai pengganti vektor string. Kami mengatur warna teks
menjadi hitam untuk contoh ini.
\end{eulercomment}
\begin{eulerprompt}
>n=10; plot2d(0:n,bin(n,0:n),>addpoints); ...
>labelbox("Binomials",styles="[]",>points,x=0.1,y=0.1, ...
>tcolor=black,>left):
\end{eulerprompt}
\eulerimg{17}{images/Plot2D_Nandhita Putri Shalsabila-061.png}
\begin{eulercomment}
Perintah ini menggambar distribusi koefisien binomial (10/k)\\
untuk k=0 hingga 10, dengan titik-titik pada setiap nilai binomialnya,
dan menambahkan kotak label "Binomials" untuk menjelaskan grafik yang
digambar.

Gaya plot ini juga tersedia di statplot(). Seperti di plot2d() warna
dapat diatur untuk setiap baris plot. Masih banyak lagi plot khusus
untuk keperluan statistik (lihat tutorial tentang statistik).
\end{eulercomment}
\begin{eulerprompt}
>statplot(1:10,random(2,10),color=[red,blue]):
\end{eulerprompt}
\eulerimg{17}{images/Plot2D_Nandhita Putri Shalsabila-062.png}
\begin{eulercomment}
Perintah ini membuat plot statistik dari dua set data
acak(masing-masing terdiri dari 10 titik), dengan nilai sumbu x dari 1
hingga 10, dan warna yang berbeda (merah dan biru) untuk masing-masing
set data.

Fitur serupa adalah fungsi textbox().

Lebarnya secara default adalah lebar maksimal baris teks. Tapi itu
bisa diatur oleh pengguna juga.
\end{eulercomment}
\begin{eulerprompt}
>function f(x) &= exp(-x)*sin(2*pi*x); ...
>plot2d("f(x)",0,2pi); ...
>textbox(latex("\(\backslash\)text\{Example of a damped oscillation\}\(\backslash\) f(x)=e^\{-x\}sin(2\(\backslash\)pi x)"),w=0.85):
\end{eulerprompt}
\eulerimg{17}{images/Plot2D_Nandhita Putri Shalsabila-063.png}
\begin{eulercomment}
Perintah ini mendefinisikan fungsi osilasi teredam, memplot grafik
dari fungsi tersebut dalam satu siklus, dan menambahkan teks
deskriptif dengan format LaTeX untuk memberikan keterangan visual pada
grafik.

Label teks, judul, kotak label, dan teks lainnya dapat berisi string
Unicode (lihat sintaks EMT untuk mengetahui lebih lanjut tentang
string Unicode).
\end{eulercomment}
\begin{eulerprompt}
>plot2d("x^3-x",title=u"x &rarr; x&sup3; - x"):
\end{eulerprompt}
\eulerimg{17}{images/Plot2D_Nandhita Putri Shalsabila-064.png}
\begin{eulercomment}
Judul tersebut menggunakan simbol Unicode untuk menggambarkan
transformasi dari x ke x\textasciicircum{}3-x, membuat representasi yang lebih rapi dan
estetis pada grafik.

Label pada sumbu x dan y bisa vertikal, begitu juga dengan sumbunya.
\end{eulercomment}
\begin{eulerprompt}
>plot2d("sinc(x)",0,2pi,xl="x",yl=u"x &rarr; sinc(x)",>vertical):
\end{eulerprompt}
\eulerimg{17}{images/Plot2D_Nandhita Putri Shalsabila-065.png}
\begin{eulercomment}
Perintah ini menggambar grafik fungsi sinc(x) dari 0 hingga 2pi,
dengan label sumbu y yang menggambarkan transformasi dari x ke
sinc(x), dan orientasi grafik dalam posisi vertikal.

\end{eulercomment}
\eulersubheading{LaTeX}
\begin{eulercomment}
Anda juga dapat memplot rumus LaTeX jika Anda telah menginstal sistem
LaTeX. Saya merekomendasikan MiKTeX. Jalur ke biner "lateks" dan
"dvipng" harus berada di jalur sistem, atau Anda harus mengatur LaTeX
di menu opsi.

Perhatikan, penguraian LaTeX lambat. Jika Anda ingin menggunakan LaTeX
dalam plot animasi, Anda harus memanggil latex() sebelum loop satu
kali dan menggunakan hasilnya (gambar dalam matriks RGB).

Pada plot berikut, kami menggunakan LaTeX untuk label x dan y, label,
kotak label, dan judul plot.
\end{eulercomment}
\begin{eulerprompt}
>plot2d("exp(-x)*sin(x)/x",a=0,b=2pi,c=0,d=1,grid=6,color=blue, ...
>  title=latex("\(\backslash\)text\{Function $\(\backslash\)Phi$\}"), ...
>  xl=latex("\(\backslash\)phi"),yl=latex("\(\backslash\)Phi(\(\backslash\)phi)")); ...
>textbox( ...
>  latex("\(\backslash\)Phi(\(\backslash\)phi) = e^\{-\(\backslash\)phi\} \(\backslash\)frac\{\(\backslash\)sin(\(\backslash\)phi)\}\{\(\backslash\)phi\}"),x=0.8,y=0.5); ...
>label(latex("\(\backslash\)Phi",color=blue),1,0.4):
\end{eulerprompt}
\eulerimg{17}{images/Plot2D_Nandhita Putri Shalsabila-066.png}
\begin{eulercomment}
Perintah ini untuk membuat plot fungsi teredam sinusoidal, memberi
judul, label pada sumbu, serta tambahan teks dan label pada bagian
tertentu dari plot untuk memperjelas.

Seringkali, kita menginginkan spasi dan label teks yang tidak
konformal pada sumbu x. Kita bisa menggunakan xaxis() dan yaxis()
seperti yang akan kita tunjukkan nanti.

Cara termudah adalah membuat plot kosong dengan bingkai menggunakan
grid=4, lalu menambahkan grid dengan ygrid() dan xgrid(). Pada contoh
berikut, kami menggunakan tiga string LaTeX untuk label pada sumbu x
dengan xtick().
\end{eulercomment}
\begin{eulerprompt}
>plot2d("sinc(x)",0,2pi,grid=4,<ticks); ...
>ygrid(-2:0.5:2,grid=6); ...
>xgrid([0:2]*pi,<ticks,grid=6);  ...
>xtick([0,pi,2pi],["0","\(\backslash\)pi","2\(\backslash\)pi"],>latex):
\end{eulerprompt}
\eulerimg{17}{images/Plot2D_Nandhita Putri Shalsabila-067.png}
\begin{eulercomment}
Perintah ini untuk menghasilkan plot yang informatif dan mudah dibaca
untuk fungsi sinc, dengan pengaturan yang jelas pada sumbu x dan y
serta grid yang memudahkan analisis visual.

Tentu saja fungsinya juga bisa digunakan.
\end{eulercomment}
\begin{eulerprompt}
>function map f(x) ...
\end{eulerprompt}
\begin{eulerudf}
  if x>0 then return x^4
  else return x^2
  endif
  endfunction
\end{eulerudf}
\begin{eulercomment}
function map f(x): mendefinisikan sebuah fungsi bernama f yang
menerima satu argumen input x.\\
\textdollar{}if x\textgreater{}0 then return x\textasciicircum{}4: sebuah pernyataan kondisi. Jika x lebih besar
dari 0, fungsi akan mengembalikan x\textasciicircum{}4.\\
\textdollar{}else return x\textasciicircum{}2: jika kondisi sebelumnya tidak terpenuhi (yaitu, jika
x tidak lebih besar dari 0), fungsi akan mengembalikan x\textasciicircum{}2.\\
\textdollar{}endif: menandai akhir dari pernyataan kondisi.\\
\textdollar{}endfunction: menandai akhir dari definisi fungsi.

Parameter "peta" membantu menggunakan fungsi untuk vektor. Untuk\\
plot, itu tidak perlu. Tapi untuk menunjukkan vektorisasi itu\\
berguna, kita menambahkan beberapa poin penting ke plot di x=-1, x=0
dan x=1.

Pada plot berikut, kami juga memasukkan beberapa kode LaTeX. Kami
menggunakannya untuk\\
dua label dan kotak teks. Tentu saja, Anda hanya bisa menggunakannya\\
LaTeX jika Anda telah menginstal LaTeX dengan benar.
\end{eulercomment}
\begin{eulerprompt}
>plot2d("f",-1,1,xl="x",yl="f(x)",grid=6);  ...
>plot2d([-1,0,1],f([-1,0,1]),>points,>add); ...
>label(latex("x^3"),0.72,f(0.72)); ...
>label(latex("x^2"),-0.52,f(-0.52),pos="ll"); ...
>textbox( ...
>  latex("f(x)=\(\backslash\)begin\{cases\} x^3 & x>0 \(\backslash\)\(\backslash\) x^2 & x \(\backslash\)le 0\(\backslash\)end\{cases\}"), ...
>  x=0.7,y=0.2):
\end{eulerprompt}
\eulerimg{17}{images/Plot2D_Nandhita Putri Shalsabila-068.png}
\begin{eulercomment}
Perintah ini menciptakan plot yang informatif dari fungsi bersyarat\\
f(x), dengan tambahan label, titik-titik penting, dan kotak teks yang
menjelaskan fungsi tersebut secara rinci.

\end{eulercomment}
\eulersubheading{Interaksi Pengguna}
\begin{eulercomment}
Saat memplot suatu fungsi atau ekspresi, parameter \textgreater{}pengguna
memungkinkan pengguna untuk memperbesar dan menggeser plot dengan
tombol kursor atau mouse. Pengguna bisa

- perbesar dengan + atau -\\
- pindahkan plot dengan tombol kursor\\
- pilih jendela plot dengan mouse\\
- atur ulang tampilan dengan spasi\\
- keluar dengan kembali

Tombol spasi akan mengatur ulang plot ke jendela plot aslinya.

Saat memplot data, flag \textgreater{}user hanya akan menunggu penekanan tombol.
\end{eulercomment}
\begin{eulerprompt}
>plot2d(\{\{"x^3-a*x",a=1\}\},>user,title="Press any key!"):
\end{eulerprompt}
\eulerimg{17}{images/Plot2D_Nandhita Putri Shalsabila-069.png}
\begin{eulercomment}
Perintah ini membuat plot dari fungsi x\textasciicircum{}3-x (karena a diatur ke 1) dan
mengizinkan pengguna untuk berinteraksi dengan parameter a untuk
melihat bagaimana perubahan nilai a mempengaruhi bentuk grafik. Judul
"Press any key!" memberikan petunjuk kepada pengguna untuk melanjutkan
ke langkah selanjutnya.
\end{eulercomment}
\begin{eulerprompt}
>plot2d("exp(x)*sin(x)",user=true, ...
>  title="+/- or cursor keys (return to exit)"):
\end{eulerprompt}
\eulerimg{17}{images/Plot2D_Nandhita Putri Shalsabila-070.png}
\begin{eulercomment}
Perintah ini utnuk menciptakan plot interaktif dari fungsi \\
exp(x).sin(x) dan memberikan instruksi kepada pengguna tentang cara
berinteraksi dengan grafik tersebut.

Berikut ini menunjukkan cara interaksi pengguna tingkat lanjut (lihat
tutorial tentang pemrograman untuk detailnya).

Fungsi bawaan mousedrag() menunggu aktivitas mouse atau keyboard. Ini
melaporkan mouse ke bawah, gerakan mouse atau mouse ke atas, dan
penekanan tombol. Fungsi dragpoints() memanfaatkan ini, dan
memungkinkan pengguna menyeret titik mana pun dalam plot.

Kita membutuhkan fungsi plot terlebih dahulu. Misalnya, kita melakukan
interpolasi pada 5 titik dengan polinomial. Fungsi tersebut harus
diplot ke dalam area plot yang tetap.
\end{eulercomment}
\begin{eulerprompt}
>function plotf(xp,yp,select) ...
\end{eulerprompt}
\begin{eulerudf}
    d=interp(xp,yp);
    plot2d("interpval(xp,d,x)";d,xp,r=2);
    plot2d(xp,yp,>points,>add);
    if select>0 then
      plot2d(xp[select],yp[select],color=red,>points,>add);
    endif;
    title("Drag one point, or press space or return!");
  endfunction
\end{eulerudf}
\begin{eulercomment}
Perhatikan parameter titik koma di plot2d (d dan xp), yang diteruskan
ke evaluasi fungsi interp(). Tanpa ini, kita harus menulis fungsi
plotinterp() terlebih dahulu, mengakses nilainya secara global.

Sekarang kita menghasilkan beberapa nilai acak, dan membiarkan
pengguna menyeret titiknya.
\end{eulercomment}
\begin{eulerprompt}
>t=-1:0.5:1; dragpoints("plotf",t,random(size(t))-0.5):
\end{eulerprompt}
\eulerimg{17}{images/Plot2D_Nandhita Putri Shalsabila-071.png}
\begin{eulercomment}
Ada juga fungsi yang memplot fungsi lain bergantung pada vektor
parameter, dan memungkinkan pengguna menyesuaikan parameter ini.

Pertama kita membutuhkan fungsi plot.
\end{eulercomment}
\begin{eulerprompt}
>function plotf([a,b]) := plot2d("exp(a*x)*cos(2pi*b*x)",0,2pi;a,b);
\end{eulerprompt}
\begin{eulercomment}
Kemudian kita memerlukan nama untuk parameter, nilai awal dan matriks
rentang nx2, opsional garis judul.\\
Ada penggeser interaktif, yang dapat menetapkan nilai oleh pengguna.
Fungsi dragvalues() menyediakan ini.
\end{eulercomment}
\begin{eulerprompt}
>dragvalues("plotf",["a","b"],[-1,2],[[-2,2];[1,10]], ...
>  heading="Drag these values:",hcolor=black):
\end{eulerprompt}
\eulerimg{17}{images/Plot2D_Nandhita Putri Shalsabila-072.png}
\begin{eulercomment}
Dimungkinkan untuk membatasi nilai yang diseret menjadi bilangan
bulat. Sebagai contoh, kita menulis fungsi plot, yang memplot
polinomial Taylor berderajat n ke fungsi kosinus.
\end{eulercomment}
\begin{eulerprompt}
>function plotf(n) ...
\end{eulerprompt}
\begin{eulerudf}
  plot2d("cos(x)",0,2pi,>square,grid=6);
  plot2d(&"taylor(cos(x),x,0,@n)",color=blue,>add);
  textbox("Taylor polynomial of degree "+n,0.1,0.02,style="t",>left);
  endfunction
\end{eulerudf}
\begin{eulercomment}
Sekarang kita izinkan derajat n bervariasi dari 0 hingga 20 dalam 20
perhentian. Hasil dragvalues() digunakan untuk memplot sketsa dengan n
ini, dan untuk memasukkan plot ke dalam buku catatan.
\end{eulercomment}
\begin{eulerprompt}
>nd=dragvalues("plotf","degree",2,[0,20],20,y=0.8, ...
>   heading="Drag the value:"); ...
>plotf(nd):
\end{eulerprompt}
\eulerimg{17}{images/Plot2D_Nandhita Putri Shalsabila-073.png}
\begin{eulercomment}
Berikut ini adalah demonstrasi sederhana dari fungsinya. Pengguna
dapat menggambar jendela plot, meninggalkan jejak titik.
\end{eulercomment}
\begin{eulerprompt}
>function dragtest ...
\end{eulerprompt}
\begin{eulerudf}
    plot2d(none,r=1,title="Drag with the mouse, or press any key!");
    start=0;
    repeat
      \{flag,m,time\}=mousedrag();
      if flag==0 then return; endif;
      if flag==2 then
        hold on; mark(m[1],m[2]); hold off;
      endif;
    end
  endfunction
\end{eulerudf}
\begin{eulerprompt}
>dragtest // lihat hasilnya dan cobalah lakukan!
\end{eulerprompt}
\eulersubheading{Gaya Plot 2D}
\begin{eulercomment}
Secara default, EMT menghitung tick sumbu otomatis dan menambahkan
label ke setiap tick. Ini dapat diubah dengan parameter grid. Gaya
default sumbu dan label dapat diubah. Selain itu, label dan judul
dapat ditambahkan secara manual. Untuk menyetel ulang ke gaya default,
gunakan reset().
\end{eulercomment}
\begin{eulerprompt}
>aspect();
>figure(3,4); ...
> figure(1); plot2d("x^3-x",grid=0); ... // no grid, frame or axis
> figure(2); plot2d("x^3-x",grid=1); ... // x-y-axis
> figure(3); plot2d("x^3-x",grid=2); ... // default ticks
> figure(4); plot2d("x^3-x",grid=3); ... // x-y- axis with labels inside
> figure(5); plot2d("x^3-x",grid=4); ... // no ticks, only labels
> figure(6); plot2d("x^3-x",grid=5); ... // default, but no margin
> figure(7); plot2d("x^3-x",grid=6); ... // axes only
> figure(8); plot2d("x^3-x",grid=7); ... // axes only, ticks at axis
> figure(9); plot2d("x^3-x",grid=8); ... // axes only, finer ticks at axis
> figure(10); plot2d("x^3-x",grid=9); ... // default, small ticks inside
> figure(11); plot2d("x^3-x",grid=10); ...// no ticks, axes only
> figure(0):
\end{eulerprompt}
\eulerimg{27}{images/Plot2D_Nandhita Putri Shalsabila-074.png}
\begin{eulercomment}
Parameter \textless{}frame mematikan frame, dan framecolor=blue mengatur frame
menjadi warna biru.

Jika Anda menginginkan tanda centang Anda sendiri, Anda dapat
menggunakan style=0, dan menambahkan semuanya nanti.
\end{eulercomment}
\begin{eulerprompt}
>aspect(1.5); 
>plot2d("x^3-x",grid=0); // plot
>frame; xgrid([-1,0,1]); ygrid(0): // add frame and grid
\end{eulerprompt}
\eulerimg{17}{images/Plot2D_Nandhita Putri Shalsabila-075.png}
\begin{eulercomment}
Untuk judul plot dan label sumbu, lihat contoh berikut.
\end{eulercomment}
\begin{eulerprompt}
>plot2d("exp(x)",-1,1);
>textcolor(black); // set the text color to black
>title(latex("y=e^x")); // title above the plot
>xlabel(latex("x")); // "x" for x-axis
>ylabel(latex("y"),>vertical); // vertical "y" for y-axis
>label(latex("(0,1)"),0,1,color=blue): // label a point
\end{eulerprompt}
\eulerimg{17}{images/Plot2D_Nandhita Putri Shalsabila-076.png}
\begin{eulercomment}
Sumbu dapat digambar secara terpisah dengan xaxis() dan yaxis().
\end{eulercomment}
\begin{eulerprompt}
>plot2d("x^3-x",<grid,<frame);
>xaxis(0,xx=-2:1,style="->"); yaxis(0,yy=-5:5,style="->"):
\end{eulerprompt}
\eulerimg{17}{images/Plot2D_Nandhita Putri Shalsabila-077.png}
\begin{eulercomment}
Teks pada plot dapat diatur dengan label(). Dalam contoh berikut, "lc"
berarti bagian tengah bawah. Ini menetapkan posisi label relatif
terhadap koordinat plot.
\end{eulercomment}
\begin{eulerprompt}
>function f(x) &= x^3-x
\end{eulerprompt}
\begin{euleroutput}
  
                                   3
                                  x  - x
  
\end{euleroutput}
\begin{eulerprompt}
>plot2d(f,-1,1,>square);
>x0=fmin(f,0,1); // compute point of minimum
>label("Rel. Min.",x0,f(x0),pos="lc"): // add a label there
\end{eulerprompt}
\eulerimg{17}{images/Plot2D_Nandhita Putri Shalsabila-078.png}
\begin{eulercomment}
Ada juga kotak teks.
\end{eulercomment}
\begin{eulerprompt}
>plot2d(&f(x),-1,1,-2,2); // function
>plot2d(&diff(f(x),x),>add,style="--",color=red); // derivative
>labelbox(["f","f'"],["-","--"],[black,red]): // label box
\end{eulerprompt}
\eulerimg{17}{images/Plot2D_Nandhita Putri Shalsabila-079.png}
\begin{eulercomment}
Perintah ini menghasilkan plot dari fungsi dan turunan pertama, serta
memberikan penjelasan visual yang jelas dengan kotak label untuk
masing-masing grafik.
\end{eulercomment}
\begin{eulerprompt}
>plot2d(["exp(x)","1+x"],color=[black,blue],style=["-","-.-"]):
\end{eulerprompt}
\eulerimg{17}{images/Plot2D_Nandhita Putri Shalsabila-080.png}
\begin{eulercomment}
Perintah ini menciptakan plot yang jelas dan terpisah untuk kedua
fungsi dengan pengaturan warna dan gaya yang membedakan keduanya,
sehingga memudahkan visualisasi dan perbandingan antara exp(x) dan
1+x.

Guna tingkat lanjut (lihat tutorial tentang pemrograman untuk
detailnya).

Fungsi bawaan mousedrag() menunggu aktivitas mouse atau keyboard. Ini
melaporkan mouse ke bawah, gerakan mouse atau mouse ke atas, dan
penekanan tombol. Fungsi dragpoints() memanfaatkan ini, dan
memungkinkan pengguna menyeret titik mana pun dalam plot.

Kita membutuhkan fungsi plot terlebih dahulu. Misalnya, kita melakukan
interpolasi pada 5 titik dengan polinomial. Fungsi tersebut harus
diplot ke dalam area plot yang tetap.
\end{eulercomment}
\begin{eulerprompt}
>gridstyle("->",color=gray,textcolor=gray,framecolor=gray);  ...
> plot2d("x^3-x",grid=1);   ...
> settitle("y=x^3-x",color=black); ...
> label("x",2,0,pos="bc",color=gray);  ...
> label("y",0,6,pos="cl",color=gray); ...
> reset():
\end{eulerprompt}
\eulerimg{27}{images/Plot2D_Nandhita Putri Shalsabila-081.png}
\begin{eulercomment}
Untuk kontrol lebih lanjut, sumbu x dan sumbu y dapat dilakukan secara
manual.

Perintah fullwindow() memperluas jendela plot karena kita tidak lagi
memerlukan tempat untuk label di luar jendela plot. Gunakan
shrinkwindow() atau reset() untuk menyetel ulang ke default.
\end{eulercomment}
\begin{eulerprompt}
>fullwindow; ...
> gridstyle(color=darkgray,textcolor=darkgray); ...
> plot2d(["2^x","1","2^(-x)"],a=-2,b=2,c=0,d=4,<grid,color=4:6,<frame); ...
> xaxis(0,-2:1,style="->"); xaxis(0,2,"x",<axis); ...
> yaxis(0,4,"y",style="->"); ...
> yaxis(-2,1:4,>left); ...
> yaxis(2,2^(-2:2),style=".",<left); ...
> labelbox(["2^x","1","2^-x"],colors=4:6,x=0.8,y=0.2); ...
> reset:
\end{eulerprompt}
\eulerimg{27}{images/Plot2D_Nandhita Putri Shalsabila-082.png}
\begin{eulercomment}
Berikut adalah contoh lain, di mana string Unicode digunakan dan
sumbunya berada di luar area plot.
\end{eulercomment}
\begin{eulerprompt}
>aspect(1.5); 
>plot2d(["sin(x)","cos(x)"],0,2pi,color=[red,green],<grid,<frame); ...
> xaxis(-1.1,(0:2)*pi,xt=["0",u"&pi;",u"2&pi;"],style="-",>ticks,>zero);  ...
> xgrid((0:0.5:2)*pi,<ticks); ...
> yaxis(-0.1*pi,-1:0.2:1,style="-",>zero,>grid); ...
> labelbox(["sin","cos"],colors=[red,green],x=0.5,y=0.2,>left); ...
> xlabel(u"&phi;"); ylabel(u"f(&phi;)"):
\end{eulerprompt}
\eulerimg{17}{images/Plot2D_Nandhita Putri Shalsabila-083.png}
\begin{eulercomment}
Perintah ini untuk membuat plot yang informatif dan estetis dari
fungsi sinus dan kosinus dengan pengaturan yang jelas pada sumbu dan
label, serta grid untuk membantu visualisasi.

\begin{eulercomment}
\eulerheading{Merencanakan Data 2D}
\begin{eulercomment}
Jika x dan y adalah vektor data, maka data tersebut akan digunakan
sebagai koordinat x dan y pada suatu kurva. Dalam hal ini, a, b, c,
dan d, atau radius r dapat ditentukan, atau jendela plot akan
menyesuaikan secara otomatis dengan data. Alternatifnya, \textgreater{}persegi
dapat diatur untuk mempertahankan rasio aspek persegi.

Merencanakan ekspresi hanyalah singkatan dari plot data. Untuk plot
data, Anda memerlukan satu atau beberapa baris nilai x, dan satu atau
beberapa baris nilai y. Dari rentang dan nilai x, fungsi plot2d akan
menghitung data yang akan diplot, secara default dengan evaluasi
fungsi yang adaptif. Untuk plot titik gunakan "\textgreater{}titik", untuk garis
dan titik campuran gunakan "\textgreater{}addpoints".

Tapi Anda bisa memasukkan data secara langsung.

- Gunakan vektor baris untuk x dan y untuk satu fungsi.\\
- Matriks untuk x dan y diplot baris demi baris.

Berikut adalah contoh dengan satu baris untuk x dan y.
\end{eulercomment}
\begin{eulerprompt}
>x=-10:0.1:10; y=exp(-x^2)*x; plot2d(x,y):
\end{eulerprompt}
\eulerimg{17}{images/Plot2D_Nandhita Putri Shalsabila-084.png}
\begin{eulercomment}
Data juga dapat diplot sebagai poin. Gunakan points=true untuk ini.
Plotnya berfungsi seperti poligon, tetapi hanya menggambar sudutnya
saja.

- style="...": Pilih dari "[]", "\textless{}\textgreater{}", "o", ".", "..", "+", "*", "[]#",
"\textless{} \textgreater{}#", "o#", "..#", "#", "\textbar{}".

Untuk memplot kumpulan titik, gunakan \textgreater{}titik. Jika warna merupakan
vektor warna, masing-masing titik\\
mendapat warna berbeda. Untuk matriks koordinat dan vektor kolom,
warna diterapkan pada baris matriks.\\
Parameter \textgreater{}addpoints menambahkan titik ke segmen garis untuk plot
data.
\end{eulercomment}
\begin{eulerprompt}
>xdata=[1,1.5,2.5,3,4]; ydata=[3,3.1,2.8,2.9,2.7]; // data
>plot2d(xdata,ydata,a=0.5,b=4.5,c=2.5,d=3.5,style="."); // lines
>plot2d(xdata,ydata,>points,>add,style="o"): // add points
\end{eulerprompt}
\eulerimg{17}{images/Plot2D_Nandhita Putri Shalsabila-085.png}
\begin{eulercomment}
Perintah ini untuk membuat plot dari data dengan garis yang
menghubungkan titik-titik, serta menambahkan titik-titik tersebut
sebagai lingkaran untuk memperjelas lokasi data dalam grafik.
\end{eulercomment}
\begin{eulerprompt}
>p=polyfit(xdata,ydata,1); // get regression line
>plot2d("polyval(p,x)",>add,color=red): // add plot of line
\end{eulerprompt}
\eulerimg{17}{images/Plot2D_Nandhita Putri Shalsabila-086.png}
\begin{eulercomment}
Perintah tersebut digunakan untuk menghitung dan menampilkan garis
regresi linier berdasarkan data yang diberikan.

\begin{eulercomment}
\eulerheading{Menggambar Daerah Yang Dibatasi Kurva}
\begin{eulercomment}
Plot data sebenarnya berbentuk poligon. Kita juga dapat memplot kurva
atau kurva terisi.

- filled=benar mengisi plot.\\
- style="...": Pilih dari "#", "/", "\textbackslash{}", "\textbackslash{}/".\\
- Fillcolor : Lihat di atas untuk mengetahui warna yang tersedia.

Warna isian ditentukan oleh argumen "fillcolor", dan pada \textless{}outline
opsional, mencegah menggambar batas untuk semua gaya kecuali gaya
default.
\end{eulercomment}
\begin{eulerprompt}
>t=linspace(0,2pi,1000); // parameter for curve
>x=sin(t)*exp(t/pi); y=cos(t)*exp(t/pi); // x(t) and y(t)
>figure(1,2); aspect(16/9)
>figure(1); plot2d(x,y,r=10); // plot curve
>figure(2); plot2d(x,y,r=10,>filled,style="/",fillcolor=red); // fill curve
>figure(0):
\end{eulerprompt}
\eulerimg{14}{images/Plot2D_Nandhita Putri Shalsabila-087.png}
\begin{eulercomment}
Dalam contoh berikut kita memplot elips terisi dan dua segi enam
terisi menggunakan kurva tertutup dengan 6 titik dengan gaya isian
berbeda.
\end{eulercomment}
\begin{eulerprompt}
>x=linspace(0,2pi,1000); plot2d(sin(x),cos(x)*0.5,r=1,>filled,style="/"):
\end{eulerprompt}
\eulerimg{14}{images/Plot2D_Nandhita Putri Shalsabila-088.png}
\begin{eulerprompt}
>t=linspace(0,2pi,6); ...
>plot2d(cos(t),sin(t),>filled,style="/",fillcolor=red,r=1.2):
\end{eulerprompt}
\eulerimg{14}{images/Plot2D_Nandhita Putri Shalsabila-089.png}
\begin{eulerprompt}
>t=linspace(0,2pi,6); plot2d(cos(t),sin(t),>filled,style="#"):
\end{eulerprompt}
\eulerimg{14}{images/Plot2D_Nandhita Putri Shalsabila-090.png}
\begin{eulercomment}
Contoh lainnya adalah septagon yang kita buat dengan 7 titik pada
lingkaran satuan.
\end{eulercomment}
\begin{eulerprompt}
>t=linspace(0,2pi,7);  ...
> plot2d(cos(t),sin(t),r=1,>filled,style="/",fillcolor=red):
\end{eulerprompt}
\eulerimg{14}{images/Plot2D_Nandhita Putri Shalsabila-091.png}
\begin{eulercomment}
Berikut adalah himpunan nilai maksimal dari empat kondisi linier yang
kurang dari atau sama dengan 3. Ini adalah A[k].v\textless{}=3 untuk semua baris
A. Untuk mendapatkan sudut yang bagus, kita menggunakan n yang relatif
besar.
\end{eulercomment}
\begin{eulerprompt}
>A=[2,1;1,2;-1,0;0,-1];
>function f(x,y) := max([x,y].A');
>plot2d("f",r=4,level=[0;3],color=green,n=111):
\end{eulerprompt}
\eulerimg{14}{images/Plot2D_Nandhita Putri Shalsabila-092.png}
\begin{eulercomment}
Poin utama dari bahasa matriks adalah memungkinkan pembuatan tabel
fungsi dengan mudah.
\end{eulercomment}
\begin{eulerprompt}
>t=linspace(0,2pi,1000); x=cos(3*t); y=sin(4*t);
\end{eulerprompt}
\begin{eulercomment}
Kami sekarang memiliki nilai vektor x dan y. plot2d() dapat memplot
nilai-nilai ini\\
sebagai kurva yang menghubungkan titik-titik tersebut. Plotnya bisa
diisi. Dalam hal ini\\
ini menghasilkan hasil yang bagus karena aturan belitan, yang
digunakan untuk\\
isi.
\end{eulercomment}
\begin{eulerprompt}
>plot2d(x,y,<grid,<frame,>filled):
\end{eulerprompt}
\eulerimg{14}{images/Plot2D_Nandhita Putri Shalsabila-093.png}
\begin{eulercomment}
Vektor interval diplot terhadap nilai x sebagai wilayah terisi\\
antara nilai interval yang lebih rendah dan lebih tinggi.

Hal ini dapat berguna untuk memplot kesalahan perhitungan. Tapi itu
bisa\\
juga dapat digunakan untuk memplot kesalahan statistik.
\end{eulercomment}
\begin{eulerprompt}
>t=0:0.1:1; ...
> plot2d(t,interval(t-random(size(t)),t+random(size(t))),style="|");  ...
> plot2d(t,t,add=true):
\end{eulerprompt}
\eulerimg{14}{images/Plot2D_Nandhita Putri Shalsabila-094.png}
\begin{eulercomment}
Jika x adalah vektor yang diurutkan, dan y adalah vektor interval,
maka plot2d akan memplot rentang interval yang terisi pada bidang.
Gaya isiannya sama dengan gaya poligon.
\end{eulercomment}
\begin{eulerprompt}
>t=-1:0.01:1; x=~t-0.01,t+0.01~; y=x^3-x;
>plot2d(t,y):
\end{eulerprompt}
\eulerimg{14}{images/Plot2D_Nandhita Putri Shalsabila-095.png}
\begin{eulercomment}
Dimungkinkan untuk mengisi wilayah nilai untuk fungsi tertentu. Untuk\\
ini, level harus berupa matriks 2xn. Baris pertama adalah batas bawah\\
dan baris kedua berisi batas atas.
\end{eulercomment}
\begin{eulerprompt}
>expr := "2*x^2+x*y+3*y^4+y"; // define an expression f(x,y)
>plot2d(expr,level=[0;1],style="-",color=blue): // 0 <= f(x,y) <= 1
\end{eulerprompt}
\eulerimg{14}{images/Plot2D_Nandhita Putri Shalsabila-096.png}
\begin{eulercomment}
Kita juga dapat mengisi rentang nilai seperti

\end{eulercomment}
\begin{eulerformula}
\[
-1 \le (x^2+y^2)^2-x^2+y^2 \le 0.
\]
\end{eulerformula}
\begin{eulercomment}
\end{eulercomment}
\begin{eulerprompt}
>plot2d("(x^2+y^2)^2-x^2+y^2",r=1.2,level=[-1;0],style="/"):
\end{eulerprompt}
\eulerimg{14}{images/Plot2D_Nandhita Putri Shalsabila-097.png}
\begin{eulerprompt}
>plot2d("cos(x)","sin(x)^3",xmin=0,xmax=2pi,>filled,style="/"):
\end{eulerprompt}
\eulerimg{14}{images/Plot2D_Nandhita Putri Shalsabila-098.png}
\begin{eulercomment}
\begin{eulercomment}
\eulerheading{Grafik Fungsi Parametrik}
\begin{eulercomment}
Nilai x tidak perlu diurutkan. (x,y) hanya menggambarkan sebuah kurva.
Jika x diurutkan, kurva tersebut merupakan grafik suatu fungsi.\\
Dalam contoh berikut, kita memplot spiral

\end{eulercomment}
\begin{eulerformula}
\[
\gamma(t) = t \cdot (\cos(2\pi t),\sin(2\pi t))
\]
\end{eulerformula}
\begin{eulercomment}
Kita perlu menggunakan banyak titik untuk tampilan yang halus atau
fungsi adaptif() untuk mengevaluasi ekspresi (lihat fungsi adaptif()
untuk lebih jelasnya).
\end{eulercomment}
\begin{eulerprompt}
>t=linspace(0,1,1000); ...
>plot2d(t*cos(2*pi*t),t*sin(2*pi*t),r=1):
\end{eulerprompt}
\eulerimg{14}{images/Plot2D_Nandhita Putri Shalsabila-099.png}
\begin{eulercomment}
Sebagai alternatif, dimungkinkan untuk menggunakan dua ekspresi untuk
kurva. Berikut ini plot kurva yang sama seperti di atas.
\end{eulercomment}
\begin{eulerprompt}
>plot2d("x*cos(2*pi*x)","x*sin(2*pi*x)",xmin=0,xmax=1,r=1):
\end{eulerprompt}
\eulerimg{14}{images/Plot2D_Nandhita Putri Shalsabila-100.png}
\begin{eulerprompt}
>t=linspace(0,1,1000); r=exp(-t); x=r*cos(2pi*t); y=r*sin(2pi*t);
>plot2d(x,y,r=1):
\end{eulerprompt}
\eulerimg{14}{images/Plot2D_Nandhita Putri Shalsabila-101.png}
\begin{eulercomment}
Pada contoh berikutnya, kita memplot kurvanya

\end{eulercomment}
\begin{eulerformula}
\[
\gamma(t) = (r(t) \cos(t), r(t) \sin(t))
\]
\end{eulerformula}
\begin{eulercomment}
with

\end{eulercomment}
\begin{eulerformula}
\[
r(t) = 1 + \dfrac{\sin(3t)}{2}.
\]
\end{eulerformula}
\begin{eulerprompt}
>t=linspace(0,2pi,1000); r=1+sin(3*t)/2; x=r*cos(t); y=r*sin(t); ...
>plot2d(x,y,>filled,fillcolor=red,style="/",r=1.5):
\end{eulerprompt}
\eulerimg{14}{images/Plot2D_Nandhita Putri Shalsabila-104.png}
\eulerheading{Menggambar Grafik Bilangan Kompleks}
\begin{eulercomment}
Serangkaian bilangan kompleks juga dapat diplot. Kemudian titik-titik
grid akan dihubungkan. Jika sejumlah garis kisi ditentukan (atau
vektor garis kisi 1x2) dalam argumen cgrid, hanya garis kisi tersebut
yang terlihat.

Matriks bilangan kompleks secara otomatis akan diplot sebagai
kisi-kisi pada bidang kompleks.

Pada contoh berikut, kita memplot gambar lingkaran satuan di bawah
fungsi eksponensial. Parameter cgrid menyembunyikan beberapa kurva
grid.
\end{eulercomment}
\begin{eulerprompt}
>aspect(); r=linspace(0,1,50); a=linspace(0,2pi,80)'; z=r*exp(I*a);...
>plot2d(z,a=-1.25,b=1.25,c=-1.25,d=1.25,cgrid=10):
\end{eulerprompt}
\eulerimg{27}{images/Plot2D_Nandhita Putri Shalsabila-105.png}
\begin{eulerprompt}
>aspect(1.25); r=linspace(0,1,50); a=linspace(0,2pi,200)'; z=r*exp(I*a);
>plot2d(exp(z),cgrid=[40,10]):
\end{eulerprompt}
\eulerimg{21}{images/Plot2D_Nandhita Putri Shalsabila-106.png}
\begin{eulerprompt}
>r=linspace(0,1,10); a=linspace(0,2pi,40)'; z=r*exp(I*a);
>plot2d(exp(z),>points,>add):
\end{eulerprompt}
\eulerimg{21}{images/Plot2D_Nandhita Putri Shalsabila-107.png}
\begin{eulercomment}
Vektor bilangan kompleks secara otomatis diplot sebagai kurva pada
bidang kompleks dengan bagian nyata dan bagian imajiner.

Dalam contoh, kita memplot lingkaran satuan dengan\\
In the example, we plot the unit circle with

\end{eulercomment}
\begin{eulerformula}
\[
\gamma(t) = e^{it}
\]
\end{eulerformula}
\begin{eulerprompt}
>t=linspace(0,2pi,1000); ...
>plot2d(exp(I*t)+exp(4*I*t),r=2):
\end{eulerprompt}
\eulerimg{21}{images/Plot2D_Nandhita Putri Shalsabila-109.png}
\eulerheading{Plot Statistik}
\begin{eulercomment}
Ada banyak fungsi yang dikhususkan pada plot statistik. Salah satu
plot yang sering digunakan adalah plot kolom.

Jumlah kumulatif dari nilai terdistribusi normal 0-1 menghasilkan
jalan acak.
\end{eulercomment}
\begin{eulerprompt}
>plot2d(cumsum(randnormal(1,1000))):
\end{eulerprompt}
\eulerimg{21}{images/Plot2D_Nandhita Putri Shalsabila-110.png}
\begin{eulercomment}
Penggunaan dua baris menunjukkan jalan dalam dua dimensi.
\end{eulercomment}
\begin{eulerprompt}
>X=cumsum(randnormal(2,1000)); plot2d(X[1],X[2]):
\end{eulerprompt}
\eulerimg{21}{images/Plot2D_Nandhita Putri Shalsabila-111.png}
\begin{eulerprompt}
>columnsplot(cumsum(random(10)),style="/",color=blue):
\end{eulerprompt}
\eulerimg{21}{images/Plot2D_Nandhita Putri Shalsabila-112.png}
\begin{eulercomment}
Itu juga dapat menampilkan string sebagai label.
\end{eulercomment}
\begin{eulerprompt}
>months=["Jan","Feb","Mar","Apr","May","Jun", ...
>  "Jul","Aug","Sep","Oct","Nov","Dec"];
>values=[10,12,12,18,22,28,30,26,22,18,12,8];
>columnsplot(values,lab=months,color=red,style="-");
>title("Temperature"):
\end{eulerprompt}
\eulerimg{21}{images/Plot2D_Nandhita Putri Shalsabila-113.png}
\begin{eulerprompt}
>k=0:10;
>plot2d(k,bin(10,k),>bar):
\end{eulerprompt}
\eulerimg{21}{images/Plot2D_Nandhita Putri Shalsabila-114.png}
\begin{eulerprompt}
>plot2d(k,bin(10,k)); plot2d(k,bin(10,k),>points,>add):
\end{eulerprompt}
\eulerimg{21}{images/Plot2D_Nandhita Putri Shalsabila-115.png}
\begin{eulerprompt}
>plot2d(normal(1000),normal(1000),>points,grid=6,style=".."):
\end{eulerprompt}
\eulerimg{21}{images/Plot2D_Nandhita Putri Shalsabila-116.png}
\begin{eulerprompt}
>plot2d(normal(1,1000),>distribution,style="O"):
\end{eulerprompt}
\eulerimg{21}{images/Plot2D_Nandhita Putri Shalsabila-117.png}
\begin{eulerprompt}
>plot2d("qnormal",0,5;2.5,0.5,>filled):
\end{eulerprompt}
\eulerimg{21}{images/Plot2D_Nandhita Putri Shalsabila-118.png}
\begin{eulercomment}
Untuk memplot distribusi statistik eksperimental, Anda dapat
menggunakan distribution=n dengan plot2d.
\end{eulercomment}
\begin{eulerprompt}
>w=randexponential(1,1000); // exponential distribution
>plot2d(w,>distribution): // or distribution=n with n intervals
\end{eulerprompt}
\eulerimg{21}{images/Plot2D_Nandhita Putri Shalsabila-119.png}
\begin{eulercomment}
Atau Anda dapat menghitung distribusi dari data dan memplot hasilnya
dengan \textgreater{}bar di plot3d, atau dengan plot kolom.
\end{eulercomment}
\begin{eulerprompt}
>w=normal(1000); // 0-1-normal distribution
>\{x,y\}=histo(w,10,v=[-6,-4,-2,-1,0,1,2,4,6]); // interval bounds v
>plot2d(x,y,>bar):
\end{eulerprompt}
\eulerimg{21}{images/Plot2D_Nandhita Putri Shalsabila-120.png}
\begin{eulercomment}
Fungsi statplot() mengatur gaya dengan string sederhana.
\end{eulercomment}
\begin{eulerprompt}
>statplot(1:10,cumsum(random(10)),"b"):
\end{eulerprompt}
\eulerimg{21}{images/Plot2D_Nandhita Putri Shalsabila-121.png}
\begin{eulerprompt}
>n=10; i=0:n; ...
>plot2d(i,bin(n,i)/2^n,a=0,b=10,c=0,d=0.3); ...
>plot2d(i,bin(n,i)/2^n,points=true,style="ow",add=true,color=blue):
\end{eulerprompt}
\eulerimg{21}{images/Plot2D_Nandhita Putri Shalsabila-122.png}
\begin{eulercomment}
Selain itu, data dapat diplot sebagai batang. Dalam hal ini, x harus
diurutkan dan satu elemen lebih panjang dari y. Batangnya akan
memanjang dari x[i] hingga x[i+1] dengan nilai y[i]. Jika x berukuran
sama dengan y, maka x akan diperpanjang satu elemen dengan spasi
terakhir.

Gaya isian dapat digunakan seperti di atas.
\end{eulercomment}
\begin{eulerprompt}
>n=10; k=bin(n,0:n); ...
>plot2d(-0.5:n+0.5,k,bar=true,fillcolor=lightgray):
\end{eulerprompt}
\eulerimg{21}{images/Plot2D_Nandhita Putri Shalsabila-123.png}
\begin{eulercomment}
Data untuk plot batang (batang=1) dan histogram (histogram=1) dapat
diberikan secara eksplisit dalam xv dan yv, atau dapat dihitung dari
distribusi empiris dalam xv dengan \textgreater{}distribusi (atau distribusi=n).
Histogram nilai xv akan dihitung secara otomatis dengan \textgreater{}histogram.
Jika \textgreater{}even ditentukan, nilai xv akan dihitung dalam interval bilangan
bulat.
\end{eulercomment}
\begin{eulerprompt}
>plot2d(normal(10000),distribution=50):
\end{eulerprompt}
\eulerimg{21}{images/Plot2D_Nandhita Putri Shalsabila-124.png}
\begin{eulerprompt}
>k=0:10; m=bin(10,k); x=(0:11)-0.5; plot2d(x,m,>bar):
\end{eulerprompt}
\eulerimg{21}{images/Plot2D_Nandhita Putri Shalsabila-125.png}
\begin{eulerprompt}
>columnsplot(m,k):
\end{eulerprompt}
\eulerimg{21}{images/Plot2D_Nandhita Putri Shalsabila-126.png}
\begin{eulerprompt}
>plot2d(random(600)*6,histogram=6):
\end{eulerprompt}
\eulerimg{21}{images/Plot2D_Nandhita Putri Shalsabila-127.png}
\begin{eulercomment}
Untuk distribusi, terdapat parameter distribution=n, yang menghitung
nilai secara otomatis dan mencetak distribusi relatif dengan n
sub-interval.
\end{eulercomment}
\begin{eulerprompt}
>plot2d(normal(1,1000),distribution=10,style="\(\backslash\)/"):
\end{eulerprompt}
\eulerimg{21}{images/Plot2D_Nandhita Putri Shalsabila-128.png}
\begin{eulercomment}
Dengan parameter even=true, ini akan menggunakan interval bilangan
bulat.
\end{eulercomment}
\begin{eulerprompt}
>plot2d(intrandom(1,1000,10),distribution=10,even=true):
\end{eulerprompt}
\eulerimg{21}{images/Plot2D_Nandhita Putri Shalsabila-129.png}
\begin{eulercomment}
Perhatikan bahwa ada banyak plot statistik yang mungkin berguna.
Silahkan lihat tutorial tentang statistik.
\end{eulercomment}
\begin{eulerprompt}
>columnsplot(getmultiplicities(1:6,intrandom(1,6000,6))):
\end{eulerprompt}
\eulerimg{21}{images/Plot2D_Nandhita Putri Shalsabila-130.png}
\begin{eulerprompt}
>plot2d(normal(1,1000),>distribution); ...
>  plot2d("qnormal(x)",color=red,thickness=2,>add):
\end{eulerprompt}
\eulerimg{21}{images/Plot2D_Nandhita Putri Shalsabila-131.png}
\begin{eulercomment}
Ada juga banyak plot khusus untuk statistik. Plot kotak menunjukkan
kuartil distribusi ini dan banyak outlier. Menurut definisinya,
outlier dalam plot kotak adalah data yang melebihi 1,5 kali rentang
50\% tengah plot.
\end{eulercomment}
\begin{eulerprompt}
>M=normal(5,1000); boxplot(quartiles(M)):
\end{eulerprompt}
\eulerimg{21}{images/Plot2D_Nandhita Putri Shalsabila-132.png}
\eulerheading{Fungsi Implisit}
\begin{eulercomment}
Plot implisit menunjukkan penyelesaian garis level f(x,y)=level,
dengan "level" dapat berupa nilai tunggal atau vektor nilai. Jika
level = "auto", akan ada garis level nc, yang akan tersebar antara
fungsi minimum dan maksimum secara merata. Warna yang lebih gelap atau
lebih terang dapat ditambahkan dengan \textgreater{}hue untuk menunjukkan nilai
fungsi. Untuk fungsi implisit, xv harus berupa fungsi atau ekspresi
parameter x dan y, atau alternatifnya, xv dapat berupa matriks nilai.

Euler dapat menandai garis level

lateks: f(x,y) = c

dari fungsi apa pun.

Untuk menggambar himpunan f(x,y)=c untuk satu atau lebih konstanta c,
Anda dapat menggunakan plot2d() dengan plot implisitnya pada bidang.
Parameter c adalah level=c, dimana c dapat berupa vektor garis level.
Selain itu, skema warna dapat digambar di latar belakang untuk
menunjukkan nilai fungsi setiap titik dalam plot. Parameter "n"
menentukan kehalusan plot.
\end{eulercomment}
\begin{eulerprompt}
>aspect(1.5); 
>plot2d("x^2+y^2-x*y-x",r=1.5,level=0,contourcolor=red):
\end{eulerprompt}
\eulerimg{17}{images/Plot2D_Nandhita Putri Shalsabila-133.png}
\begin{eulerprompt}
>expr := "2*x^2+x*y+3*y^4+y"; // define an expression f(x,y)
>plot2d(expr,level=0): // Solutions of f(x,y)=0
\end{eulerprompt}
\eulerimg{17}{images/Plot2D_Nandhita Putri Shalsabila-134.png}
\begin{eulerprompt}
>plot2d(expr,level=0:0.5:20,>hue,contourcolor=white,n=200): // nice
\end{eulerprompt}
\eulerimg{17}{images/Plot2D_Nandhita Putri Shalsabila-135.png}
\begin{eulerprompt}
>plot2d(expr,level=0:0.5:20,>hue,>spectral,n=200,grid=4): // nicer
\end{eulerprompt}
\eulerimg{17}{images/Plot2D_Nandhita Putri Shalsabila-136.png}
\begin{eulercomment}
Ini juga berfungsi untuk plot data. Namun Anda harus menentukan
rentangnya\\
untuk label sumbu.
\end{eulercomment}
\begin{eulerprompt}
>x=-2:0.05:1; y=x'; z=expr(x,y);
>plot2d(z,level=0,a=-1,b=2,c=-2,d=1,>hue):
\end{eulerprompt}
\eulerimg{17}{images/Plot2D_Nandhita Putri Shalsabila-137.png}
\begin{eulerprompt}
>plot2d("x^3-y^2",>contour,>hue,>spectral):
\end{eulerprompt}
\eulerimg{17}{images/Plot2D_Nandhita Putri Shalsabila-138.png}
\begin{eulerprompt}
>plot2d("x^3-y^2",level=0,contourwidth=3,>add,contourcolor=red):
\end{eulerprompt}
\eulerimg{17}{images/Plot2D_Nandhita Putri Shalsabila-139.png}
\begin{eulerprompt}
>z=z+normal(size(z))*0.2;
>plot2d(z,level=0.5,a=-1,b=2,c=-2,d=1):
\end{eulerprompt}
\eulerimg{17}{images/Plot2D_Nandhita Putri Shalsabila-140.png}
\begin{eulerprompt}
>plot2d(expr,level=[0:0.2:5;0.05:0.2:5.05],color=lightgray):
\end{eulerprompt}
\eulerimg{17}{images/Plot2D_Nandhita Putri Shalsabila-141.png}
\begin{eulerprompt}
>plot2d("x^2+y^3+x*y",level=1,r=4,n=100):
\end{eulerprompt}
\eulerimg{17}{images/Plot2D_Nandhita Putri Shalsabila-142.png}
\begin{eulerprompt}
>plot2d("x^2+2*y^2-x*y",level=0:0.1:10,n=100,contourcolor=white,>hue):
\end{eulerprompt}
\eulerimg{17}{images/Plot2D_Nandhita Putri Shalsabila-143.png}
\begin{eulercomment}
Dimungkinkan juga untuk mengisi set

\end{eulercomment}
\begin{eulerformula}
\[
a \le f(x,y) \le b
\]
\end{eulerformula}
\begin{eulercomment}
dengan rentang level.

Dimungkinkan untuk mengisi wilayah nilai untuk fungsi tertentu. Untuk
ini, level harus berupa matriks 2xn. Baris pertama adalah batas bawah
dan baris kedua berisi batas atas.
\end{eulercomment}
\begin{eulerprompt}
>plot2d(expr,level=[0;1],style="-",color=blue): // 0 <= f(x,y) <= 1
\end{eulerprompt}
\eulerimg{17}{images/Plot2D_Nandhita Putri Shalsabila-145.png}
\begin{eulercomment}
Plot implisit juga dapat menunjukkan rentang level. Maka level harus
berupa matriks interval level 2xn, di mana baris pertama berisi awal
dan baris kedua berisi akhir setiap interval. Alternatifnya, vektor
baris sederhana dapat digunakan untuk level, dan parameter dl
memperluas nilai level ke interval.
\end{eulercomment}
\begin{eulerprompt}
>plot2d("x^4+y^4",r=1.5,level=[0;1],color=blue,style="/"):
\end{eulerprompt}
\eulerimg{17}{images/Plot2D_Nandhita Putri Shalsabila-146.png}
\begin{eulerprompt}
>plot2d("x^2+y^3+x*y",level=[0,2,4;1,3,5],style="/",r=2,n=100):
\end{eulerprompt}
\eulerimg{17}{images/Plot2D_Nandhita Putri Shalsabila-147.png}
\begin{eulerprompt}
>plot2d("x^2+y^3+x*y",level=-10:20,r=2,style="-",dl=0.1,n=100):
\end{eulerprompt}
\eulerimg{17}{images/Plot2D_Nandhita Putri Shalsabila-148.png}
\begin{eulerprompt}
>plot2d("sin(x)*cos(y)",r=pi,>hue,>levels,n=100):
\end{eulerprompt}
\eulerimg{17}{images/Plot2D_Nandhita Putri Shalsabila-149.png}
\begin{eulercomment}
Dimungkinkan juga untuk menandai suatu wilayah

\end{eulercomment}
\begin{eulerformula}
\[
a \le f(x,y) \le b.
\]
\end{eulerformula}
\begin{eulercomment}
Hal ini dilakukan dengan menambahkan level dengan dua baris.
\end{eulercomment}
\begin{eulerprompt}
>plot2d("(x^2+y^2-1)^3-x^2*y^3",r=1.3, ...
>  style="#",color=red,<outline, ...
>  level=[-2;0],n=100):
\end{eulerprompt}
\eulerimg{17}{images/Plot2D_Nandhita Putri Shalsabila-150.png}
\begin{eulercomment}
Dimungkinkan untuk menentukan level tertentu. Misalnya, kita dapat
memplot solusi persamaan seperti

\end{eulercomment}
\begin{eulerformula}
\[
x^3-xy+x^2y^2=6
\]
\end{eulerformula}
\begin{eulerprompt}
>plot2d("x^3-x*y+x^2*y^2",r=6,level=1,n=100):
\end{eulerprompt}
\eulerimg{17}{images/Plot2D_Nandhita Putri Shalsabila-152.png}
\begin{eulerprompt}
>function starplot1 (v, style="/", color=green, lab=none) ...
\end{eulerprompt}
\begin{eulerudf}
    if !holding() then clg; endif;
    w=window(); window(0,0,1024,1024);
    h=holding(1);
    r=max(abs(v))*1.2;
    setplot(-r,r,-r,r);
    n=cols(v); t=linspace(0,2pi,n);
    v=v|v[1]; c=v*cos(t); s=v*sin(t);
    cl=barcolor(color); st=barstyle(style);
    loop 1 to n
      polygon([0,c[#],c[#+1]],[0,s[#],s[#+1]],1);
      if lab!=none then
        rlab=v[#]+r*0.1;
        \{col,row\}=toscreen(cos(t[#])*rlab,sin(t[#])*rlab);
        ctext(""+lab[#],col,row-textheight()/2);
      endif;
    end;
    barcolor(cl); barstyle(st);
    holding(h);
    window(w);
  endfunction
\end{eulerudf}
\begin{eulercomment}
Tidak ada tanda centang kotak atau sumbu di sini. Selain itu, kami
menggunakan jendela penuh untuk plotnya.

Kami memanggil reset sebelum kami menguji plot ini untuk mengembalikan
default grafis. Ini tidak perlu dilakukan jika Anda yakin plot Anda
berhasil.
\end{eulercomment}
\begin{eulerprompt}
>reset; starplot1(normal(1,10)+5,color=red,lab=1:10):
\end{eulerprompt}
\eulerimg{27}{images/Plot2D_Nandhita Putri Shalsabila-153.png}
\begin{eulercomment}
Terkadang, Anda mungkin ingin merencanakan sesuatu yang plot2d tidak
bisa lakukan, tapi hampir.

Dalam fungsi berikut, kita membuat plot impuls logaritmik. plot2d
dapat melakukan plot logaritmik, tetapi tidak untuk batang impuls.
\end{eulercomment}
\begin{eulerprompt}
>function logimpulseplot1 (x,y) ...
\end{eulerprompt}
\begin{eulerudf}
    \{x0,y0\}=makeimpulse(x,log(y)/log(10));
    plot2d(x0,y0,>bar,grid=0);
    h=holding(1);
    frame();
    xgrid(ticks(x));
    p=plot();
    for i=-10 to 10;
      if i<=p[4] and i>=p[3] then
         ygrid(i,yt="10^"+i);
      endif;
    end;
    holding(h);
  endfunction
\end{eulerudf}
\begin{eulercomment}
Mari kita uji dengan nilai yang terdistribusi secara eksponensial.
\end{eulercomment}
\begin{eulerprompt}
>aspect(1.5); x=1:10; y=-log(random(size(x)))*200; ...
>logimpulseplot1(x,y):
\end{eulerprompt}
\eulerimg{17}{images/Plot2D_Nandhita Putri Shalsabila-154.png}
\begin{eulercomment}
Mari kita menganimasikan kurva 2D menggunakan plot langsung. Perintah
plot(x,y) hanya memplot kurva ke dalam jendela plot. setplot(a,b,c,d)
menyetel jendela ini.

Fungsi wait(0) memaksa plot muncul di jendela grafis. Jika tidak,
pengundian ulang akan dilakukan dalam interval waktu yang jarang.
\end{eulercomment}
\begin{eulerprompt}
>function animliss (n,m) ...
\end{eulerprompt}
\begin{eulerudf}
  t=linspace(0,2pi,500);
  f=0;
  c=framecolor(0);
  l=linewidth(2);
  setplot(-1,1,-1,1);
  repeat
    clg;
    plot(sin(n*t),cos(m*t+f));
    wait(0);
    if testkey() then break; endif;
    f=f+0.02;
  end;
  framecolor(c);
  linewidth(l);
  endfunction
\end{eulerudf}
\begin{eulercomment}
Tekan tombol apa saja untuk menghentikan animasi ini.
\end{eulercomment}
\begin{eulerprompt}
>animliss(2,3); // lihat hasilnya, jika sudah puas, tekan ENTER
\end{eulerprompt}
\eulerheading{Plot Logaritmik}
\begin{eulercomment}
EMT menggunakan parameter "logplot" untuk skala logaritmik.\\
Plot logaritma dapat diplot menggunakan skala logaritma di y dengan
logplot=1, atau menggunakan skala logaritma di x dan y dengan
logplot=2, atau di x dengan logplot=3.

\end{eulercomment}
\begin{eulerttcomment}
 - logplot=1: y-logaritma
 - logplot=2: x-y-logaritma
 - logplot=3: x-logaritma
\end{eulerttcomment}
\begin{eulerprompt}
>plot2d("exp(x^3-x)*x^2",1,5,logplot=1):
\end{eulerprompt}
\eulerimg{17}{images/Plot2D_Nandhita Putri Shalsabila-155.png}
\begin{eulercomment}
exp(x\textasciicircum{}3 - x) * x\textasciicircum{}2: fungsi yang akan diplot, di mana e adalah basis
logaritma natural. Fungsi ini menghitung nilai eksponensial dari x\textasciicircum{}3-x
dan kemudian mengalikan hasilnya dengan x\textasciicircum{}2\\
1, 5: Ini menunjukkan bahwa grafik akan diplot untuk nilai x dari 1
hingga 5.\\
logplot=1: Ini berarti bahwa sumbu y akan menggunakan skala
logaritmik, sementara sumbu x tetap dalam skala linear.
\end{eulercomment}
\begin{eulerprompt}
>plot2d("exp(x+sin(x))",0,100,logplot=1):
\end{eulerprompt}
\eulerimg{17}{images/Plot2D_Nandhita Putri Shalsabila-156.png}
\begin{eulercomment}
exp(x + sin(x)): fungsi eksponensial yang akan diplot, di mana\\
e adalah basis logaritma natural. Fungsi ini menghitung nilai
eksponensial dari x+sin(x).\\
0, 100: Ini menunjukkan bahwa grafik akan diplot untuk nilai x dari 0
hingga 100.\\
logplot=1: Ini berarti bahwa sumbu y akan menggunakan skala
logaritmik, sementara sumbu x tetap dalam skala linear.
\end{eulercomment}
\begin{eulerprompt}
>plot2d("exp(x+sin(x))",10,100,logplot=2):
\end{eulerprompt}
\eulerimg{17}{images/Plot2D_Nandhita Putri Shalsabila-157.png}
\begin{eulercomment}
Perintah ini akan menghasilkan grafik yang menunjukkan perilaku fungsi
eksponensial dalam rentang yang ditentukan, dengan penekanan pada
pertumbuhan fungsi yang cepat karena eksponensial, sambil menggunakan
skala logaritmik pada sumbu x.
\end{eulercomment}
\begin{eulerprompt}
>plot2d("gamma(x)",1,10,logplot=1):
\end{eulerprompt}
\eulerimg{17}{images/Plot2D_Nandhita Putri Shalsabila-158.png}
\begin{eulercomment}
Perintah ini akan menghasilkan grafik yang menunjukkan perilaku fungsi
gamma di atas rentang yang ditentukan, dengan penekanan pada
nilai-nilai yang lebih besar dengan menggunakan skala logaritmik untuk
sumbu y.
\end{eulercomment}
\begin{eulerprompt}
>plot2d("log(x*(2+sin(x/100)))",10,1000,logplot=3):
\end{eulerprompt}
\eulerimg{17}{images/Plot2D_Nandhita Putri Shalsabila-159.png}
\begin{eulercomment}
Perintah ini akan mengasilkan grafik yang menunjukkan perilaku fungsi
logaritma di atas rentang yang ditentukan, untuk melihat pola dan
pertumbuhan fungsi dengan lebih jelas.

Ini juga berfungsi dengan plot data.
\end{eulercomment}
\begin{eulerprompt}
>x=10^(1:20); y=x^2-x;
>plot2d(x,y,logplot=2):
\end{eulerprompt}
\eulerimg{17}{images/Plot2D_Nandhita Putri Shalsabila-160.png}
\begin{eulercomment}
Perintah ini akan menghasilkan grafik yang menunjukkan hubungan antara\\
x dan y dengan sumbu x dalam skala logaritmik, untuk melihat pola
pertumbuhan fungsi kuadrat dengan lebih jelas.

\end{eulercomment}
\eulersubheading{Latihan soal}
\begin{eulercomment}
1. buatlah plot 2d dari fungsi\\
\end{eulercomment}
\begin{eulerformula}
\[
y=a\cdot\frac {x^2}{a}
\]
\end{eulerformula}
\begin{eulercomment}
dengan a=4 dan interval dari x=-1 sampai x=1. lalu, buatlah plot
tersebut dalam grid 3 dan ketebalan 3!.
\end{eulercomment}
\begin{eulerprompt}
>a=4; expr &= exp(a*(x^2)/a);
>plot2d(expr,-1,1,grid=3,thickness=3):
\end{eulerprompt}
\eulerimg{27}{images/Plot2D_Nandhita Putri Shalsabila-162.png}
\begin{eulercomment}
2. buatlah plot 2d dari fungsi\\
\end{eulercomment}
\begin{eulerformula}
\[
y(x,a)={x^2}-a.x
\]
\end{eulerformula}
\begin{eulercomment}
dengan nilai parameternya adalah 5, dengan interval x nya dari 1
sampai 2.
\end{eulercomment}
\begin{eulerprompt}
>function f(x,a):=x^2-a.x
>a=5; plot2d("f",1,2;a):
\end{eulerprompt}
\eulerimg{27}{images/Plot2D_Nandhita Putri Shalsabila-164.png}
\begin{eulercomment}
3.selidiki dimanakah fungsi f(x) berikut naik dan turun\\
\end{eulercomment}
\begin{eulerformula}
\[
f(x)=(\frac 1 3) {x^3}-{x^2}-3x+4
\]
\end{eulerformula}
\begin{eulercomment}
untuk interval x = -4 sampai x = 5
\end{eulercomment}
\begin{eulerprompt}
>function f(x):= (1/3)*x^3-x^2-3x+4
>plot2d ("f",-4,5):
\end{eulerprompt}
\eulerimg{27}{images/Plot2D_Nandhita Putri Shalsabila-166.png}
\eulerheading{Rujukan Lengkap Fungsi plot2d()}
\begin{eulercomment}
\end{eulercomment}
\begin{eulerttcomment}
  function plot2d (xv, yv, btest, a, b, c, d, xmin, xmax, r, n,  ..
  logplot, grid, frame, framecolor, square, color, thickness, style, ..
  auto, add, user, delta, points, addpoints, pointstyle, bar, histogram,  ..
  distribution, even, steps, own, adaptive, hue, level, contour,  ..
  nc, filled, fillcolor, outline, title, xl, yl, maps, contourcolor, ..
  contourwidth, ticks, margin, clipping, cx, cy, insimg, spectral,  ..
  cgrid, vertical, smaller, dl, niveau, levels)
\end{eulerttcomment}
\begin{eulercomment}
Multipurpose plot function for plots in the plane (2D plots). This function can do
plots of functions of one variables, data plots, curves in the plane, bar plots, grids
of complex numbers, and implicit plots of functions of two variables.

Parameters
\\
x,y       : equations, functions or data vectors\\
a,b,c,d   : Plot area (default a=-2,b=2)\\
r         : if r is set, then a=cx-r, b=cx+r, c=cy-r, d=cy+r\\
\end{eulercomment}
\begin{eulerttcomment}
            r can be a vector [rx,ry] or a vector [rx1,rx2,ry1,ry2].
\end{eulerttcomment}
\begin{eulercomment}
xmin,xmax : range of the parameter for curves\\
auto      : Determine y-range automatically (default)\\
square    : if true, try to keep square x-y-ranges\\
n         : number of intervals (default is adaptive)\\
grid      : 0 = no grid and labels,\\
\end{eulercomment}
\begin{eulerttcomment}
            1 = axis only,
            2 = normal grid (see below for the number of grid lines)
            3 = inside axis
            4 = no grid
            5 = full grid including margin
            6 = ticks at the frame
            7 = axis only
            8 = axis only, sub-ticks
\end{eulerttcomment}
\begin{eulercomment}
frame     : 0 = no frame\\
framecolor: color of the frame and the grid\\
margin    : number between 0 and 0.4 for the margin around the plot\\
color     : Color of curves. If this is a vector of colors,\\
\end{eulercomment}
\begin{eulerttcomment}
            it will be used for each row of a matrix of plots. In the case of
            point plots, it should be a column vector. If a row vector or a
            full matrix of colors is used for point plots, it will be used for
            each data point.
\end{eulerttcomment}
\begin{eulercomment}
thickness : line thickness for curves\\
\end{eulercomment}
\begin{eulerttcomment}
            This value can be smaller than 1 for very thin lines.
\end{eulerttcomment}
\begin{eulercomment}
style     : Plot style for lines, markers, and fills.\\
\end{eulercomment}
\begin{eulerttcomment}
            For points use
            "[]", "<>", ".", "..", "...",
            "*", "+", "|", "-", "o"
            "[]#", "<>#", "o#" (filled shapes)
            "[]w", "<>w", "ow" (non-transparent)
            For lines use
            "-", "--", "-.", ".", ".-.", "-.-", "->"
            For filled polygons or bar plots use
            "#", "#O", "O", "/", "\(\backslash\)", "\(\backslash\)/",
            "+", "|", "-", "t"
\end{eulerttcomment}
\begin{eulercomment}
points    : plot single points instead of line segments\\
addpoints : if true, plots line segments and points\\
add       : add the plot to the existing plot\\
user      : enable user interaction for functions\\
delta     : step size for user interaction\\
bar       : bar plot (x are the interval bounds, y the interval values)\\
histogram : plots the frequencies of x in n subintervals\\
distribution=n : plots the distribution of x with n subintervals\\
even      : use inter values for automatic histograms.\\
steps     : plots the function as a step function (steps=1,2)\\
adaptive  : use adaptive plots (n is the minimal number of steps)\\
level     : plot level lines of an implicit function of two variables\\
outline   : draws boundary of level ranges.
\\
If the level value is a 2xn matrix, ranges of levels will be drawn\\
in the color using the given fill style. If outline is true, it\\
will be drawn in the contour color. Using this feature, regions of\\
f(x,y) between limits can be marked.
\\
hue       : add hue color to the level plot to indicate the function\\
\end{eulercomment}
\begin{eulerttcomment}
            value
\end{eulerttcomment}
\begin{eulercomment}
contour   : Use level plot with automatic levels\\
nc        : number of automatic level lines\\
title     : plot title (default "")\\
xl, yl    : labels for the x- and y-axis\\
smaller   : if \textgreater{}0, there will be more space to the left for labels.\\
vertical  :\\
\end{eulercomment}
\begin{eulerttcomment}
  Turns vertical labels on or off. This changes the global variable
  verticallabels locally for one plot. The value 1 sets only vertical
  text, the value 2 uses vertical numerical labels on the y axis.
\end{eulerttcomment}
\begin{eulercomment}
filled    : fill the plot of a curve\\
fillcolor : fill color for bar and filled curves\\
outline   : boundary for filled polygons\\
logplot   : set logarithmic plots\\
\end{eulercomment}
\begin{eulerttcomment}
            1 = logplot in y,
            2 = logplot in xy,
            3 = logplot in x
\end{eulerttcomment}
\begin{eulercomment}
own       :\\
\end{eulercomment}
\begin{eulerttcomment}
  A string, which points to an own plot routine. With >user, you get
  the same user interaction as in plot2d. The range will be set
  before each call to your function.
\end{eulerttcomment}
\begin{eulercomment}
maps      : map expressions (0 is faster), functions are always mapped.\\
contourcolor : color of contour lines\\
contourwidth : width of contour lines\\
clipping  : toggles the clipping (default is true)\\
title     :\\
\end{eulercomment}
\begin{eulerttcomment}
  This can be used to describe the plot. The title will appear above
  the plot. Moreover, a label for the x and y axis can be added with
  xl="string" or yl="string". Other labels can be added with the
  functions label() or labelbox(). The title can be a unicode
  string or an image of a Latex formula.
\end{eulerttcomment}
\begin{eulercomment}
cgrid     :\\
\end{eulercomment}
\begin{eulerttcomment}
  Determines the number of grid lines for plots of complex grids.
  Should be a divisor of the the matrix size minus 1 (number of
  subintervals). cgrid can be a vector [cx,cy].
\end{eulerttcomment}
\begin{eulercomment}

Overview

The function can plot

- expressions, call collections or functions of one variable,\\
- parametric curves,\\
- x data against y data,\\
- implicit functions,\\
- bar plots,\\
- complex grids,\\
- polygons.

If a function or expression for xv is given, plot2d() will compute\\
values in the given range using the function or expression. The\\
expression must be an expression in the variable x. The range must\\
be defined in the parameters a and b unless the default range\\
[-2,2] should be used. The y-range will be computed automatically,\\
unless c and d are specified, or a radius r, which yields the range\\
[-r,r] for x and y. For plots of functions, plot2d will use an\\
adaptive evaluation of the function by default. To speed up the\\
plot for complicated functions, switch this off with \textless{}adaptive, and\\
optionally decrease the number of intervals n. Moreover, plot2d()\\
will by default use mapping. I.e., it will compute the plot element\\
for element. If your expression or your functions can handle a\\
vector x, you can switch that off with \textless{}maps for faster evaluation.

Note that adaptive plots are always computed element for element. \\
If functions or expressions for both xv and for yv are specified,\\
plot2d() will compute a curve with the xv values as x-coordinates\\
and the yv values as y-coordinates. In this case, a range should be\\
defined for the parameter using xmin, xmax. Expressions contained\\
in strings must always be expressions in the parameter variable x.
\end{eulercomment}
\end{eulernotebook}
\end{document}
