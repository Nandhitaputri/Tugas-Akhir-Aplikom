\documentclass{article}

\usepackage{eumat}

\begin{document}
\begin{eulernotebook}
\begin{eulercomment}
Nama : Nandhita Putri Shalsabila\\
Nim : 23030630075\\
Kelas : Matematika B 2023


\begin{eulercomment}
\eulerheading{Menggambar Plot 3D dengan EMT}
\begin{eulercomment}
Ini adalah pengenalan plot 3D di Euler. Kita memerlukan plot 3D untuk
memvisualisasikan fungsi dua variabel.

Euler menggambar fungsi-fungsi tersebut dengan menggunakan algoritme
pengurutan untuk menyembunyikan bagian-bagian di latar belakang.
Secara umum, Euler menggunakan proyeksi pusat. Standarnya adalah dari
kuadran x-y positif ke arah asal x=y=z=0, tetapi sudut=0° terlihat
dari arah sumbu-y. Sudut pandang dan ketinggian dapat diubah.

Euler dapat memetakan\\
-permukaan dengan bayangan dan garis-garis level atau rentang level,\\
-awan titik-titik,\\
-kurva parametrik,\\
-permukaan implisit.

Plot 3D dari sebuah fungsi menggunakan plot3d. Cara termudah adalah
memplot ekspresi dalam x dan y. Parameter r mengatur rentang plot di
sekitar (0,0).
\end{eulercomment}
\begin{eulerprompt}
>aspect(1.5); plot3d("x^2+sin(y)",-5,5,0,6*pi):
\end{eulerprompt}
\eulerimg{17}{images/Plot3D_Nandhita Putri Shalsabila-001.png}
\begin{eulercomment}
aspect(1.5);\\
untuk mengatur rasio aspek dari plot. Dalam hal ini, aspek ditetapkan
menjadi 1.5, yang berarti rasio antara sumbu-sumbu grafik akan
disesuaikan agar proporsional.\\
plot3d("x\textasciicircum{}2+sin(y)", -5, 5, 0, 6*pi); untuk membuat grafik 3D dari
fungsi z=x\textasciicircum{}2 + sin(y), di mana:\\
x\textasciicircum{}2 + \textbackslash{}sin(y) adalah fungsi yang dipetakan ke sumbu z berdasarkan
nilai\\
x dan y.\\
-5 dan 5 adalah batas-batas untuk nilai x (dari -5 sampai 5).\\
0 dan 6*pi adalah batas-batas untuk nilai y (dari 0 hingga 6pi).
\end{eulercomment}
\begin{eulerprompt}
>plot3d("x^2+x*sin(y)",-5,5,0,6*pi):
\end{eulerprompt}
\eulerimg{17}{images/Plot3D_Nandhita Putri Shalsabila-002.png}
\begin{eulercomment}
plot3d("x\textasciicircum{}2+x*sin(y)", -5, 5, 0, 6*pi); untuk menghasilkan grafik 3D
dari fungsi z=x\textasciicircum{}2 + xsin(y), dengan:\\
Fungsi x\textasciicircum{}2 + xsin(y) yang diplot di sumbu z.\\
Batas nilai x dari -5 hingga 5.\\
Batas nilai y dari 0 hingga 6pi.

\end{eulercomment}
\eulersubheading{Latihan Soal}
\begin{eulercomment}
Silakan lakukan modifikasi agar gambar "talang bergelombang" tersebut
tidak lurus melainkan melengkung/melingkar, baik melingkar secara
mendatar maupun melingkar turun/naik (seperti papan peluncur pada
kolam renang. Temukan rumusnya.

1. talang bergelombang melengkung secara mendatar (koordinat polar)
\end{eulercomment}
\begin{eulerprompt}
>plot3d("x^2+x*sin(y)",0,5,0,2*pi):
\end{eulerprompt}
\eulerimg{17}{images/Plot3D_Nandhita Putri Shalsabila-003.png}
\begin{eulerprompt}
>plot3d("x^2+x*sin(y)",0,5,0,4*pi):
\end{eulerprompt}
\eulerimg{17}{images/Plot3D_Nandhita Putri Shalsabila-004.png}
\eulerheading{Fungsi dari dua Variabel}
\begin{eulercomment}
Untuk grafik sebuah fungsi, gunakan

-ekspresi sederhana dalam x dan y,\\
-nama fungsi dari dua variabell\\
-atau matriks data.

Standarnya adalah kisi-kisi kawat yang terisi dengan warna yang
berbeda di kedua sisi. Perhatikan bahwa jumlah default interval grid
adalah 10, namun plot menggunakan jumlah default 40x40 persegi panjang
untuk membangun permukaan. Hal ini dapat diubah.

-n=40, n=[40,40]: jumlah garis kisi di setiap arah\\
-grid=10, grid=[10,10]: jumlah garis grid di setiap arah.

Kami menggunakan default n=40 dan grid=10.
\end{eulercomment}
\begin{eulerprompt}
>plot3d("x^2+y^2"):
\end{eulerprompt}
\eulerimg{17}{images/Plot3D_Nandhita Putri Shalsabila-005.png}
\begin{eulercomment}
Fungsi z=x\textasciicircum{}2 + y\textasciicircum{}2 : fungsi kuadratik yang menghasilkan permukaan
berbentuk paraboloid. Nilai z akan meningkat secara simetris karena
x\textasciicircum{}2 dan y\textasciicircum{}2 selalu bernilai positif atau nol.\\
Plot3d: menghasilkan grafik 3D dari fungsi yang diberikan dalam
rentang default untuk x dan y (biasanya dari -5 hingga 5 jika tidak
ditentukan).

Interaksi pengguna dapat dilakukan dengan \textgreater{}user Parameter. Pengguna
dapat menekan tombol berikut ini.

- kiri, kanan, atas, bawah: memutar sudut pandang\\
- +,-: memperbesar atau memperkecil\\
- a: menghasilkan anaglyph (lihat di bawah)\\
- l: beralih memutar sumber cahaya (lihat di bawah)\\
- spasi: mengatur ulang ke default\\
- kembali: mengakhiri interaksi
\end{eulercomment}
\begin{eulerprompt}
>plot3d("exp(-x^2+y^2)",>user, ...
>  title="Turn with the vector keys (press return to finish)"):
\end{eulerprompt}
\eulerimg{17}{images/Plot3D_Nandhita Putri Shalsabila-006.png}
\begin{eulercomment}
Rentang plot untuk fungsi dapat ditentukan dengan

- a,b: rentang x\\
- c,d: rentang y\\
- r : persegi simetris di sekitar (0,0).\\
- n : jumlah subinterval untuk plot.

Ada beberapa parameter untuk menskalakan fungsi atau mengubah tampilan
grafik.

fscale: menskalakan ke nilai fungsi (defaultnya adalah \textless{}fscale).\\
skala: angka atau vektor 1x2 untuk menskalakan ke arah x dan y.\\
bingkai: jenis bingkai (default 1).
\end{eulercomment}
\begin{eulerprompt}
>plot3d("exp(-(x^2+y^2)/5)",r=10,n=80,fscale=4,scale=1.2,frame=3,>user):
\end{eulerprompt}
\eulerimg{17}{images/Plot3D_Nandhita Putri Shalsabila-007.png}
\begin{eulercomment}
Perintah ini menghasilkan grafik 3D dari fungsi Gaussian berbentuk
lonceng dengan distribusi yang diperlebar, rentang dari -10 hingga 10\\
pada sumbu x dan y, dan dengan resolusi tinggi (80 titik per sumbu).
Grafik ini juga diskalakan pada sumbu vertikal (fscale 4) dan
keseluruhan grafik diperbesar 1.2 kali.

Tampilan dapat diubah dengan berbagai cara.

- Jarak: jarak pandang ke plot.\\
- zoom: nilai zoom.\\
- sudut: sudut terhadap sumbu y negatif dalam radian.\\
- tinggi: ketinggian pandangan dalam radian.

Nilai default dapat diperiksa atau diubah dengan fungsi view(). Ini
mengembalikan parameter dalam urutan di atas.
\end{eulercomment}
\begin{eulerprompt}
>view
\end{eulerprompt}
\begin{euleroutput}
  [5,  2.6,  2,  0.4]
\end{euleroutput}
\begin{eulercomment}
Jarak yang lebih dekat membutuhkan lebih sedikit zoom. Efeknya lebih
seperti lensa sudut lebar.

Pada contoh berikut, sudut=0 dan tinggi=0 dilihat dari sumbu y
negatif. Label sumbu untuk y disembunyikan dalam kasus ini.
\end{eulercomment}
\begin{eulerprompt}
>plot3d("x^2+y",distance=3,zoom=1,angle=pi/2,height=0):
\end{eulerprompt}
\eulerimg{17}{images/Plot3D_Nandhita Putri Shalsabila-008.png}
\begin{eulercomment}
Perintah ini menghasilkan grafik 3D dari fungsi z=x\textasciicircum{}2 + y dengan sudut
pandang khusus:\\
Kamera berada pada jarak 3 unit dari grafik.\\
Grafik ditampilkan tanpa pembesaran (zoom 1).\\
Kamera diputar 90 derajat (sudut pi/2), sehingga tampilan grafik lebih
dari samping.\\
Kamera ditempatkan sejajar dengan bidang horizontal (ketinggian 0).

Plot selalu terlihat berada di tengah kubus plot. Anda dapat
memindahkan bagian tengah dengan parameter tengah.
\end{eulercomment}
\begin{eulerprompt}
>plot3d("x^4+y^2",a=0,b=1,c=-1,d=1,angle=-20°,height=20°, ...
>  center=[0.4,0,0],zoom=5):
\end{eulerprompt}
\eulerimg{17}{images/Plot3D_Nandhita Putri Shalsabila-009.png}
\begin{eulercomment}
Perintah ini menghasilkan grafik 3D dari fungsi z=x\textasciicircum{}4 + y\textasciicircum{}2 dengan
aturan sebagai berikut:\\
Rentang nilai x dari 0 hingga 1 dan nilai y dari -1 hingga 1.\\
Sudut pandang sedikit diputar (-20 derajat) dan kamera diposisikan 20
derajat di atas bidang horizontal.\\
Grafik berpusat di titik (0.4, 0, 0).\\
Grafik diperbesar lima kali dari ukuran normal.

Plotnya diskalakan agar sesuai dengan unit kubus untuk dilihat. Jadi
tidak perlu mengubah jarak atau zoom tergantung ukuran plot. Namun
labelnya mengacu pada ukuran sebenarnya.

Jika Anda mematikannya dengan scale=false, Anda harus berhati-hati
agar plot tetap masuk ke dalam jendela plotting, dengan mengubah jarak
pandang atau zoom, dan memindahkan bagian tengah.
\end{eulercomment}
\begin{eulerprompt}
>plot3d("5*exp(-x^2-y^2)",r=2,<fscale,<scale,distance=13,height=50°, ...
>  center=[0,0,-2],frame=3):
\end{eulerprompt}
\eulerimg{17}{images/Plot3D_Nandhita Putri Shalsabila-010.png}
\begin{eulercomment}
Perintah ini menghasilkan grafik 3D dari fungsi
z=5.exp(-x\textasciicircum{}2-y\textasciicircum{}2)dengan aturan sebagai berikut:\\
Rentang nilai x dan y dari -2 hingga 2.\\
Kamera diposisikan pada jarak 13 unit dari grafik dan pada ketinggian
50 derajat, memberikan sudut pandang yang baik.\\
Grafik berpusat di titik (0, 0, -2).\\
Skala fungsi dan skala keseluruhan grafik menggunakan nilai default.\\
Tampilan dengan kerangka atau grid 3D yang membantu visualisasi.

Plot kutub juga tersedia. Parameter polar=true menggambar plot kutub.
Fungsi tersebut harus tetap merupakan fungsi dari x dan y. Parameter
"fscale" menskalakan fungsi dengan skalanya sendiri. Kalau tidak,
fungsinya akan diskalakan agar sesuai dengan kubus.
\end{eulercomment}
\begin{eulerprompt}
>plot3d("1/(x^2+y^2+1)",r=5,>polar, ...
>fscale=2,>hue,n=100,zoom=4,>contour,color=blue):
\end{eulerprompt}
\eulerimg{17}{images/Plot3D_Nandhita Putri Shalsabila-011.png}
\begin{eulercomment}
Perintah ini menghasilkan grafik 3D dari fungsi \\
z=1/(x\textasciicircum{}2 + y\textasciicircum{}2 + 1)dengan aturan sebagai berikut:\\
Rentang nilai x dan y dari -5 hingga 5.\\
Grafik ditampilkan dalam koordinat polar.\\
Skala vertikal diperbesar dua kali (fscale=2).\\
Warna permukaan berdasarkan hue (nuansa).\\
100 titik digunakan untuk resolusi yang lebih halus.\\
Grafik diperbesar empat kali dari ukuran normal.\\
Garis kontur juga ditampilkan pada grafik.\\
Warna grafik ditetapkan ke biru.
\end{eulercomment}
\begin{eulerprompt}
>function f(r) := exp(-r/2)*cos(r); ...
>plot3d("f(x^2+y^2)",>polar,scale=[1,1,0.4],r=pi,frame=3,zoom=4):
\end{eulerprompt}
\eulerimg{17}{images/Plot3D_Nandhita Putri Shalsabila-012.png}
\begin{eulercomment}
Perintah ini menghasilkan grafik 3D dari fungsi\\
f(r)=e\textasciicircum{}-r/2 . cos(r) dengan aturan sebagai berikut:\\
Fungsi z ditentukan dengan menggunakan r=x\textasciicircum{}2 + y\textasciicircum{}2\\
Grafik ditampilkan dalam koordinat polar, sehingga lebih intuitif
untuk fungsi radial.\\
Skala pada sumbu z dikurangi menjadi 0.4 untuk membuat tampilan lebih
datar.\\
Rentang nilai r dibatasi hingga pi.\\
Tampilan dengan kerangka 3D yang lebih terperinci.\\
Grafik diperbesar empat kali untuk detail yang lebih baik.

Parameter memutar memutar fungsi di x di sekitar sumbu x.

- putar=1: Menggunakan sumbu x\\
- putar=2: Menggunakan sumbu z
\end{eulercomment}
\begin{eulerprompt}
>plot3d("x^2+1",a=-1,b=1,rotate=true,grid=5):
\end{eulerprompt}
\eulerimg{17}{images/Plot3D_Nandhita Putri Shalsabila-013.png}
\begin{eulercomment}
plot3d: digunakan untuk menggambar grafik 3D.\\
"x\textasciicircum{}2+1": ekspresi fungsi matematika yang akan digambar, yaitu y=x\textasciicircum{}2 +
1. Fungsi ini menggambarkan parabola yang digeser satu unit ke atas.\\
a=-1, b=1: batas rentang x. Dalam hal ini, x dibatasi dari -1 hingga
1.\\
rotate=true: Opsi ini memungkinkan grafik berputar sehingga pengguna
bisa melihatnya dari berbagai sudut.\\
Grid=5: mengatur jumlah grid atau partisi pada sumbu untuk membantu
visualisasi. Nilai 5 menunjukkan grafik akan dibagi menjadi 5 bagian
pada setiap sumbu.
\end{eulercomment}
\begin{eulerprompt}
>plot3d("x^2+1",a=-1,b=1,rotate=2,grid=5):
\end{eulerprompt}
\eulerimg{17}{images/Plot3D_Nandhita Putri Shalsabila-014.png}
\begin{eulercomment}
plot3d: digunakan untuk menggambar grafik 3D.\\
"x\textasciicircum{}2+1": ekspresi fungsi matematika yang akan digambar, yaitu y=x\textasciicircum{}2+1,
sebuah parabola yang digeser satu unit ke atas.\\
a=-1, b=1: rentang untuk variabel x, yang dibatasi dari -1 hingga 1.\\
rotate=2: Nilai rotate=2 menentukan rotasi awal grafik 3D pada
tampilan. Berbeda dengan rotate=true yang memungkinkan rotasi
interaktif, rotate=2 merujuk pada sudut atau posisi tetap rotasi
tertentu yang telah ditentukan.\\
grid=5: Mengatur jumlah grid atau partisi pada sumbu untuk membantu
visualisasi. Grid akan dibagi menjadi 5 bagian per sumbu, memberikan
tampilan grafik dengan detail yang lebih sederhana.
\end{eulercomment}
\begin{eulerprompt}
>plot3d("sqrt(25-x^2)",a=0,b=5,rotate=1):
\end{eulerprompt}
\eulerimg{17}{images/Plot3D_Nandhita Putri Shalsabila-015.png}
\begin{eulercomment}
plot3d: untuk menggambar grafik 3D.\\
sqrt(25-x\textasciicircum{}2)": fungsi matematika yang digambarkan, yaitu y= sqrt
(25x\textasciicircum{}2). Fungsi ini adalah setengah lingkaran dengan jari-jari 5 pada
bidang xy, didefinisikan hanya untuk x dari -5 sampai 5 karena nilai
dalam akar kuadrat harus positif.\\
a=0, b=5: menentukan rentang x dari 0 hingga 5. Karena fungsi ini
merepresentasikan setengah lingkaran, rentang xhanya setengah dari
lingkaran tersebut (bagian positif).\\
rotate=1: menentukan rotasi awal grafik 3D, memberikan sudut pandang
tertentu pada grafik saat ditampilkan.
\end{eulercomment}
\begin{eulerprompt}
>plot3d("x*sin(x)",a=0,b=6pi,rotate=2):
\end{eulerprompt}
\eulerimg{17}{images/Plot3D_Nandhita Putri Shalsabila-016.png}
\begin{eulercomment}
plot3d: digunakan untuk menggambar grafik 3D.\\
"x*sin(x)": ekspresi fungsi matematika yang akan digambar, yaitu\\
y=xsin(x). Fungsi ini menghasilkan grafik gelombang sinus yang
amplitudonya berubah seiring dengan bertambahnya nilai x.\\
a=0, b=6*pi: rentang untuk variabel x. Nilai x diatur dari 0 hingga\\
6pi(sekitar 18.85), yang mencakup beberapa siklus dari fungsi sinus.\\
rotate=2: Parameter ini menentukan rotasi awal grafik dalam tampilan
3D. Nilai rotate=2 berarti grafik akan dimulai dengan sudut atau
posisi rotasi tertentu yang telah ditetapkan.

Berikut adalah plot dengan tiga fungsi.
\end{eulercomment}
\begin{eulerprompt}
>plot3d("x","x^2+y^2","y",r=2,zoom=3.5,frame=3):
\end{eulerprompt}
\eulerimg{17}{images/Plot3D_Nandhita Putri Shalsabila-017.png}
\begin{eulercomment}
plot3d: digunakan untuk menggambar grafik 3D.\\
"x", "x\textasciicircum{}2+y\textasciicircum{}2", "y": tiga ekspresi fungsi matematika yang
merepresentasikan tiga dimensi (sumbu x,z, dan y) dari grafik 3D.\\
x berfungsi sebagai sumbu x,\\
x\textasciicircum{}2+ y\textasciicircum{}2 menggambarkan ketinggian (sumbu z, biasanya terkait dengan
fungsi atau permukaan),\\
y merepresentasikan sumbu y.\\
Secara keseluruhan, ini menggambarkan permukaan 3D yang berbentuk
paraboloid (karena x\textasciicircum{}2 + y\textasciicircum{}2 adalah persamaan paraboloid di ruang 3D).\\
r=2: parameter untuk mengatur radius dari grid yang dipetakan dalam
grafik, sehingga grafik akan diplot dalam rentang dengan jari-jari 2
unit.\\
zoom=3.5: mengatur tingkat pembesaran (zoom) pada grafik. Nilai ini
memperbesar atau memperkecil tampilan grafik agar lebih jelas.\\
frame=3: Parameter ini menentukan tipe bingkai yang akan digunakan
untuk grafik. Nilai frame=3 mungkin menunjukkan pengaturan tertentu
untuk bingkai grafik, seperti bagaimana sumbu-sumbu ditampilkan atau
bagaimana grid ditata di sekitar grafik.

\begin{eulercomment}
\eulerheading{Plot Kontur}
\begin{eulercomment}
Untuk plotnya, Euler menambahkan garis grid. Sebaliknya, dimungkinkan
untuk menggunakan garis datar dan rona satu warna atau rona warna
spektral. Euler dapat menggambar ketinggian fungsi pada plot dengan
arsiran. Di semua plot 3D, Euler dapat menghasilkan anaglyph
merah/cyan.

-\textgreater{}hue: Mengaktifkan bayangan cahaya, bukan kabel.\\
-\textgreater{}kontur: Membuat plot garis kontur otomatis pada plot.\\
- level=... (atau level): Vektor nilai garis kontur.

Standarnya adalah level="auto", yang menghitung beberapa garis level
secara otomatis. Seperti yang Anda lihat di plot, level-level tersebut
sebenarnya adalah rentang level.

Gaya default dapat diubah. Untuk plot kontur berikut, kami menggunakan
grid yang lebih halus dengan ukuran 100x100 titik, menskalakan fungsi
dan plot, dan menggunakan sudut pandang yang berbeda.
\end{eulercomment}
\begin{eulerprompt}
>plot3d("exp(-x^2-y^2)",r=2,n=100,level="thin", ...
> >contour,>spectral,fscale=1,scale=1.1,angle=45°,height=20°):
\end{eulerprompt}
\eulerimg{17}{images/Plot3D_Nandhita Putri Shalsabila-018.png}
\begin{eulercomment}
plot3d: digunakan untuk menggambar grafik 3D.\\
"exp(-x\textasciicircum{}2-y\textasciicircum{}2)": ekspresi fungsi matematika yang akan digambar. Fungsi
ini adalah fungsi Gaussian 2D (fungsi lonceng) yang memiliki puncak di\\
(0,0) dan menurun dengan cepat menuju nol saat x\textasciicircum{}2 + y\textasciicircum{}2 bertambah.
Grafik ini menggambarkan permukaan yang simetris berbentuk lonceng.\\
r=2: mengatur rentang x dan y dari -2 hingga 2. Grafik akan diplot
dalam lingkup jari-jari 2 unit pada sumbu x dan y.\\
n=100: Ini menentukan resolusi grid, dengan 100 titik di sepanjang
setiap sumbu. Semakin tinggi nilainya, semakin halus grafik yang
dihasilkan.\\
level="thin": Ini mengatur tampilan level (kontur) yang lebih tipis
pada grafik, sehingga garis kontur yang dihasilkan lebih halus dan
lebih presisi.\\
\textgreater{}contour: menambahkan garis kontur pada grafik 3D. Garis-garis kontur
digunakan untuk menunjukkan nilai-nilai yang sama pada permukaan
grafik, memberikan informasi tambahan tentang bentuk fungsi pada
bidang tertentu.\\
\textgreater{}spectral: mengatur spektrum warna untuk grafik, yang berarti warna
yang digunakan akan didasarkan pada skala spektral, biasanya warna
pelangi atau gradasi warna lainnya.\\
fscale=1: mengatur faktor skala untuk kontur dan grafik ke nilai
standar (tanpa pembesaran atau pengecilan tambahan).\\
scale=1.1: memperbesar grafik secara keseluruhan sebesar 1.1 kali
lipat, sehingga grafik terlihat sedikit lebih besar dari ukuran
aslinya.\\
angle=45°: mengatur sudut rotasi tampilan grafik 3D sebesar 45 derajat
di sekitar sumbu vertikal (sumbu z), sehingga grafik akan terlihat
dari sudut miring.\\
height=20°: Ini mengatur sudut elevasi atau ketinggian dari mana
grafik dilihat. Dalam hal ini, grafik akan dilihat dari sudut 20
derajat di atas bidang horizontal.
\end{eulercomment}
\begin{eulerprompt}
>plot3d("exp(x*y)",angle=100°,>contour,color=green):
\end{eulerprompt}
\eulerimg{17}{images/Plot3D_Nandhita Putri Shalsabila-019.png}
\begin{eulercomment}
plot3d: digunakan untuk menggambar grafik 3D.\\
"exp(x*y)": ekspresi fungsi matematika yang akan digambar, yaitu \\
z=exp(x.y). Fungsi ini eksponensial di mana variabel\\
x dan y saling dikalikan. Grafik ini akan menunjukkan peningkatan yang
cepat di satu arah dan penurunan di arah yang lain, membentuk
permukaan yang cukup curam di beberapa area.\\
angle=100°: mengatur sudut rotasi tampilan grafik 3D sebesar 100
derajat di sekitar sumbu vertikal (z). Grafik akan diputar untuk
dilihat dari sudut ini, memungkinkan tampilan yang berbeda dari
grafik.\\
\textgreater{}contour: Perintah ini menambahkan garis kontur pada grafik 3D. Garis
kontur digunakan untuk menunjukkan nilai-nilai yang sama pada
permukaan grafik, yang membantu dalam memahami variasi elevasi atau
perubahan fungsi dalam bidang tertentu.\\
color=green: Ini mengatur warna grafik menjadi hijau. Grafik dan
kontur akan ditampilkan menggunakan skema warna hijau.

Bayangan defaultnya menggunakan warna abu-abu. Namun rentang warna
spektral juga tersedia.

-\textgreater{}spektral: Menggunakan skema spektral default\\
- color=...: Menggunakan warna khusus atau skema spektral

Untuk plot berikut, kami menggunakan skema spektral default dan
menambah jumlah titik untuk mendapatkan tampilan yang sangat mulus.
\end{eulercomment}
\begin{eulerprompt}
>plot3d("x^2+y^2",>spectral,>contour,n=100):
\end{eulerprompt}
\eulerimg{17}{images/Plot3D_Nandhita Putri Shalsabila-020.png}
\begin{eulercomment}
Selain garis level otomatis, kita juga dapat menetapkan nilai garis
level. Ini akan menghasilkan garis level yang tipis, bukan rentang
level.
\end{eulercomment}
\begin{eulerprompt}
>plot3d("x^2-y^2",0,5,0,5,level=-1:0.1:1,color=redgreen):
\end{eulerprompt}
\eulerimg{17}{images/Plot3D_Nandhita Putri Shalsabila-021.png}
\begin{eulercomment}
Dalam plot berikut, kita menggunakan dua pita tingkat yang sangat luas
dari -0,1 hingga 1, dan dari 0,9 hingga 1. Ini dimasukkan sebagai
matriks dengan batas tingkat sebagai kolom.

Selain itu, kami melapisi grid dengan 10 interval di setiap arah.
\end{eulercomment}
\begin{eulerprompt}
>plot3d("x^2+y^3",level=[-0.1,0.9;0,1], ...
>  >spectral,angle=30°,grid=10,contourcolor=gray):
\end{eulerprompt}
\eulerimg{17}{images/Plot3D_Nandhita Putri Shalsabila-022.png}
\begin{eulercomment}
Pada contoh berikut, kita memplot himpunan, di mana

\end{eulercomment}
\begin{eulerformula}
\[
f(x,y) = x^y-y^x = 0
\]
\end{eulerformula}
\begin{eulercomment}
Kami menggunakan satu garis tipis untuk garis level.
\end{eulercomment}
\begin{eulerprompt}
>plot3d("x^y-y^x",level=0,a=0,b=6,c=0,d=6,contourcolor=red,n=100):
\end{eulerprompt}
\eulerimg{17}{images/Plot3D_Nandhita Putri Shalsabila-023.png}
\begin{eulercomment}
Dimungkinkan untuk menampilkan bidang kontur di bawah plot. Warna dan
jarak ke plot dapat ditentukan.
\end{eulercomment}
\begin{eulerprompt}
>plot3d("x^2+y^4",>cp,cpcolor=green,cpdelta=0.2):
\end{eulerprompt}
\eulerimg{17}{images/Plot3D_Nandhita Putri Shalsabila-024.png}
\begin{eulercomment}
Berikut beberapa gaya lainnya. Kami selalu mematikan bingkai, dan
menggunakan berbagai skema warna untuk plot dan kisi.
\end{eulercomment}
\begin{eulerprompt}
>figure(2,2); ...
>expr="y^3-x^2"; ...
>figure(1);  ...
>  plot3d(expr,<frame,>cp,cpcolor=spectral); ...
>figure(2);  ...
>  plot3d(expr,<frame,>spectral,grid=10,cp=2); ...
>figure(3);  ...
>  plot3d(expr,<frame,>contour,color=gray,nc=5,cp=3,cpcolor=greenred); ...
>figure(4);  ...
>  plot3d(expr,<frame,>hue,grid=10,>transparent,>cp,cpcolor=gray); ...
>figure(0):
\end{eulerprompt}
\eulerimg{17}{images/Plot3D_Nandhita Putri Shalsabila-025.png}
\begin{eulercomment}
Ada beberapa skema spektral lainnya, yang diberi nomor dari 1 hingga
9. Namun Anda juga dapat menggunakan warna=nilai, di mana nilai

- spektral: untuk rentang dari biru hingga merah\\
- putih: untuk rentang yang lebih redup\\
- kuningbiru,unguhijau,birukuning,hijaumerah\\
- birukuning, hijauungu,kuningbiru,merahhijau
\end{eulercomment}
\begin{eulerprompt}
>figure(3,3); ...
>for i=1:9;  ...
>  figure(i); plot3d("x^2+y^2",spectral=i,>contour,>cp,<frame,zoom=4);  ...
>end; ...
>figure(0):
\end{eulerprompt}
\eulerimg{17}{images/Plot3D_Nandhita Putri Shalsabila-026.png}
\begin{eulercomment}
Sumber cahaya dapat diubah dengan l dan tombol kursor selama interaksi
pengguna. Itu juga dapat diatur dengan parameter.

- cahaya : arah datangnya cahaya\\
- amb: cahaya sekitar antara 0 dan 1

Perhatikan bahwa program ini tidak membuat perbedaan antara sisi plot.
Tidak ada bayangan. Untuk ini, Anda memerlukan Povray.
\end{eulercomment}
\begin{eulerprompt}
>plot3d("-x^2-y^2", ...
>  hue=true,light=[0,1,1],amb=0,user=true, ...
>  title="Press l and cursor keys (return to exit)"):
\end{eulerprompt}
\eulerimg{17}{images/Plot3D_Nandhita Putri Shalsabila-027.png}
\begin{eulercomment}
Parameter warna mengubah warna permukaan. Warna garis level juga bisa
diubah.
\end{eulercomment}
\begin{eulerprompt}
>plot3d("-x^2-y^2",color=rgb(0.2,0.2,0),hue=true,frame=false, ...
>  zoom=3,contourcolor=red,level=-2:0.1:1,dl=0.01):
\end{eulerprompt}
\eulerimg{17}{images/Plot3D_Nandhita Putri Shalsabila-028.png}
\begin{eulercomment}
Warna 0 memberikan efek pelangi yang istimewa.
\end{eulercomment}
\begin{eulerprompt}
>plot3d("x^2/(x^2+y^2+1)",color=0,hue=true,grid=10):
\end{eulerprompt}
\eulerimg{17}{images/Plot3D_Nandhita Putri Shalsabila-029.png}
\begin{eulercomment}
Permukaannya juga bisa transparan.
\end{eulercomment}
\begin{eulerprompt}
>plot3d("x^2+y^2",>transparent,grid=10,wirecolor=red):
\end{eulerprompt}
\eulerimg{17}{images/Plot3D_Nandhita Putri Shalsabila-030.png}
\eulerheading{Plot Implisit}
\begin{eulercomment}
Ada juga plot implisit dalam tiga dimensi. Euler menghasilkan
pemotongan melalui objek. Fitur plot3d mencakup plot implisit. Plot
ini menunjukkan himpunan nol suatu fungsi dalam tiga variabel.\\
Solusi dari

\end{eulercomment}
\begin{eulerformula}
\[
f(x,y,z) = 0
\]
\end{eulerformula}
\begin{eulercomment}
dapat divisualisasikan dalam potongan yang sejajar dengan bidang x-y-,
x-z- dan y-z.

- implisit=1: dipotong sejajar bidang y-z\\
- implisit=2: dipotong sejajar dengan bidang x-z\\
- implisit=4: dipotong sejajar bidang x-y

Tambahkan nilai-nilai ini, jika Anda mau. Dalam contoh kita memplot\\
\end{eulercomment}
\begin{eulerformula}
\[
M = \{ (x,y,z) : x^2+y^3+zy=1 \}
\]
\end{eulerformula}
\begin{eulerprompt}
>plot3d("x^2+y^3+z*y-1",r=5,implicit=3):
\end{eulerprompt}
\eulerimg{17}{images/Plot3D_Nandhita Putri Shalsabila-031.png}
\begin{eulerprompt}
>c=1; d=1;
>plot3d("((x^2+y^2-c^2)^2+(z^2-1)^2)*((y^2+z^2-c^2)^2+(x^2-1)^2)*((z^2+x^2-c^2)^2+(y^2-1)^2)-d",r=2,<frame,>implicit,>user): 
\end{eulerprompt}
\eulerimg{17}{images/Plot3D_Nandhita Putri Shalsabila-032.png}
\begin{eulercomment}
Perintah ini menggambarkan grafik 3D dari persamaan implisit yang
cukup kompleks, yang melibatkan fungsi-fungsi kuadrat dan konstanta c
dan d. Rentang plot dibatasi dari -2 hingga 2 di setiap sumbu, dan
grafik ini ditampilkan dengan bingkai (frame) sumbu koordinat.
Persamaan ini menghasilkan bentuk geometris yang rumit dalam ruang 3D
yang mungkin menyerupai permukaan non-linear.
\end{eulercomment}
\begin{eulerprompt}
>plot3d("x^2+y^2+4*x*z+z^3",>implicit,r=2,zoom=2.5):
\end{eulerprompt}
\eulerimg{17}{images/Plot3D_Nandhita Putri Shalsabila-033.png}
\begin{eulercomment}
Perintah ini menggambarkan grafik tiga dimensi dari persamaan implisit
x\textasciicircum{}2 + y\textasciicircum{}2 + 4*x*z + z\textasciicircum{}3, di mana grafik diplot dalam rentang dari -2
hingga 2 pada setiap sumbu, dengan tingkat pembesaran 2.5 kali. Grafik
ini akan menunjukkan bentuk geometris non-linear yang kompleks dalam
ruang 3D sesuai dengan persamaan tersebut.

\begin{eulercomment}
\eulerheading{Latihan soal}
\begin{eulercomment}
1. gambarlah fungsi implisit berikut dalam 3D

\end{eulercomment}
\begin{eulerformula}
\[
f(x,y,z)=x^2+y^2-z^2-1
\]
\end{eulerformula}
\begin{eulerprompt}
>plot3d("x^2+y^2-z^2-1",r=8,implicit=3):
\end{eulerprompt}
\eulerimg{17}{images/Plot3D_Nandhita Putri Shalsabila-034.png}
\begin{eulercomment}
2. gambarlah fungsi 3Ddari fungsi implisit verikut ini

\end{eulercomment}
\begin{eulerformula}
\[
f(x,y,z)=xy+x^3y^2+xz^3-9=0
\]
\end{eulerformula}
\begin{eulercomment}
dengan r=4
\end{eulercomment}
\begin{eulerprompt}
>plot3d("x*y+x^3*y^2+x*z^3-9", r=4, implicit=3):
\end{eulerprompt}
\eulerimg{17}{images/Plot3D_Nandhita Putri Shalsabila-035.png}
\eulerheading{Merencanakan Data 3D}
\begin{eulercomment}
Sama seperti plot2d, plot3d menerima data. Untuk objek 3D, Anda perlu
menyediakan matriks nilai x-, y- dan z, atau tiga fungsi atau ekspresi
fx(x,y), fy(x,y), fz(x,y).

\end{eulercomment}
\begin{eulerformula}
\[
\gamma(t,s) = (x(t,s),y(t,s),z(t,s))
\]
\end{eulerformula}
\begin{eulercomment}
Karena x,y,z adalah matriks, kita asumsikan bahwa (t,s) melewati grid
persegi. Hasilnya, Anda dapat memplot gambar persegi panjang di ruang
angkasa.

Anda dapat menggunakan bahasa matriks Euler untuk menghasilkan
koordinat secara efektif.

Dalam contoh berikut, kita menggunakan vektor nilai t dan vektor kolom
nilai s untuk membuat parameter permukaan bola. Dalam gambar kita
dapat menandai wilayah, dalam kasus kita wilayah kutub.
\end{eulercomment}
\begin{eulerprompt}
>t=linspace(0,2pi,180); s=linspace(-pi/2,pi/2,90)'; ...
>x=cos(s)*cos(t); y=cos(s)*sin(t); z=sin(s); ...
>plot3d(x,y,z,>hue, ...
>color=blue,<frame,grid=[10,20], ...
>values=s,contourcolor=red,level=[90°-24°;90°-22°], ...
>scale=1.4,height=50°):
\end{eulerprompt}
\eulerimg{17}{images/Plot3D_Nandhita Putri Shalsabila-036.png}
\begin{eulercomment}
Berikut ini contohnya yaitu grafik suatu fungsi.
\end{eulercomment}
\begin{eulerprompt}
>t=-1:0.1:1; s=(-1:0.1:1)'; plot3d(t,s,t*s,grid=10):
\end{eulerprompt}
\eulerimg{17}{images/Plot3D_Nandhita Putri Shalsabila-037.png}
\begin{eulercomment}
Namun, kita bisa membuat berbagai macam permukaan. Berikut adalah
permukaan yang sama sebagai suatu fungsi

\end{eulercomment}
\begin{eulerformula}
\[
x = y \, z
\]
\end{eulerformula}
\begin{eulerprompt}
>plot3d(t*s,t,s,angle=180°,grid=10):
\end{eulerprompt}
\eulerimg{17}{images/Plot3D_Nandhita Putri Shalsabila-038.png}
\begin{eulercomment}
Dengan lebih banyak usaha, kita dapat menghasilkan banyak permukaan.

Dalam contoh berikut kita membuat tampilan bayangan dari bola yang
terdistorsi. Koordinat bola yang biasa adalah

\end{eulercomment}
\begin{eulerformula}
\[
\gamma(t,s) = (\cos(t)\cos(s),\sin(t)\sin(s),\cos(s))
\]
\end{eulerformula}
\begin{eulercomment}
dengan

\end{eulercomment}
\begin{eulerformula}
\[
0 \le t \le 2\pi, \quad \frac{-\pi}{2} \le s \le \frac{\pi}{2}.
\]
\end{eulerformula}
\begin{eulercomment}
Kami mendistorsi ini dengan sebuah faktor

\end{eulercomment}
\begin{eulerformula}
\[
d(t,s) = \frac{\cos(4t)+\cos(8s)}{4}.
\]
\end{eulerformula}
\begin{eulerprompt}
>t=linspace(0,2pi,320); s=linspace(-pi/2,pi/2,160)'; ...
>d=1+0.2*(cos(4*t)+cos(8*s)); ...
>plot3d(cos(t)*cos(s)*d,sin(t)*cos(s)*d,sin(s)*d,hue=1, ...
>  light=[1,0,1],frame=0,zoom=5):
\end{eulerprompt}
\eulerimg{17}{images/Plot3D_Nandhita Putri Shalsabila-039.png}
\begin{eulercomment}
Tentu saja, point cloud juga dimungkinkan. Untuk memplot data titik
dalam ruang, kita memerlukan tiga vektor untuk koordinat titik-titik
tersebut.

Gayanya sama seperti di plot2d dengan points=true;
\end{eulercomment}
\begin{eulerprompt}
>n=500;  ...
>  plot3d(normal(1,n),normal(1,n),normal(1,n),points=true,style="."):
\end{eulerprompt}
\eulerimg{17}{images/Plot3D_Nandhita Putri Shalsabila-040.png}
\begin{eulercomment}
Dimungkinkan juga untuk memplot kurva dalam 3D. Dalam hal ini, lebih
mudah untuk menghitung terlebih dahulu titik-titik kurva. Untuk kurva
pada bidang kita menggunakan barisan koordinat dan parameter
wire=true.
\end{eulercomment}
\begin{eulerprompt}
>t=linspace(0,8pi,500); ...
>plot3d(sin(t),cos(t),t/10,>wire,zoom=3):
\end{eulerprompt}
\eulerimg{17}{images/Plot3D_Nandhita Putri Shalsabila-041.png}
\begin{eulercomment}
Perintah ini menggambar spiral 3D menggunakan fungsi sinus dan cosinus
dengan variasi di sumbu x, y, dan z. Grafiknya diplot sebagai kurva
wireframe (tanpa permukaan) dengan zoom sebesar 3 kali. Ini
menghasilkan grafik yang menyerupai heliks atau spiral, karena nilai
sin(t) dan cos(t) membentuk lingkaran dalam ruang x-y, sementara t/10
secara linear meningkat pada sumbu z.
\end{eulercomment}
\begin{eulerprompt}
>t=linspace(0,4pi,1000); plot3d(cos(t),sin(t),t/2pi,>wire, ...
>linewidth=3,wirecolor=blue):
\end{eulerprompt}
\eulerimg{17}{images/Plot3D_Nandhita Putri Shalsabila-042.png}
\begin{eulercomment}
Perintah ini menggambar spiral 3D dalam ruang menggunakan fungsi
cosinus dan sinus untuk menghasilkan kurva melingkar dalam bidang x-y,
sementara variabel t/2p memberikan pergerakan linier pada sumbu z.
Grafik diplot sebagai kurva wireframe dengan ketebalan garis 3 dan
warna biru, sehingga menghasilkan spiral biru yang tebal dalam ruang
3D.
\end{eulercomment}
\begin{eulerprompt}
>X=cumsum(normal(3,100)); ...
> plot3d(X[1],X[2],X[3],>anaglyph,>wire):
\end{eulerprompt}
\eulerimg{17}{images/Plot3D_Nandhita Putri Shalsabila-043.png}
\begin{eulercomment}
Perintah ini menggambar grafik 3D dari jalur acak yang dihasilkan
berdasarkan distribusi normal dalam ruang 3D, diplot sebagai kurva
wireframe (hanya garis) dengan efek anaglyph untuk memberikan
pengalaman visual 3D. Jalur ini dihasilkan melalui jumlah kumulatif
dari vektor-vektor acak, yang menghasilkan jalur yang terus terhubung
tetapi acak.

EMT juga dapat membuat plot dalam mode anaglyph. Untuk melihat plot
seperti itu, Anda memerlukan kacamata berwarna merah/sian.
\end{eulercomment}
\begin{eulerprompt}
> plot3d("x^2+y^3",>anaglyph,>contour,angle=30°):
\end{eulerprompt}
\eulerimg{17}{images/Plot3D_Nandhita Putri Shalsabila-044.png}
\begin{eulercomment}
Perintah ini digunakan untuk menggambar grafik 3D dari fungsi x\textasciicircum{}2 +
y\textasciicircum{}3, dengan tampilan dalam mode anaglyph (untuk efek 3D), menampilkan
garis kontur untuk menunjukkan level ketinggian pada permukaan, dan
dengan sudut pandang 30 derajat. Hasilnya adalah tampilan 3D yang
memungkinkan pengguna melihat bentuk permukaan dan variasi ketinggian
melalui garis-garis kontur.

Seringkali skema warna spektral digunakan untuk plot. Ini menekankan
ketinggian fungsinya.
\end{eulercomment}
\begin{eulerprompt}
>plot3d("x^2*y^3-y",>spectral,>contour,zoom=3.2):
\end{eulerprompt}
\eulerimg{17}{images/Plot3D_Nandhita Putri Shalsabila-045.png}
\begin{eulercomment}
Perintah ini menggambar grafik 3D dari fungsi x\textasciicircum{}2*y\textasciicircum{}3 - y dalam ruang
tiga dimensi, dengan tampilan warna spektral yang mewakili ketinggian
pada permukaan, serta garis kontur untuk membantu visualisasi level
ketinggian. Pembesaran (zoom) diatur pada 3.2 kali lipat, memungkinkan
tampilan lebih dekat pada grafik. Hasilnya adalah grafik permukaan
non-linear yang ditampilkan secara visual dengan warna dan kontur
untuk menunjukkan perbedaan ketinggian di permukaan.

Euler juga dapat memplot permukaan yang diparameterisasi, jika
parameternya adalah nilai x, y, dan z dari gambar kotak persegi
panjang di ruang tersebut.

Untuk demo berikut, kami menyiapkan parameter u- dan v-, dan
menghasilkan koordinat ruang dari parameter tersebut.
\end{eulercomment}
\begin{eulerprompt}
>u=linspace(-1,1,10); v=linspace(0,2*pi,50)'; ...
>X=(3+u*cos(v/2))*cos(v); Y=(3+u*cos(v/2))*sin(v); Z=u*sin(v/2); ...
\end{eulerprompt}
\begin{eulercomment}
Perintah ini menghasilkan titik-titik dalam ruang 3D yang digunakan
untuk membentuk permukaan kompleks. Parameter u dan v digunakan untuk
mendefinisikan koordinat X, Y, dan Z berdasarkan fungsi trigonometrik,
yang memungkinkan penciptaan bentuk yang menarik. Setelah ini, Anda
dapat menggunakan fungsi plot3d untuk memvisualisasikan permukaan ini.
Permukaan yang dihasilkan akan memiliki bentuk yang menarik dan
berputar, mungkin menyerupai struktur seperti torus atau permukaan
kompleks lainnya tergantung pada rumus yang digunakan.
\end{eulercomment}
\begin{eulerprompt}
>plot3d(X,Y,Z,>anaglyph,<frame,>wire,scale=2.3):
\end{eulerprompt}
\eulerimg{17}{images/Plot3D_Nandhita Putri Shalsabila-046.png}
\begin{eulercomment}
Perintah ini digunakan untuk menggambar grafik 3D dari titik-titik
yang telah dihitung sebelumnya menggunakan fungsi plot3d, dengan
tampilan dalam mode anaglyph untuk efek 3D, menampilkan bingkai di
sekitar grafik, dan menggunakan representasi wireframe untuk menyoroti
struktur permukaan. Skala 2,3 kali memperbesar tampilan grafik,
memungkinkan pengguna untuk melihat detail lebih dekat pada bentuk
yang dihasilkan. Ini sangat berguna untuk menganalisis bentuk kompleks
atau permukaan dalam ruang 3D.

Berikut adalah contoh yang lebih rumit, yang megah dengan kacamata
merah/cyan.
\end{eulercomment}
\begin{eulerprompt}
>u:=linspace(-pi,pi,160); v:=linspace(-pi,pi,400)';  ...
>x:=(4*(1+.25*sin(3*v))+cos(u))*cos(2*v); ...
>y:=(4*(1+.25*sin(3*v))+cos(u))*sin(2*v); ...
> z=sin(u)+2*cos(3*v); ...
>plot3d(x,y,z,frame=0,scale=1.5,hue=1,light=[1,0,-1],zoom=2.8,>anaglyph):
\end{eulerprompt}
\eulerimg{17}{images/Plot3D_Nandhita Putri Shalsabila-047.png}
\begin{eulercomment}
Perintah ini menghasilkan grafik 3D dari permukaan yang kompleks
berdasarkan fungsi x, y, dan z yang telah ditentukan dengan parameter
yang menciptakan bentuk menarik. Grafik tersebut diplot dalam mode
anaglyph untuk efek 3D, tanpa bingkai, dengan skala dan zoom yang
ditingkatkan, serta pencahayaan yang diatur untuk meningkatkan
visualisasi. Hasilnya adalah tampilan 3D yang menarik dan kaya warna,
memungkinkan analisis bentuk dan struktur permukaan yang dihasilkan
oleh fungsi tersebut.

\begin{eulercomment}
\eulerheading{Plot Statistik}
\begin{eulercomment}
Plot batang juga dimungkinkan. Untuk itu, kita harus menyediakannya

- x: vektor baris dengan n+1 elemen\\
- y: vektor kolom dengan n+1 elemen\\
- z: matriks nilai nxn.

z bisa lebih besar, tetapi hanya nilai nxn yang akan digunakan.

Dalam contoh ini, pertama-tama kita menghitung nilainya. Kemudian kita
sesuaikan x dan y, sehingga vektor-vektornya berpusat pada nilai yang
digunakan.
\end{eulercomment}
\begin{eulerprompt}
>x=-1:0.1:1; y=x'; z=x^2+y^2; ...
>xa=(x|1.1)-0.05; ya=(y_1.1)-0.05; ...
>plot3d(xa,ya,z,bar=true):
\end{eulerprompt}
\eulerimg{17}{images/Plot3D_Nandhita Putri Shalsabila-048.png}
\begin{eulercomment}
Dimungkinkan untuk membagi plot suatu permukaan menjadi dua bagian
atau lebih.
\end{eulercomment}
\begin{eulerprompt}
>x=-1:0.1:1; y=x'; z=x+y; d=zeros(size(x)); ...
>plot3d(x,y,z,disconnect=2:2:20):
\end{eulerprompt}
\eulerimg{17}{images/Plot3D_Nandhita Putri Shalsabila-049.png}
\begin{eulercomment}
Jika memuat atau menghasilkan matriks data M dari file dan perlu
memplotnya dalam 3D, Anda dapat menskalakan matriks ke [-1,1] dengan
skala(M), atau menskalakan matriks dengan \textgreater{}zscale. Hal ini dapat
dikombinasikan dengan faktor penskalaan individual yang diterapkan
sebagai tambahan.
\end{eulercomment}
\begin{eulerprompt}
>i=1:20; j=i'; ...
>plot3d(i*j^2+100*normal(20,20),>zscale,scale=[1,1,1.5],angle=-40°,zoom=1.8):
\end{eulerprompt}
\eulerimg{17}{images/Plot3D_Nandhita Putri Shalsabila-050.png}
\begin{eulercomment}
Perintah ini menghasilkan grafik 3D dari permukaan yang ditentukan
oleh kombinasi variabel i dan j dengan penambahan noise acak dari
distribusi normal. Grafik ini diplot dengan penyesuaian skala pada
sumbu z untuk memberikan visualisasi yang lebih baik dari ketinggian,
dengan sudut dan zoom yang diatur untuk meningkatkan tampilan
keseluruhan. Hasilnya adalah permukaan yang kompleks dengan variasi
acak, memberikan representasi visual yang menarik dan informatif.
\end{eulercomment}
\begin{eulerprompt}
>Z=intrandom(5,100,6); v=zeros(5,6); ...
>loop 1 to 5; v[#]=getmultiplicities(1:6,Z[#]); end; ...
>columnsplot3d(v',scols=1:5,ccols=[1:5]):
\end{eulerprompt}
\eulerimg{17}{images/Plot3D_Nandhita Putri Shalsabila-051.png}
\begin{eulercomment}
Perintah ini digunakan untuk menghasilkan matriks acak dari angka
bulat, menghitung frekuensi kemunculan angka 1 hingga 6 dalam setiap
baris dari matriks tersebut, dan kemudian memvisualisasikan frekuensi
tersebut dalam bentuk grafik 3D. Hasil grafik ini akan memberikan
pemahaman yang lebih baik tentang distribusi angka yang dihasilkan
secara acak dalam matriks Z.

\begin{eulercomment}
\eulerheading{Permukaan Benda Putar}
\begin{eulerprompt}
>plot2d("(x^2+y^2-1)^3-x^2*y^3",r=1.3, ...
>style="#",color=red,<outline, ...
>level=[-2;0],n=100):
\end{eulerprompt}
\eulerimg{17}{images/Plot3D_Nandhita Putri Shalsabila-052.png}
\begin{eulercomment}
Secara keseluruhan, perintah ini digunakan untuk menggambar grafik 2D
dari fungsi implicit yang kompleks, dengan parameter untuk menentukan
gaya garis, warna, level, dan jumlah titik yang digunakan untuk
membentuk kurva. Hasilnya akan menjadi visualisasi yang jelas dan
informatif tentang bentuk fungsi yang digambarkan, dengan detail pada
dua level yang ditentukan.
\end{eulercomment}
\begin{eulerprompt}
>ekspresi &= (x^2+y^2-1)^3-x^2*y^3; $ekspresi
\end{eulerprompt}
\begin{eulerformula}
\[
\left(y^2+x^2-1\right)^3-x^2\,y^3
\]
\end{eulerformula}
\begin{eulercomment}
Kami ingin memutar kurva hati di sekitar sumbu y. Inilah ungkapan yang
mendefinisikan hati:

\end{eulercomment}
\begin{eulerformula}
\[
f(x,y)=(x^2+y^2-1)^3-x^2.y^3.
\]
\end{eulerformula}
\begin{eulercomment}
Selanjutnya kita atur

\end{eulercomment}
\begin{eulerformula}
\[
x=r.cos(a),\quad y=r.sin(a).
\]
\end{eulerformula}
\begin{eulerprompt}
>function fr(r,a) &= ekspresi with [x=r*cos(a),y=r*sin(a)] | trigreduce; $fr(r,a)
\end{eulerprompt}
\begin{eulerformula}
\[
\left(r^2-1\right)^3+\frac{\left(\sin \left(5\,a\right)-\sin \left(  3\,a\right)-2\,\sin a\right)\,r^5}{16}
\]
\end{eulerformula}
\begin{eulercomment}
Hal ini memungkinkan untuk mendefinisikan fungsi numerik, yang
menyelesaikan r, jika a diberikan. Dengan fungsi tersebut kita dapat
memplot jantung yang diputar sebagai permukaan parametrik.
\end{eulercomment}
\begin{eulerprompt}
>function map f(a) := bisect("fr",0,2;a); ...
>t=linspace(-pi/2,pi/2,100); r=f(t);  ...
>s=linspace(pi,2pi,100)'; ...
>plot3d(r*cos(t)*sin(s),r*cos(t)*cos(s),r*sin(t), ...
>>hue,<frame,color=red,zoom=4,amb=0,max=0.7,grid=12,height=50°):
\end{eulerprompt}
\eulerimg{17}{images/Plot3D_Nandhita Putri Shalsabila-055.png}
\begin{eulercomment}
Berikut ini adalah plot 3D dari gambar di atas yang diputar
mengelilingi sumbu z. Kami mendefinisikan fungsi yang mendeskripsikan
objek.
\end{eulercomment}
\begin{eulerprompt}
>function f(x,y,z) ...
\end{eulerprompt}
\begin{eulerudf}
  r=x^2+y^2;
  return (r+z^2-1)^3-r*z^3;
   endfunction
\end{eulerudf}
\begin{eulerprompt}
>plot3d("f(x,y,z)", ...
>xmin=0,xmax=1.2,ymin=-1.2,ymax=1.2,zmin=-1.2,zmax=1.4, ...
>implicit=1,angle=-30°,zoom=2.5,n=[10,100,60],>anaglyph):
\end{eulerprompt}
\eulerimg{17}{images/Plot3D_Nandhita Putri Shalsabila-056.png}
\begin{eulercomment}
perintah ini memplot grafik 3D dengan\\
batas x dari 0 hingga 1.2,\\
batas y -1.2 hingga 1.2,\\
batas z dari -1.2 hingga 1.4,\\
dengan implisit 1\\
angel 30 derajat berarti sudut pandangnya 30 derajat,\\
grafik di zoom 2.5

\begin{eulercomment}
\eulerheading{Plot 3D Khusus}
\begin{eulercomment}
Fungsi plot3d bagus untuk dimiliki, tetapi tidak memenuhi semua
kebutuhan. Selain rutinitas yang lebih mendasar, dimungkinkan untuk
mendapatkan plot berbingkai dari objek apa pun yang Anda suka.

Meskipun Euler bukan program 3D, ia dapat menggabungkan beberapa objek
dasar. Kami mencoba memvisualisasikan paraboloid dan garis
singgungnya.
\end{eulercomment}
\begin{eulerprompt}
>function myplot ...
\end{eulerprompt}
\begin{eulerudf}
    y=-1:0.01:1; x=(-1:0.01:1)';
    plot3d(x,y,0.2*(x-0.1)/2,<scale,<frame,>hue, ..
      hues=0.5,>contour,color=orange);
    h=holding(1);
    plot3d(x,y,(x^2+y^2)/2,<scale,<frame,>contour,>hue);
    holding(h);
  endfunction
\end{eulerudf}
\begin{eulercomment}
Sekarang framedplot() menyediakan bingkai, dan mengatur tampilan.
\end{eulercomment}
\begin{eulerprompt}
>framedplot("myplot",[-1,1,-1,1,0,1],height=0,angle=-30°, ...
>  center=[0,0,-0.7],zoom=3):
\end{eulerprompt}
\eulerimg{17}{images/Plot3D_Nandhita Putri Shalsabila-057.png}
\begin{eulercomment}
Dengan cara yang sama, Anda dapat memplot bidang kontur secara manual.
Perhatikan bahwa plot3d() menyetel jendela ke fullwindow() secara
default, tetapi plotcontourplane() berasumsi demikian.
\end{eulercomment}
\begin{eulerprompt}
>x=-1:0.02:1.1; y=x'; z=x^2-y^4;
>function myplot (x,y,z) ...
\end{eulerprompt}
\begin{eulerudf}
    zoom(2);
    wi=fullwindow();
    plotcontourplane(x,y,z,level="auto",<scale);
    plot3d(x,y,z,>hue,<scale,>add,color=white,level="thin");
    window(wi);
    reset();
  endfunction
\end{eulerudf}
\begin{eulerprompt}
>myplot(x,y,z):
\end{eulerprompt}
\eulerimg{27}{images/Plot3D_Nandhita Putri Shalsabila-058.png}
\begin{eulercomment}
perintah ini mendefinisikan fungsi yang menggambar grafik kontur dan
grafik 3D dari fungsi matematis z=x\textasciicircum{}2-y\textasciicircum{}4. Fungsi ini memungkinkan
pengguna untuk dengan mudah menghasilkan visualisasi dari permukaan
dan kontur dalam satu langkah, dengan pengaturan tampilan yang telah
disesuaikan. Hasilnya adalah representasi grafis yang informatif
tentang hubungan antara variabel x, y, dan z.

\begin{eulercomment}
\eulerheading{Animasi}
\begin{eulercomment}
Euler dapat menggunakan frame untuk melakukan pra-komputasi animasi.

Salah satu fungsi yang memanfaatkan teknik ini adalah memutar. Itu
dapat mengubah sudut pandang dan menggambar ulang plot 3D. Fungsi ini
memanggil addpage() untuk setiap plot baru. Akhirnya ia menganimasikan
plotnya.

Silakan pelajari sumber rotasi untuk melihat lebih detail.
\end{eulercomment}
\begin{eulerprompt}
>function testplot () := plot3d("x^2+y^3"); ...
>rotate("testplot"); testplot():
\end{eulerprompt}
\eulerimg{27}{images/Plot3D_Nandhita Putri Shalsabila-059.png}
\begin{eulercomment}
Secara keseluruhan, perintah ini menghasilkan grafik 3D dari fungsi
z=x\textasciicircum{}2+y\textasciicircum{}3 melalui fungsi testplot, kemudian memutar tampilan grafik
tersebut untuk memberikan perspektif yang berbeda. Dengan cara ini,
pengguna dapat menjelajahi visualisasi dari permukaan tersebut dengan
lebih interaktif, melihat fitur dan karakteristik dari grafik dalam
ruang 3D.

\begin{eulercomment}
\eulerheading{Menggambar Povray}
\begin{eulercomment}
Dengan bantuan file Euler povray.e, Euler dapat menghasilkan file
Povray. Hasilnya sangat bagus untuk dilihat.

Anda perlu menginstal Povray (32bit atau 64bit) dari
http://www.povray.org/, dan meletakkan sub-direktori "bin" Povray ke jalur lingkungan, atau mengatur variabel "defaultpovray" dengan jalur lengkap yang mengarah ke "pvengine.exe".

Antarmuka Povray Euler menghasilkan file Povray di direktori home
pengguna, dan memanggil Povray untuk menguraikan file-file ini. Nama
file default adalah current.pov, dan direktori default adalah
eulerhome(), biasanya c:\textbackslash{}Users\textbackslash{}Username\textbackslash{}Euler. Povray menghasilkan
file PNG, yang dapat dimuat oleh Euler ke dalam notebook. Untuk
membersihkan file-file ini, gunakan povclear().

Fungsi pov3d memiliki semangat yang sama dengan plot3d. Ini dapat
menghasilkan grafik fungsi f(x,y), atau permukaan dengan koordinat
X,Y,Z dalam matriks, termasuk garis level opsional. Fungsi ini memulai
raytracer secara otomatis, dan memuat adegan ke dalam notebook Euler.

Selain pov3d(), ada banyak fungsi yang menghasilkan objek Povray.
Fungsi-fungsi ini mengembalikan string, yang berisi kode Povray untuk
objek. Untuk menggunakan fungsi ini, mulai file Povray dengan
povstart(). Kemudian gunakan writeln(...) untuk menulis objek ke file
adegan. Terakhir, akhiri file dengan povend(). Secara default,
raytracer akan dimulai, dan PNG akan dimasukkan ke dalam notebook
Euler.

Fungsi objek memiliki parameter yang disebut "tampilan", yang
memerlukan string dengan kode Povray untuk tekstur dan penyelesaian
objek. Fungsi povlook() dapat digunakan untuk menghasilkan string ini.
Ini memiliki parameter untuk warna, transparansi, Phong Shading dll.

Perhatikan bahwa alam semesta Povray memiliki sistem koordinat lain.
Antarmuka ini menerjemahkan semua koordinat ke sistem Povray. Jadi
Anda dapat terus berpikir dalam sistem koordinat Euler dengan z
menunjuk vertikal ke atas, dan sumbu x,y,z di tangan kanan.\\
Anda perlu memuat file povray.
\end{eulercomment}
\begin{eulerprompt}
>load povray;
\end{eulerprompt}
\begin{eulercomment}
Pastikan, direktori Povray bin ada di jalurnya. Jika tidak, edit
variabel berikut sehingga berisi jalur ke povray yang dapat
dieksekusi.
\end{eulercomment}
\begin{eulerprompt}
>defaultpovray="C:\(\backslash\)Program Files\(\backslash\)POV-Ray\(\backslash\)v3.7\(\backslash\)bin\(\backslash\)pvengine.exe"
\end{eulerprompt}
\begin{euleroutput}
  C:\(\backslash\)Program Files\(\backslash\)POV-Ray\(\backslash\)v3.7\(\backslash\)bin\(\backslash\)pvengine.exe
\end{euleroutput}
\begin{eulercomment}
Untuk kesan pertama, kami memplot fungsi sederhana. Perintah berikut
menghasilkan file povray di direktori pengguna Anda, dan menjalankan
Povray untuk penelusuran sinar file ini.

Jika Anda memulai perintah berikut, GUI Povray akan terbuka,
menjalankan file, dan menutup secara otomatis. Karena alasan keamanan,
Anda akan ditanya apakah Anda ingin mengizinkan file exe dijalankan.
Anda dapat menekan batal untuk menghentikan pertanyaan lebih lanjut.
Anda mungkin harus menekan OK di jendela Povray untuk mengonfirmasi
dialog pengaktifan Povray.
\end{eulercomment}
\begin{eulerprompt}
>plot3d("x^2+y^2",zoom=2):
\end{eulerprompt}
\eulerimg{27}{images/Plot3D_Nandhita Putri Shalsabila-060.png}
\begin{eulercomment}
Secara keseluruhan, perintah ini digunakan untuk menghasilkan grafik
3D dari fungsi Z=x\textasciicircum{}2+y\textasciicircum{}2 dengan tingkat zoom senilai 2.
\end{eulercomment}
\begin{eulerprompt}
>pov3d("x^2+y^2",zoom=3);
\end{eulerprompt}
\eulerimg{27}{images/Plot3D_Nandhita Putri Shalsabila-061.png}
\begin{eulercomment}
Secara keseluruhan, perintah ini digunakan untuk menghasilkan grafik
3D dari fungsi z=x\textasciicircum{}2+y\textasciicircum{}2 dengan tingkat zoom senilai 3.

Kita dapat membuat fungsinya transparan dan menambahkan penyelesaian
lainnya. Kita juga dapat menambahkan garis level ke plot fungsi.
\end{eulercomment}
\begin{eulerprompt}
>pov3d("x^2+y^3",axiscolor=red,angle=-45°,>anaglyph, ...
>  look=povlook(cyan,0.2),level=-1:0.5:1,zoom=3.8);
\end{eulerprompt}
\eulerimg{27}{images/Plot3D_Nandhita Putri Shalsabila-062.png}
\begin{eulerprompt}
>pov3d("((x-1)^2+y^2)*((x+1)^2+y^2)/40",r=2, ...
>  angle=-120°,level=1/40,dlevel=0.005,light=[-1,1,1],height=10°,n=50, ...
>  <fscale,zoom=3.8);
\end{eulerprompt}
\eulerimg{27}{images/Plot3D_Nandhita Putri Shalsabila-063.png}
\begin{eulercomment}
Terkadang perlu untuk mencegah penskalaan fungsi, dan menskalakan
fungsi secara manual.

Kita memplot himpunan titik pada bidang kompleks, dimana hasil kali
jarak ke 1 dan -1 sama dengan 1.

\end{eulercomment}
\eulersubheading{Latihan Soal}
\begin{eulerprompt}
>pov3d("((x-1)^2+y^2)*((x+1)^2+y^2)/40",r=1.5, ...
>  angle=120°,level=1/40,dlevel=0.005,light=[-1,1,1],height=10°,n=50, ...
>  <fscale,zoom=3.8);
\end{eulerprompt}
\eulerimg{27}{images/Plot3D_Nandhita Putri Shalsabila-064.png}
\eulerheading{Merencanakan dengan Koordinat}
\begin{eulercomment}
Daripada menggunakan fungsi, kita bisa memplotnya dengan koordinat.
Seperti di plot3d, kita memerlukan tiga matriks untuk mendefinisikan
objek.

Dalam contoh ini kita memutar suatu fungsi di sekitar sumbu z.
\end{eulercomment}
\begin{eulerprompt}
>function f(x) := x^3-x+1; ...
>x=-1:0.01:1; t=linspace(0,2pi,50)'; ...
>Z=x; X=cos(t)*f(x); Y=sin(t)*f(x); ...
>pov3d(X,Y,Z,angle=40°,look=povlook(red,0.1),height=50°,axis=0,zoom=4,light=[10,5,15]);
\end{eulerprompt}
\eulerimg{27}{images/Plot3D_Nandhita Putri Shalsabila-065.png}
\begin{eulercomment}
Pada contoh berikut, kita memplot gelombang teredam. Kami menghasilkan
gelombang dengan bahasa matriks Euler.

Kami juga menunjukkan, bagaimana objek tambahan dapat ditambahkan ke
adegan pov3d. Untuk pembuatan objek, lihat contoh berikut. Perhatikan
bahwa plot3d menskalakan plot, sehingga cocok dengan kubus satuan.
\end{eulercomment}
\begin{eulerprompt}
>r=linspace(0,1,80); phi=linspace(0,2pi,80)'; ...
>x=r*cos(phi); y=r*sin(phi); z=exp(-5*r)*cos(8*pi*r)/3;  ...
>pov3d(x,y,z,zoom=6,axis=0,height=30°,add=povsphere([0.5,0,0.25],0.15,povlook(red)), ...
>  w=500,h=300);
\end{eulerprompt}
\eulerimg{16}{images/Plot3D_Nandhita Putri Shalsabila-066.png}
\begin{eulercomment}
Dengan metode peneduh canggih Povray, sangat sedikit titik yang dapat
menghasilkan permukaan yang sangat halus. Hanya pada batas-batas dan
dalam bayangan, triknya mungkin terlihat jelas.

Untuk ini, kita perlu menjumlahkan vektor normal di setiap titik
matriks.
\end{eulercomment}
\begin{eulerprompt}
>Z &= x^2*y^3
\end{eulerprompt}
\begin{euleroutput}
  
                                   2  3
                                  x  y
  
\end{euleroutput}
\begin{eulercomment}
Persamaan permukaannya adalah [x,y,Z]. Kami menghitung dua turunan
dari x dan y dan mengambil perkalian silangnya sebagai normal.
\end{eulercomment}
\begin{eulerprompt}
>dx &= diff([x,y,Z],x); dy &= diff([x,y,Z],y);
\end{eulerprompt}
\begin{eulercomment}
Kami mendefinisikan normal sebagai produk silang dari turunan ini, dan
mendefinisikan fungsi koordinat.
\end{eulercomment}
\begin{eulerprompt}
>N &= crossproduct(dx,dy); NX &= N[1]; NY &= N[2]; NZ &= N[3]; N,
\end{eulerprompt}
\begin{euleroutput}
  
                                 3       2  2
                         [- 2 x y , - 3 x  y , 1]
  
\end{euleroutput}
\begin{eulercomment}
Kami hanya menggunakan 25 poin.
\end{eulercomment}
\begin{eulerprompt}
>x=-1:0.5:1; y=x';
>pov3d(x,y,Z(x,y),angle=10°, ...
>  xv=NX(x,y),yv=NY(x,y),zv=NZ(x,y),<shadow);
\end{eulerprompt}
\eulerimg{27}{images/Plot3D_Nandhita Putri Shalsabila-067.png}
\begin{eulercomment}
Berikut ini adalah simpul Trefoil yang dilakukan oleh A. Busser di
Povray. Ada versi yang lebih baik dalam contoh ini.

See: Examples\textbackslash{}Trefoil Knot \textbar{} Trefoil Knot

Untuk tampilan yang bagus dengan poin yang tidak terlalu banyak, kami
menambahkan vektor normal di sini. Kami menggunakan Maxima untuk
menghitung normalnya bagi kami. Pertama, tiga fungsi koordinat sebagai
ekspresi simbolik.
\end{eulercomment}
\begin{eulerprompt}
>X &= ((4+sin(3*y))+cos(x))*cos(2*y); ...
>Y &= ((4+sin(3*y))+cos(x))*sin(2*y); ...
>Z &= sin(x)+2*cos(3*y);
\end{eulerprompt}
\begin{eulercomment}
Kemudian kedua vektor turunan ke x dan y.
\end{eulercomment}
\begin{eulerprompt}
>dx &= diff([X,Y,Z],x); dy &= diff([X,Y,Z],y);
\end{eulerprompt}
\begin{eulercomment}
Sekarang normalnya, yaitu perkalian silang kedua turunannya.
\end{eulercomment}
\begin{eulerprompt}
>dn &= crossproduct(dx,dy);
\end{eulerprompt}
\begin{eulercomment}
Kami sekarang mengevaluasi semua ini secara numerik.
\end{eulercomment}
\begin{eulerprompt}
>x:=linspace(-%pi,%pi,40); y:=linspace(-%pi,%pi,100)';
\end{eulerprompt}
\begin{eulercomment}
Vektor normal adalah evaluasi ekspresi simbolik dn[i] untuk i=1,2,3.
Sintaksnya adalah \&"ekspresi"(parameter). Ini adalah alternatif dari
metode pada contoh sebelumnya, di mana kita mendefinisikan ekspresi
simbolik NX, NY, NZ terlebih dahulu.
\end{eulercomment}
\begin{eulerprompt}
>pov3d(X(x,y),Y(x,y),Z(x,y),>anaglyph,axis=0,zoom=5,w=450,h=350, ...
>  <shadow,look=povlook(blue), ...
>  xv=&"dn[1]"(x,y), yv=&"dn[2]"(x,y), zv=&"dn[3]"(x,y));
\end{eulerprompt}
\eulerimg{21}{images/Plot3D_Nandhita Putri Shalsabila-068.png}
\begin{eulercomment}
Kami juga dapat menghasilkan grid dalam 3D.
\end{eulercomment}
\begin{eulerprompt}
>povstart(zoom=4); ...
>x=-1:0.5:1; r=1-(x+1)^2/6; ...
>t=(0°:30°:360°)'; y=r*cos(t); z=r*sin(t); ...
>writeln(povgrid(x,y,z,d=0.02,dballs=0.05)); ...
>povend();
\end{eulerprompt}
\eulerimg{27}{images/Plot3D_Nandhita Putri Shalsabila-069.png}
\begin{eulercomment}
Dengan povgrid(), kurva dimungkinkan.
\end{eulercomment}
\begin{eulerprompt}
>povstart(center=[0,0,1],zoom=3.6); ...
>t=linspace(0,2,1000); r=exp(-t); ...
>x=cos(2*pi*10*t)*r; y=sin(2*pi*10*t)*r; z=t; ...
>writeln(povgrid(x,y,z,povlook(red))); ...
>writeAxis(0,2,axis=3); ...
>povend();
\end{eulerprompt}
\eulerimg{27}{images/Plot3D_Nandhita Putri Shalsabila-070.png}
\eulerheading{Objek Povray}
\begin{eulercomment}
Di atas, kami menggunakan pov3d untuk memplot permukaan. Antarmuka
povray di Euler juga dapat menghasilkan objek Povray. Objek ini
disimpan sebagai string di Euler, dan perlu ditulis ke file Povray.

Kami memulai output dengan povstart().
\end{eulercomment}
\begin{eulerprompt}
>povstart(zoom=4);
\end{eulerprompt}
\begin{eulercomment}
Pertama kita mendefinisikan tiga silinder, dan menyimpannya dalam
string di Euler.

Fungsi povx() dll. hanya mengembalikan vektor [1,0,0], yang dapat
digunakan sebagai gantinya.
\end{eulercomment}
\begin{eulerprompt}
>c1=povcylinder(-povx,povx,1,povlook(red)); ...
>c2=povcylinder(-povy,povy,1,povlook(yellow)); ...
>c3=povcylinder(-povz,povz,1,povlook(blue)); ...
\end{eulerprompt}
\begin{eulercomment}
String tersebut berisi kode Povray, yang tidak perlu kita pahami pada
saat itu.
\end{eulercomment}
\begin{eulerprompt}
>c2
\end{eulerprompt}
\begin{euleroutput}
  cylinder \{ <0,0,-1>, <0,0,1>, 1
   texture \{ pigment \{ color rgb <0.941176,0.941176,0.392157> \}  \} 
   finish \{ ambient 0.2 \} 
   \}
\end{euleroutput}
\begin{eulercomment}
Seperti yang Anda lihat, kami menambahkan tekstur pada objek dalam
tiga warna berbeda.

Hal ini dilakukan oleh povlook(), yang mengembalikan string dengan
kode Povray yang relevan. Kita dapat menggunakan warna default Euler,
atau menentukan warna kita sendiri. Kita juga dapat menambahkan
transparansi, atau mengubah cahaya sekitar.
\end{eulercomment}
\begin{eulerprompt}
>povlook(rgb(0.1,0.2,0.3),0.1,0.5)
\end{eulerprompt}
\begin{euleroutput}
   texture \{ pigment \{ color rgbf <0.101961,0.2,0.301961,0.1> \}  \} 
   finish \{ ambient 0.5 \} 
  
\end{euleroutput}
\begin{eulercomment}
Sekarang kita mendefinisikan objek persimpangan, dan menulis hasilnya
ke file.
\end{eulercomment}
\begin{eulerprompt}
>writeln(povintersection([c1,c2,c3]));
\end{eulerprompt}
\begin{eulercomment}
Persimpangan tiga silinder sulit untuk divisualisasikan jika Anda
belum pernah melihatnya sebelumnya.
\end{eulercomment}
\begin{eulerprompt}
>povend;
\end{eulerprompt}
\eulerimg{27}{images/Plot3D_Nandhita Putri Shalsabila-071.png}
\begin{eulercomment}
Fungsi berikut menghasilkan fraktal secara rekursif.

Fungsi pertama menunjukkan bagaimana Euler menangani objek Povray
sederhana. Fungsi povbox() mengembalikan string, yang berisi koordinat
kotak, tekstur, dan hasil akhir.

\end{eulercomment}
\begin{eulerprompt}
>function onebox(x,y,z,d) := povbox([x,y,z],[x+d,y+d,z+d],povlook());
>function fractal (x,y,z,h,n) ...
\end{eulerprompt}
\begin{eulerudf}
   if n==1 then writeln(onebox(x,y,z,h));
   else
     h=h/3;
     fractal(x,y,z,h,n-1);
     fractal(x+2*h,y,z,h,n-1);
     fractal(x,y+2*h,z,h,n-1);
     fractal(x,y,z+2*h,h,n-1);
     fractal(x+2*h,y+2*h,z,h,n-1);
     fractal(x+2*h,y,z+2*h,h,n-1);
     fractal(x,y+2*h,z+2*h,h,n-1);
     fractal(x+2*h,y+2*h,z+2*h,h,n-1);
     fractal(x+h,y+h,z+h,h,n-1);
   endif;
  endfunction
\end{eulerudf}
\begin{eulerprompt}
>povstart(fade=10,<shadow);
>fractal(-1,-1,-1,2,4);
>povend();
\end{eulerprompt}
\eulerimg{27}{images/Plot3D_Nandhita Putri Shalsabila-072.png}
\begin{eulercomment}
Perbedaan memungkinkan pemisahan satu objek dari objek lainnya.
Seperti persimpangan, ada bagian dari objek CSG di Povray.
\end{eulercomment}
\begin{eulerprompt}
>povstart(light=[5,-5,5],fade=10);
\end{eulerprompt}
\begin{eulercomment}
Untuk demonstrasi ini, kami mendefinisikan objek di Povray, alih-alih
menggunakan string di Euler. Definisi segera ditulis ke file.

Koordinat kotak -1 berarti [-1,-1,-1].
\end{eulercomment}
\begin{eulerprompt}
>povdefine("mycube",povbox(-1,1));
\end{eulerprompt}
\begin{eulercomment}
Kita bisa menggunakan objek ini di povobject(), yang mengembalikan
string seperti biasa.
\end{eulercomment}
\begin{eulerprompt}
>c1=povobject("mycube",povlook(red));
\end{eulerprompt}
\begin{eulercomment}
Kami membuat kubus kedua, dan memutar serta menskalakannya sedikit.
\end{eulercomment}
\begin{eulerprompt}
>c2=povobject("mycube",povlook(yellow),translate=[1,1,1], ...
>  rotate=xrotate(10°)+yrotate(10°), scale=1.2);
\end{eulerprompt}
\begin{eulercomment}
Lalu kita ambil selisih kedua benda tersebut.
\end{eulercomment}
\begin{eulerprompt}
>writeln(povdifference(c1,c2));
\end{eulerprompt}
\begin{eulercomment}
Sekarang tambahkan tiga sumbu.
\end{eulercomment}
\begin{eulerprompt}
>writeAxis(-1.2,1.2,axis=1); ...
>writeAxis(-1.2,1.2,axis=2); ...
>writeAxis(-1.2,1.2,axis=4); ...
>povend();
\end{eulerprompt}
\eulerimg{27}{images/Plot3D_Nandhita Putri Shalsabila-073.png}
\eulerheading{Fungsi Implisit}
\begin{eulercomment}
Povray dapat memplot himpunan di mana f(x,y,z)=0, seperti parameter
implisit di plot3d. Namun hasilnya terlihat jauh lebih baik.

Sintaks untuk fungsinya sedikit berbeda. Anda tidak dapat menggunakan
keluaran ekspresi Maxima atau Euler.

\end{eulercomment}
\begin{eulerformula}
\[
((x^2+y^2-c^2)^2+(z^2-1)^2)*((y^2+z^2-c^2)^2+(x^2-1)^2)*((z^2+x^2-c^2)^2+(y^2-1)^2)=d
\]
\end{eulerformula}
\begin{eulerprompt}
>povstart(angle=70°,height=50°,zoom=4);
>c=0.1;
>d=0.1;
>writeln(povsurface("(pow(pow(x,2)+pow(y,2)-pow(c,2),2)+pow(pow(z,2)-1,2))*"+"(pow(pow(y,2)+pow(z,2)-pow(c,2),2)+pow(pow(x,2)-1,2))*"+"(pow(pow(z,2)+pow(x,2)-pow(c,2),2)+pow(pow(y,2)-1,2))-d",povlook(red))); ...
>povend();
\end{eulerprompt}
\begin{euleroutput}
  Error : Povray error!
  
  Error generated by error() command
  
  povray:
      error("Povray error!");
  Try "trace errors" to inspect local variables after errors.
  povend:
      povray(file,w,h,aspect,exit); 
\end{euleroutput}
\begin{eulerprompt}
>povstart(angle=25°,height=10°); 
>writeln(povsurface("pow(x,2)+pow(y,2)*pow(z,2)-1",povlook(blue),povbox(-2,2,"")));
>povend();
\end{eulerprompt}
\eulerimg{27}{images/Plot3D_Nandhita Putri Shalsabila-074.png}
\begin{eulerprompt}
>povstart(angle=70°,height=50°,zoom=4);
\end{eulerprompt}
\begin{eulercomment}
Buat permukaan implisit. Perhatikan sintaksis yang berbeda dalam
ekspresi.
\end{eulercomment}
\begin{eulerprompt}
>writeln(povsurface("pow(x,2)*y-pow(y,3)-pow(z,2)",povlook(green))); ...
>writeAxes(); ...
>povend();
\end{eulerprompt}
\eulerimg{27}{images/Plot3D_Nandhita Putri Shalsabila-075.png}
\eulerheading{Objek Jaring}
\begin{eulercomment}
Dalam contoh ini, kami menunjukkan cara membuat objek mesh, dan
menggambarnya dengan informasi tambahan.

Kita ingin memaksimalkan xy pada kondisi x+y=1 dan mendemonstrasikan
sentuhan tangensial garis datar.
\end{eulercomment}
\begin{eulerprompt}
>povstart(angle=-10°,center=[0.5,0.5,0.5],zoom=7);
\end{eulerprompt}
\begin{eulercomment}
Kita tidak dapat menyimpan objek dalam string seperti sebelumnya,
karena terlalu besar. Jadi kita mendefinisikan objek dalam file Povray
menggunakan #declare. Fungsi povtriangle() melakukan ini secara
otomatis. Ia dapat menerima vektor normal seperti pov3d().

Berikut ini definisi objek mesh, dan segera menuliskannya ke dalam
file.
\end{eulercomment}
\begin{eulerprompt}
>x=0:0.02:1; y=x'; z=x*y; vx=-y; vy=-x; vz=1;
>mesh=povtriangles(x,y,z,"",vx,vy,vz);
\end{eulerprompt}
\begin{eulercomment}
Sekarang kita mendefinisikan dua cakram, yang akan berpotongan dengan
permukaan.
\end{eulercomment}
\begin{eulerprompt}
>cl=povdisc([0.5,0.5,0],[1,1,0],2); ...
>ll=povdisc([0,0,1/4],[0,0,1],2);
\end{eulerprompt}
\begin{eulercomment}
Tulis permukaannya dikurangi kedua cakram.
\end{eulercomment}
\begin{eulerprompt}
>writeln(povdifference(mesh,povunion([cl,ll]),povlook(green)));
\end{eulerprompt}
\begin{eulercomment}
Tuliskan kedua perpotongan tersebut.
\end{eulercomment}
\begin{eulerprompt}
>writeln(povintersection([mesh,cl],povlook(red))); ...
>writeln(povintersection([mesh,ll],povlook(gray)));
\end{eulerprompt}
\begin{eulercomment}
Tulis poin maksimal.
\end{eulercomment}
\begin{eulerprompt}
>writeln(povpoint([1/2,1/2,1/4],povlook(gray),size=2*defaultpointsize));
\end{eulerprompt}
\begin{eulercomment}
Tambahkan sumbu dan selesai.
\end{eulercomment}
\begin{eulerprompt}
>writeAxes(0,1,0,1,0,1,d=0.015); ...
>povend();
\end{eulerprompt}
\eulerimg{27}{images/Plot3D_Nandhita Putri Shalsabila-076.png}
\begin{eulercomment}
perintah ini digunakan untuk menggambar sumbu koordinat 3D (x, y, z)
dengan rentang dari 0 hingga 1 pada masing-masing sumbu dan dengan
margin kecil 0.015, lalu menutup blok POV rendering.

\begin{eulercomment}
\eulerheading{Anaglyph di Povray}
\begin{eulercomment}
Untuk menghasilkan anaglyph untuk kacamata merah/cyan, Povray harus
dijalankan dua kali dari posisi kamera berbeda. Ini menghasilkan dua
file Povray dan dua file PNG, yang dimuat dengan fungsi
loadanaglyph().

Tentu saja, Anda memerlukan kacamata berwarna merah/cyan untuk melihat
contoh berikut dengan benar.

Fungsi pov3d() memiliki saklar sederhana untuk menghasilkan anaglyph.
\end{eulercomment}
\begin{eulerprompt}
>pov3d("-exp(-x^2-y^2)/2",r=2,height=45°,>anaglyph, ...
>  center=[0,0,0.5],zoom=3.5);
\end{eulerprompt}
\eulerimg{27}{images/Plot3D_Nandhita Putri Shalsabila-077.png}
\begin{eulercomment}
Jika Anda membuat adegan dengan objek, Anda perlu memasukkan pembuatan
adegan ke dalam fungsi, dan menjalankannya dua kali dengan nilai
berbeda untuk parameter anaglyph.
\end{eulercomment}
\begin{eulerprompt}
>function myscene ...
\end{eulerprompt}
\begin{eulerudf}
    s=povsphere(povc,1);
    cl=povcylinder(-povz,povz,0.5);
    clx=povobject(cl,rotate=xrotate(90°));
    cly=povobject(cl,rotate=yrotate(90°));
    c=povbox([-1,-1,0],1);
    un=povunion([cl,clx,cly,c]);
    obj=povdifference(s,un,povlook(red));
    writeln(obj);
    writeAxes();
  endfunction
\end{eulerudf}
\begin{eulercomment}
Fungsi povanaglyph() melakukan semua ini. Parameternya seperti
gabungan povstart() dan povend().
\end{eulercomment}
\begin{eulerprompt}
>povanaglyph("myscene",zoom=4.5);
\end{eulerprompt}
\eulerimg{27}{images/Plot3D_Nandhita Putri Shalsabila-078.png}
\begin{eulercomment}
perintah ini digunakan untuk menghasilkan tampilan 3D dari adegan yang
bernama "myscene" dengan tingkat pembesaran tertentu, yang dapat
dilihat menggunakan kacamata anaglyph (misalnya merah-biru atau
merah-hijau).

\begin{eulercomment}
\eulerheading{Mendefinisikan Objek sendiri}
\begin{eulercomment}
Antarmuka povray Euler berisi banyak objek. Namun Anda tidak dibatasi
pada hal ini. Anda dapat membuat objek sendiri, yang menggabungkan
objek lain, atau merupakan objek yang benar-benar baru.

Kami mendemonstrasikan torus. Perintah Povray untuk ini adalah
"torus". Jadi kami mengembalikan string dengan perintah ini dan
parameternya. Perhatikan bahwa torus selalu berpusat pada titik asal.
\end{eulercomment}
\begin{eulerprompt}
>function povdonat (r1,r2,look="") ...
\end{eulerprompt}
\begin{eulerudf}
    return "torus \{"+r1+","+r2+look+"\}";
  endfunction
\end{eulerudf}
\begin{eulercomment}
Ini torus pertama kami.
\end{eulercomment}
\begin{eulerprompt}
>t1=povdonat(0.8,0.2)
\end{eulerprompt}
\begin{euleroutput}
  torus \{0.8,0.2\}
\end{euleroutput}
\begin{eulercomment}
Mari kita gunakan objek ini untuk membuat torus kedua, diterjemahkan
dan diputar.
\end{eulercomment}
\begin{eulerprompt}
>t2=povobject(t1,rotate=xrotate(90°),translate=[0.8,0,0])
\end{eulerprompt}
\begin{euleroutput}
  object \{ torus \{0.8,0.2\}
   rotate 90 *x 
   translate <0.8,0,0>
   \}
\end{euleroutput}
\begin{eulercomment}
Sekarang kita tempatkan objek-objek tersebut ke dalam sebuah adegan.
Untuk tampilannya kami menggunakan Phong Shading.
\end{eulercomment}
\begin{eulerprompt}
>povstart(center=[0.4,0,0],angle=0°,zoom=3.8,aspect=1.5); ...
>writeln(povobject(t1,povlook(green,phong=1))); ...
>writeln(povobject(t2,povlook(green,phong=1))); ...
\end{eulerprompt}
\begin{eulercomment}
\textgreater{}povend();

memanggil program Povray. Namun, jika terjadi kesalahan, kesalahan
tersebut tidak ditampilkan. Oleh karena itu Anda harus menggunakan

\end{eulercomment}
\begin{eulerttcomment}
 >povend(<keluar);
\end{eulerttcomment}
\begin{eulercomment}

jika ada yang tidak berhasil. Ini akan membiarkan jendela Povray
terbuka.
\end{eulercomment}
\begin{eulerprompt}
>povend(h=320,w=480);
\end{eulerprompt}
\eulerimg{18}{images/Plot3D_Nandhita Putri Shalsabila-079.png}
\begin{eulercomment}
Berikut adalah contoh yang lebih rumit. Kami memecahkannya

\end{eulercomment}
\begin{eulerformula}
\[
Ax \le b, \quad x \ge 0, \quad c.x \to \text{Max.}
\]
\end{eulerformula}
\begin{eulercomment}
dan menunjukkan titik-titik yang layak dan optimal dalam plot 3D.
\end{eulercomment}
\begin{eulerprompt}
>A=[10,8,4;5,6,8;6,3,2;9,5,6];
>b=[10,10,10,10]';
>c=[1,1,1];
\end{eulerprompt}
\begin{eulercomment}
Pertama, mari kita periksa, apakah contoh ini punya solusinya.
\end{eulercomment}
\begin{eulerprompt}
>x=simplex(A,b,c,>max,>check)'
\end{eulerprompt}
\begin{euleroutput}
  [0,  1,  0.5]
\end{euleroutput}
\begin{eulercomment}
Ya, sudah.

Selanjutnya kita mendefinisikan dua objek. Yang pertama adalah pesawat

\end{eulercomment}
\begin{eulerformula}
\[
a \cdot x \le b
\]
\end{eulerformula}
\begin{eulerprompt}
>function oneplane (a,b,look="") ...
\end{eulerprompt}
\begin{eulerudf}
    return povplane(a,b,look)
  endfunction
\end{eulerudf}
\begin{eulercomment}
Kemudian kita tentukan perpotongan semua setengah ruang dan kubus.
\end{eulercomment}
\begin{eulerprompt}
>function adm (A, b, r, look="") ...
\end{eulerprompt}
\begin{eulerudf}
    ol=[];
    loop 1 to rows(A); ol=ol|oneplane(A[#],b[#]); end;
    ol=ol|povbox([0,0,0],[r,r,r]);
    return povintersection(ol,look);
  endfunction
\end{eulerudf}
\begin{eulercomment}
Sekarang kita dapat merencanakan adegannya.
\end{eulercomment}
\begin{eulerprompt}
>povstart(angle=120°,center=[0.5,0.5,0.5],zoom=3.5); ...
>writeln(adm(A,b,2,povlook(green,0.4))); ...
>writeAxes(0,1.3,0,1.6,0,1.5); ...
\end{eulerprompt}
\begin{eulercomment}
Berikut ini adalah lingkaran di sekitar optimal.
\end{eulercomment}
\begin{eulerprompt}
>writeln(povintersection([povsphere(x,0.5),povplane(c,c.x')], ...
>  povlook(red,0.9)));
\end{eulerprompt}
\begin{eulercomment}
Dan kesalahan ke arah optimal.
\end{eulercomment}
\begin{eulerprompt}
>writeln(povarrow(x,c*0.5,povlook(red)));
\end{eulerprompt}
\begin{eulercomment}
Kami menambahkan teks ke layar. Teks hanyalah objek 3D. Kita perlu
menempatkan dan memutarnya sesuai dengan pandangan kita.
\end{eulercomment}
\begin{eulerprompt}
>writeln(povtext("Linear Problem",[0,0.2,1.3],size=0.05,rotate=5°)); ...
>povend();
\end{eulerprompt}
\eulerimg{27}{images/Plot3D_Nandhita Putri Shalsabila-080.png}
\begin{eulercomment}
perintah ini digunakan untuk menempatkan teks "Linear Problem" pada
posisi tertentu dalam grafik atau ruang 3D, dengan ukuran dan rotasi
tertentu, dan kemudian mengakhiri proses render gambar

\begin{eulercomment}
\eulerheading{Lebih Banyak Contoh}
\begin{eulercomment}
Anda dapat menemukan beberapa contoh Povray di Euler di file berikut.

See: Examples/Dandelin Spheres\\
See: Examples/Donat Math\\
See: Examples/Trefoil Knot\\
See: Examples/Optimization by Affine Scaling

\begin{eulercomment}
\eulerheading{Latihan Soal}
\begin{eulercomment}
1. buatlah gabungan 2 silinder dengan fungsi povx() berwarna merah dan
povz() berwarna kuning dan zoom 4

\end{eulercomment}
\begin{eulerprompt}
>povstart(zoom=4)
>c1 = povcylinder (-povx,povx,1,povlook(red));
>c2 = povcylinder (-povz,povz,1,povlook(yellow));
>writeln (povintersection([c1,c2]));
>povend()
\end{eulerprompt}
\eulerimg{27}{images/Plot3D_Nandhita Putri Shalsabila-081.png}
\end{eulernotebook}
\end{document}
